% !Mode:: "TeX:UTF-8"
%% \usepackage{metalogo}
%% \usepackage{doc}
\usepackage{ctex}
\usepackage[english]{babel}
%% \usepackage[french]{babel}

\usepackage{scrextend}
%% part of KOMA-Script) which allows to choose arbitrary font sizes like 7.5pt -- be sure to use a vector font like Latin Modern

% 汉化
\renewcommand{\contentsname}{目\qquad 录}
\renewcommand\listfigurename{插\ 图\ 目\ 录}
\renewcommand\listtablename{表\ 格\ 目\ 录}
\renewcommand\bibname{参\ 考\ 文\ 献}
\renewcommand{\figurename}{图}
\renewcommand{\tablename}{表}

\usepackage{amsmath, amsthm, amssymb}

%图形
\usepackage{graphicx}
\usepackage{pstricks}
\usepackage{graphicx}
\usepackage{subfigure}
\usepackage{tikz} %pdf figure
\usetikzlibrary{positioning}
\def\pgfsysdriver{pgfsys-dvipdfmx.def}
\pgfsetxvec{\pgfpoint{10pt}{0}}
\pgfsetyvec{\pgfpoint{0}{10pt}}
\tikzset{box/.style={rectangle, rounded corners=6pt, minimum width=50pt, minimum height=20pt, inner sep=6pt, draw=gray,thick, fill=lightgray}}
\usepackage{ccaption}
\usepackage{epstopdf} %convert eps to pdf

%% bmp, gif not supported
\DeclareGraphicsExtensions{.eps,.mps,.pdf,.jpg,.png}
\DeclareGraphicsRule{*}{eps}{*}{}
\graphicspath{{figure/}{../figure/}}

%% \setbeameroption{show notes on second screen = left} needs the following package
\usepackage{pgfpages}

% code
%% verbatim 中不能使用\textbf 等格式, alltt可以使用
\usepackage{alltt}

\usepackage{listings}
\lstset{
    backgroundcolor=\color{white},
%%     basicstyle=\wuhao\ttfamily,
    columns=flexible,
    breakatwhitespace=false,
    breaklines=true,
    captionpos=n, % none
%%     captionpos=b, % bottom
%%     frame=single,
    numbers=left,
    numbersep=5pt,
    showspaces=false,
    showstringspaces=false,
    showtabs=false,
    stepnumber=1,
    rulecolor=\color{black},
    tabsize=4,
    texcl=true,
    title=\lstname,
    escapeinside={\%*}{*)},
    extendedchars=false,
    mathescape=true,
    xleftmargin=3em,
    xrightmargin=3em,
    numberstyle=\color{gray},
    keywordstyle=\color{blue},
    commentstyle=\color{dkgreen},
    stringstyle=\color{mauve},
	mathescape=false,
}

% 编译通不过, 加了CJKbookmarks=true 之后可以使用中文
\hypersetup{colorlinks,linkcolor=blue,urlcolor=green,anchorcolor=red,citecolor=green,CJKbookmarks=true}

%表格
%allow the use of \toprule, \midrule, and \bottomrule
\usepackage{booktabs}
\usepackage{tabularx}
\usepackage{multirow}
\usepackage{colortbl}
\usepackage{longtable}
\usepackage{diagbox}

%\usepackage{dirtree}  %目录结构图

%颜色
\usepackage{color}
\def\red#1{\textcolor[rgb]{1.00,0.00,0.00}{#1}}
\def\yellow#1{\textcolor[rgb]{1.00,1.00,0.00}{#1}}

%% 化学
%% \usepackage[version=3]{mhchem}
%% ex: Ammonium sulphate is \ce{(NH4)2SO4}.

\usepackage{fixltx2e}
%% 默认情况下只能使用\textsuperscript,加了这个package之后可以使用\textsubscript

%Math
%% Macro
\def\vecteur#1{(#1_1,~#1_2,~\ldots,~#1_n)}
%% \def\vecteur#1{\ensuremath{(#1_1,~#1_2,~\ldots,~#1_n)}}

\newcommand{\R}{\mathbb{R}}   %the real number set
%% \newcommand{\C}{\mathbb{C}} %xelatex says \C has already been used.
\newcommand{\N}{\mathbb{N}}
\newcommand{\Z}{\mathbb{Z}}
\newcommand{\Q}{\mathbb{Q}}

%% \newcommand{\si}{\textrm{ si }}
%% \newcommand{\sinon}{\textrm{ si non}}
%% \newcommand{\et}{\textrm{ et }}
%% \newcommand{\ou}{\textrm{ ou }}
%% \newcommand{\non}{\textrm{non }}
%% \newcommand{\ssi}{si et seulement si }
\newcommand{\si}{\textrm{ if }}
\newcommand{\sinon}{\textrm{ else }}
\newcommand{\et}{\textrm{ and }}
\newcommand{\ou}{\textrm{ or }}
\newcommand{\non}{\textrm{no }}
\newcommand{\ssi}{if and only if }
\newcommand{\infinity}{\infty}

%定义运算符
\DeclareMathOperator{\arccot}{arcot}
\DeclareMathOperator{\arcth}{arcth}
\DeclareMathOperator{\arcsh}{arcsh}
\DeclareMathOperator{\arch}{arch}
\DeclareMathOperator{\ch}{ch}
\DeclareMathOperator{\dth}{th} %\th 已经被定义了
\DeclareMathOperator{\sh}{sh}

\newcommand*\laplace{\mathop{}\!\mathbin\bigtriangleup}
\newcommand*\dalambert{\mathop{}\!\mathbin\Box}
\newcommand{\grad}[1]{\nabla #1}
\newcommand{\gradien}[1]{\nabla #1}
\newcommand{\divergence}[1]{\nabla \cdot #1}
\newcommand{\rotationnel}[1]{\nabla \times #1}
\newcommand{\rot}[1]{\nabla \times #1}

\newcommand{\stcomp}[1]{\overline{#1}} % set complement

\usepackage{xspace}
%produit scalaire
\newcommand{\ps}[2]{\ensuremath{\langle #1 , #2\rangle}\xspace}

%use \lasteq to reference the last equation
\newcommand\lasteq{(\theequation)}
\newcommand{\eqspace}{\hspace{0.5cm}}
\newcommand{\norm}[1]{\left\Vert #1\right\Vert}
%% insert text in math mode being treated as normal text
\newcommand{\eqnote}[1]{\text{ #1 }}
%% generate a fraction but without the line
\newcommand\mytop[2]{\genfrac{}{}{0pt}{}{#1}{#2}}

%% 特殊符号
%% \usepackage{pifont}
%% \ding{number}, 通过number 来调用不同的符号

\mode<presentation> {
%% \usetheme{default}
%% \usetheme{AnnArbor}
%% \usetheme{Antibes}
%% \usetheme{Bergen}
%% \usetheme{Berkeley}
%% \usetheme{Berlin}
%% \usetheme{Boadilla}
%% \usetheme{CambridgeUS}
%% \usetheme{Copenhagen}
\usetheme{Darmstadt}
%% \usetheme{Dresden}
%% \usetheme{Frankfurt}
%% \usetheme{Goettingen}
%% \usetheme{Hannover}
%% \usetheme{Ilmenau}
%% \usetheme{JuanLesPins}
%% \usetheme{Luebeck}
%% \usetheme{Madrid}
%% \usetheme{Malmoe}
%% \usetheme{Marburg}
%% \usetheme{Montpellier}
%% \usetheme{PaloAlto}
%% \usetheme{Pittsburgh}
%% \usetheme{Rochester}
%% \usetheme{Singapore}
%% \usetheme{Szeged}
%% \usetheme{Warsaw}

%% \usecolortheme{albatross}
%% \usecolortheme{beaver}
%% \usecolortheme{beetle}
%% \usecolortheme{crane}
%% \usecolortheme{dolphin}
%% \usecolortheme{dove}
%% \usecolortheme{fly}
%% \usecolortheme{lily}
%% \usecolortheme{orchid}
%% \usecolortheme{rose}
%% \usecolortheme{seagull}
\usecolortheme{seahorse}
%% \usecolortheme{whale}
%% \usecolortheme{wolverine}

\usefonttheme{professionalfonts}
\useinnertheme{rectangles}

%% \setbeamertemplate{footline} % To remove the footer line in all slides uncomment this line
\setbeamertemplate{footline}[page number] % To replace the footer line in all slides with a simple slide count uncomment this line

\setbeamertemplate{navigation symbols}{} % To remove the navigation symbols from the bottom of all slides uncomment this line

%% set the subbody and subsubbody font size
\setbeamerfont{itemize/enumerate body}{size=\normalsize}
\setbeamerfont{itemize/enumerate subbody}{size=\footnotesize}
\setbeamerfont{itemize/enumerate subsubbody}{size=\scriptsize}

%% set itemize leading symbol
\setbeamertemplate{itemize item}{\normalsize \hbox{\donotcoloroutermaths$\blacktriangleright$}}
\setbeamertemplate{itemize subitem}{\footnotesize \hbox{\donotcoloroutermaths$\diamondsuit$}}
\setbeamertemplate{itemize subsubitem}{\scriptsize \hbox{\donotcoloroutermaths$\bullet$}}

%% set enumerate leading symbol
\setbeamertemplate{enumerate item}{\insertenumlabel.}
\setbeamertemplate{enumerate subitem}{\insertenumlabel.\insertsubenumlabel}
\setbeamertemplate{enumerate subsubitem}{\insertenumlabel.\insertsubenumlabel.\insertsubsubenumlabel}
\setbeamertemplate{enumerate mini template}{\insertenumlabel}

%% 设置页脚字体和背景颜色
%% \setbeamercolor{footcolor}{fg=black,bg=red!40}

\setbeamertemplate{footline}{%
  \leavevmode%
  \hbox{%
%%   \begin{beamercolorbox}[wd=.333333\paperwidth,ht=2.25ex,dp=1ex,center]{author in head/foot}%
%%     \usebeamerfont{author in head/foot}\insertshortauthor~~(\insertshortinstitute)
%%   \end{beamercolorbox}%
%%   \begin{beamercolorbox}[wd=.333333\paperwidth,ht=2.25ex,dp=1ex,center]{title in head/foot}%
%%     \usebeamerfont{title in head/foot}\insertshorttitle
%%   \end{beamercolorbox}%
    \begin{beamercolorbox}[wd=.126\paperwidth,ht=2.25ex,dp=2ex,right]{footcolor}%
%%        \insertframenumber{} / \inserttotalframenumber\hspace*{5ex} %% 当前页码/总页码
       \insertframenumber{}\hspace*{5ex} %%	只显示当前页码
    \end{beamercolorbox}}%
  \vskip0pt%
}
}

