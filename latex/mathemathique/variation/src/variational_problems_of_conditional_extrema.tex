% !Mode:: "TeX:UTF-8"
\section{条件极值的变分问题}
\subsection{完整约束的变分问题}
本节主要研究
\begin{equation}
J[y]=\int_{x_0}^{x_1}F(x,y_1,y_2, \cdots , y_n,y_1',y_2', \cdots , y_n')dx
\label{equation.fonctionGenerale}
\end{equation}
在约束条件
\begin{equation}
\varphi_i(x,y_1,y_2, \cdots , y_n)=0 ~ (i=0,1,2,\cdots ,m;m<n)
\label{equation.condition.constraint}
\end{equation}
及边界条件
\begin{equation}
y_j(x_0)=y_{jo},y_j(x_1)=y_{j1}~ (j=1,2,\cdots ,n)
\label{equation.condition.limit}
\end{equation}
的极值问题
\begin{theorem}
拉格朗日定理:在完整约束~\ref{equation.condition.constraint}和边界约束~\ref{equation.condition.limit}下使目标泛函~\ref{equation.fonctionGenerale}取得极值,则存在待定函数$\lambda_i(x)$,使得函数$y_1,y_2,\cdots,y_n$满足由下列泛函
\begin{equation}
J^{*}[y]=\int_{x_0}^{x_1}[F+\sum_{i=1}^{m}\lambda_i(x)\varphi_i]dx=\int_{x_0}^{x_1}Hdx
\label{equation.fonctionGenerale.help}
\end{equation}
所给出的欧拉方程组
$$H_{y_j} - \frac{d}{dx}H_{y_{j}'}=0 ~ (j=1,2,\cdots,n)$$
方程~\ref{equation.fonctionGenerale.help}称为辅助泛函
在对~\ref{equation.fonctionGenerale.help}进行变分运算时,应把$y_i,y_i',\lambda_i(x)$都看作是泛函$J^{*}[y]$的独立函数
\end{theorem}
微分,等周等其他约束也可以采用拉格朗日乘数法解决
