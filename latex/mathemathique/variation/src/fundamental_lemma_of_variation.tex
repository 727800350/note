% !Mode:: "TeX:UTF-8"
\section{Fundamental lemma of variation}
\begin{theorem}
    Soit une fonction $f(x)$ est continue sur $[a,b]$,$ \forall \eta(x)$ sur $\in C^{n}[a,b]$,et $\exists m>0$,tel que $\eta^{k}(a)=\eta^{k}(b)=0 (\forall k \in [[0,n]])$ \newline
    si$ \int_{a}^{b}f(x)\eta(x)=0$ est toujours vrai, \newline
    alors, sur $[a,b],f(x)\equiv 0$ \newline
    (证明时用反证法,且需要构造$\eta(x)$为特殊的形式)
    $\exists \xi,f(\xi)>0 \Rightarrow \exists $une intervale $[a_0,b_0]$,tel que sur $[a_0,b_0],f(x)$ est positive \newline
 on peut construire $\eta(x)$ comme cidessous:
$\eta(x)=
    \left\{
      \begin{array}{ll}
        [(x-a_0)(x-b_0)]^{2n+2} & x \in [a_0,b_0] \\
        0 & sinon
      \end{array}
    \right.
    $
\end{theorem}


\textbf{Quand m=n=0}
\begin{theorem}
    Soit une fonction $f(x)$ est continue sur $[a,b]$,$ \forall \eta(x)$ sur $\in C^{0}[a,b]$,et $\exists m>0$,tel que $\eta(a)=\eta(b)=0$ \newline
    si$ \int_{a}^{b}f(x)\eta(x)=0$ est toujours vrai, \newline
    alors, sur $[a,b],f(x)\equiv 0$
\end{theorem}

\begin{theorem}
    Soit une fonction $f(x)$ est continue sur $[a,b]$,$ \forall \eta(x)$ sur $\in C^{1}[a,b]$,et $\eta(a)=\eta(b)=0$ \newline
    si$ \int_{a}^{b}f(x)\eta'(x)=0$ est toujours vrai, \newline
    alors, sur $[a,b],f(x)\equiv Constante$
\end{theorem}

\begin{theorem}
    Soit une fonction $f(x,y)$ est continue sur $D$,$ \forall \eta(x,y)$ sur $\in C^{1}[a,b]$,et 在D的边界上, $\eta(x,y)=0$\newline
    si$ \int_{a}^{b}f(x,y)\eta(x,y)=0$ est toujours vrai, \newline
    alors, sur $D,f(x,y)\equiv 0$
\end{theorem}
