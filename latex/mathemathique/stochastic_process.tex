% !Mode:: "TeX:UTF-8"
\documentclass{article}
% !Mode:: "TeX:UTF-8"
\usepackage[english]{babel}
\usepackage[UTF8]{ctex}
\usepackage{amsmath, amsthm, amssymb}

% Figure
\usepackage{graphicx}
\usepackage{float} %% H can fix the location
\usepackage{caption}
\usepackage[format=hang,singlelinecheck=0,font={sf,small},labelfont=bf]{subfig}
\usepackage[noabbrev]{cleveref}
\captionsetup[subfigure]{subrefformat=simple,labelformat=simple,listofformat=subsimple}
\renewcommand\thesubfigure{(\alph{subfigure})}

\usepackage{epstopdf} %% convert eps to pdf
\DeclareGraphicsExtensions{.eps,.mps,.pdf,.jpg,.png} %% bmp, gif not supported
\DeclareGraphicsRule{*}{eps}{*}{}
\graphicspath{{img/}{figure/}{../figure/}} %% fig directorys

%% \usepackage{pstricks} %% a set of macros that allow the inclusion of PostScript drawings directly inside TeX or LaTeX code
%% \usepackage{wrapfig} %% Wrapping text around figures

% Table
\usepackage{booktabs} %% allow the use of \toprule, \midrule, and \bottomrule
\usepackage{tabularx}
\usepackage{multirow}
\usepackage{colortbl}
\usepackage{longtable}
\usepackage{supertabular}

\usepackage[colorinlistoftodos]{todonotes}

% Geometry
\usepackage[paper=a4paper, top=1.5cm, bottom=1.5cm, left=1cm, right=1cm]{geometry}
%% \usepackage[paper=a4paper, top=2.54cm, bottom=2.54cm, left=3.18cm, right=3.18cm]{geometry} %% ms word
%% \usepackage[top=0.1cm, bottom=0.1cm, left=0.1cm, right=0.1cm, paperwidth=9cm, paperheight=11.7cm]{geometry} %% kindle

% Code
%% \usepackage{alltt} %% \textbf can be used in alltt, but not in verbatim

\usepackage{listings}
\lstset{
    backgroundcolor=\color{white},
    columns=flexible,
    breakatwhitespace=false,
    breaklines=true,
    captionpos=tt,
    frame=single, %% Frame: show a box around, possible values are: none|leftline|topline|bottomline|lines|single|shadowbox
    numbers=left, %% possible values are: left, right, none
    numbersep=5pt,
    showspaces=false,
    showstringspaces=false,
    showtabs=false,
    stepnumber=1, %% interval of lines to display the line number
    rulecolor=\color{black},
    tabsize=2,
    texcl=true,
    title=\lstname,
    escapeinside={\%*}{*)},
    extendedchars=false,
    mathescape=true,
    xleftmargin=3em,
    xrightmargin=3em,
    numberstyle=\color{gray},
    keywordstyle=\color{blue},
    commentstyle=\color{green},
    stringstyle=\color{red},
}

% Reference
%% \bibliographystyle{plain} % reference style

% Color
\usepackage[colorlinks, linkcolor=blue, anchorcolor=red, citecolor=green, CJKbookmarks=true]{hyperref}
\usepackage{color}
\def\red#1{\textcolor[rgb]{1.00,0.00,0.00}{#1}}
\newcommand\warning[1]{\red{#1}}

% Other
%% \usepackage{fixltx2e} %% for use of \textsubscript
%% \usepackage{dirtree}  %% directory structure, like the result of command tree in bash shell


% !Mode:: "TeX:UTF-8"
%+++++++++++++++++++++++++++++++++++article+++++++++++++++++++++++++++++++++
%customize the numbering of equation, to make it like section-subsection-equation style, for example,1-2-3
\makeatletter\@addtoreset{equation}{subsection}\makeatother
\renewcommand\theequation{%
\thepart\arabic{section}%
-\thepart\arabic{subsection}%
-\thepart\arabic{equation}%
}
%theorem
\newtheorem{definition}{D\'efintion} %% 整篇文章的全局编号
\newtheorem*{thmwn}{Thm} %% without numbers
\newtheorem{theorem}{Th\'eor\`eme}[section] %% 从属于section编号
\newtheorem{corollary}{Corollary}[theorem] %% 从属于theorem编号
\newtheorem{lemma}{Lemma}
\newtheorem{proposition}{Proposition}[section]
\newtheorem{example}{Example}
\newtheorem*{attention}{Attention}
\newtheorem*{note}{Note}
\newtheorem*{remark}{Remark}
\newtheorem{question}{Question}[section]
\newtheorem{problem}{Problem}
\newtheorem{fact}{Fact}


\begin{document}
\title{随机过程}
\author{Common}
%% \maketitle
%% \newpage
%% \tableofcontents
%% \newpage

一族无穷多个,相互有关的随机变量,这就是随机过程

在概率论中,我们研究了用一个或有限多个随机变量来描述的随机现象.
然而对有些现象还需要研究它的发展变化过程,这类现象若仅用一个或有限多个随机变量描述它,就不能揭示其全部统计规律性,于是出现了随机过程的理论.
\begin{example}
用$X(t)$表示在每天的$t$时刻$t \in [a, b]$的气温.对固定的$t$,
表示一个随机变量,当取遍$t \in [a,b ]$时,得到一族随机变量$\{X(t); t \in T\}$
\end{example}

$X(\cdot, \omega)$ 是一个关于参数$t \in T$ 的函数, 通常称为样本函数,或称随机过程的一次实现.\\
所有样本函数的集合确定一随机过程

随机过程$\{X(t);t \in T\}$ 可能取值的全体所构成的集合称为此随机过程的状态空间,记作$S$.\\
$S$中的元素称为状态.状态空间可以由复数,实数或更一般的抽象空间构成.

\begin{example}
抛掷一枚硬币,样本空间为$\Omega = \{T, H\}$ ,借此定义
$$
X(t)=
\left\{
  \begin{array}{ll}
   \cos(\pi t) , & \hbox{Head;} \\
    2t, & \hbox{Tail.}
  \end{array}
\right.
$$
其中$P(H)=P(T)=1/2$, 则$\{X(t), t \in (-\infty, \infty)\}$是一随机过程.

样本函数为$\cos(\pi t)$和$2t$, 由于样本函数的取值:
$$\cos(\pi t)\ \in [-1,1], 2t \in (-\infty, \infty)$$
所以状态空间
$$S=(-\infty, \infty)$$
\end{example}

\section{相关函数和协方差函数}
\href{http://202.117.122.42:9001/xhxt/xhyxt/xuexi/chart9/c\_9\_3\_2\_001.htm}{相关函数和协方差函数}

\textbf{自相关函数(Autocorrelation function)}\\
自相关函数是描述随机信号$X(k)$在任意两个不同时刻$k_1,k_2$的取值$X(k_1) $和$X(k_2) $之间的相关程度
$$
R_X(k_1, k_2)
= E[X(k_1) X(k_2)]
= \int_{-\infty}^{\infty} \int_{-\infty}^{\infty} x_1 x_2 f(x_1, x_2, k_1, k_2)dx_1 dx_2
$$
$f(x_1, x_2, k_1, k_2)$ 为$X$在$k_1$时刻取$x_1$, $k_2$时刻取$x_2$的概率密度函数.

\textbf{自协方差函数(Autocovariance function)}\\
同理,自协方差函数是描述随机信号$X(k)$在任意两个不同时刻k1,k2的取值\textbf{起伏变化之间的相关程度}
$$ C_X(k_1,k_2) = E[(X(k_1) - \mu_X(k_1)) (X(k_2) - \mu_X(k_2))] $$
$\mu_X(k_1)$ 为$X(k)$在$k_1$时刻的预期值, 然后$(X(k_1) - \mu_X(k_1))$就为$X(k)$在$k_1$时刻的起伏变化.

$$
\begin{aligned}
C_X(k_1,k_2)
& = E[(X(k_1) - \mu_X(k_1)) (X(k_2) - \mu_X(k_2))] \\
& = E[(X(k_1) X(k_2) - \mu_X(k_1)(X(k_2) - X(k_1)\mu_X(k_2) + \mu_X(k_1) \mu_X(k_2)] \\
& = E[(X(k_1) X(k_2)] - \mu_X(k_1)E[(X(k_2)] - E[X(k_1)]\mu_X(k_2) + \mu_X(k_1) \mu_X(k_2) \\
& = E[(X(k_1) X(k_2)] - \mu_X(k_1) \mu_X(k_2) - \mu_X(k_1) \mu_X(k_2) + \mu_X(k_1) \mu_X(k_2) \\
& = E[(X(k_1) X(k_2)] - \mu_X(k_1) \mu_X(k_2)\\
& = R_X(k_1, k_2) - \mu_X(k_1) \mu_X(k_2)
\end{aligned}
$$
\textbf{互相关函数(Cross correlation function)}\\
类似于连续随机信号的情况,两个随机序列 之间的相关程度由互相关函数和互协方差函数描述.两个随机序列的互相关函数为
$$ R_{XY}(k_1, k_2) = E[X(k_1) Y(k_2)]$$

\textbf{互协方差函数(Cross covariance function)}\\
两个随机序列 之间的互协方差函数为
$$ C_{XY}(k_1,k_2) = E[(X(k_1) - \mu_X(k_1)) (Y(k_2) - \mu_Y(k_2))] = R_{XY}(k_1,k_2) - \mu_X(k_1) \mu_Y(k_2) $$

如果两个随机过程 $\{X(t); t \in T\}$ 和 $\{Y(t); t \in T\}$,
对于任意的两个参数$s,t \in T$, 有 $$ C_{XY}(s,t) = 0 $$
则称随机过程 $\{X(t); t \in T\}$ 和 $\{Y(t); t \in T\}$是统计不相关的或不相关的.\\
因为若随机过程 $\{X(t); t \in T\}$ 和 $\{Y(t); t \in T\}$是不相关的, 那么
$$ E[X(k_1) Y(k_2)] = E[X(k_1)] E[Y(k_2)] $$
即
$$R_X(k_1, k_2) = \mu_X(k_1) \mu_X(k_2)$$

有一类重要的随机过程,它处于某种概率平衡状态其主要性质只与变量$X(t)$之间的时间间隔有关而与我们考查的起始点无关, 
或者说,整个随机过程的统计特性不随时间的推移而变化.这类过程叫做平稳过程.

\begin{example}
例如,飞机在某一水平高度h上飞行时,由于受到气流的影响,
实际飞行高度H(t)总是在理论设计高度h水平上下随机波动,此时飞机的实际飞行高度H(t)是一个随机过程,
显然此过程可看作不随机推移面变化的过程,这个随机过程,我们把它看作是平衡的随机过程
\end{example}

此外当我们知道一个随机过程是平稳过程时,它应不随时间的推移而变幻无常.
例如当我们要测定一个电阻的热噪声的统计特性,由于它是平稳过程,因而我们在任何时间进行测试都能得到相同的结果

\end{document}
