% !Mode:: "TeX:UTF-8"
\documentclass{article}
% !Mode:: "TeX:UTF-8"
\usepackage[english]{babel}
\usepackage[UTF8]{ctex}
\usepackage{amsmath, amsthm, amssymb}

% Figure
\usepackage{graphicx}
\usepackage{float} %% H can fix the location
\usepackage{caption}
\usepackage[format=hang,singlelinecheck=0,font={sf,small},labelfont=bf]{subfig}
\usepackage[noabbrev]{cleveref}
\captionsetup[subfigure]{subrefformat=simple,labelformat=simple,listofformat=subsimple}
\renewcommand\thesubfigure{(\alph{subfigure})}

\usepackage{epstopdf} %% convert eps to pdf
\DeclareGraphicsExtensions{.eps,.mps,.pdf,.jpg,.png} %% bmp, gif not supported
\DeclareGraphicsRule{*}{eps}{*}{}
\graphicspath{{img/}{figure/}{../figure/}} %% fig directorys

%% \usepackage{pstricks} %% a set of macros that allow the inclusion of PostScript drawings directly inside TeX or LaTeX code
%% \usepackage{wrapfig} %% Wrapping text around figures

% Table
\usepackage{booktabs} %% allow the use of \toprule, \midrule, and \bottomrule
\usepackage{tabularx}
\usepackage{multirow}
\usepackage{colortbl}
\usepackage{longtable}
\usepackage{supertabular}

\usepackage[colorinlistoftodos]{todonotes}

% Geometry
\usepackage[paper=a4paper, top=1.5cm, bottom=1.5cm, left=1cm, right=1cm]{geometry}
%% \usepackage[paper=a4paper, top=2.54cm, bottom=2.54cm, left=3.18cm, right=3.18cm]{geometry} %% ms word
%% \usepackage[top=0.1cm, bottom=0.1cm, left=0.1cm, right=0.1cm, paperwidth=9cm, paperheight=11.7cm]{geometry} %% kindle

% Code
%% \usepackage{alltt} %% \textbf can be used in alltt, but not in verbatim

\usepackage{listings}
\lstset{
    backgroundcolor=\color{white},
    columns=flexible,
    breakatwhitespace=false,
    breaklines=true,
    captionpos=tt,
    frame=single, %% Frame: show a box around, possible values are: none|leftline|topline|bottomline|lines|single|shadowbox
    numbers=left, %% possible values are: left, right, none
    numbersep=5pt,
    showspaces=false,
    showstringspaces=false,
    showtabs=false,
    stepnumber=1, %% interval of lines to display the line number
    rulecolor=\color{black},
    tabsize=2,
    texcl=true,
    title=\lstname,
    escapeinside={\%*}{*)},
    extendedchars=false,
    mathescape=true,
    xleftmargin=3em,
    xrightmargin=3em,
    numberstyle=\color{gray},
    keywordstyle=\color{blue},
    commentstyle=\color{green},
    stringstyle=\color{red},
}

% Reference
%% \bibliographystyle{plain} % reference style

% Color
\usepackage[colorlinks, linkcolor=blue, anchorcolor=red, citecolor=green, CJKbookmarks=true]{hyperref}
\usepackage{color}
\def\red#1{\textcolor[rgb]{1.00,0.00,0.00}{#1}}
\newcommand\warning[1]{\red{#1}}

% Other
%% \usepackage{fixltx2e} %% for use of \textsubscript
%% \usepackage{dirtree}  %% directory structure, like the result of command tree in bash shell


% !Mode:: "TeX:UTF-8"
%+++++++++++++++++++++++++++++++++++article+++++++++++++++++++++++++++++++++
%customize the numbering of equation, to make it like section-subsection-equation style, for example,1-2-3
\makeatletter\@addtoreset{equation}{subsection}\makeatother
\renewcommand\theequation{%
\thepart\arabic{section}%
-\thepart\arabic{subsection}%
-\thepart\arabic{equation}%
}
%theorem
\newtheorem{definition}{D\'efintion} %% 整篇文章的全局编号
\newtheorem*{thmwn}{Thm} %% without numbers
\newtheorem{theorem}{Th\'eor\`eme}[section] %% 从属于section编号
\newtheorem{corollary}{Corollary}[theorem] %% 从属于theorem编号
\newtheorem{lemma}{Lemma}
\newtheorem{proposition}{Proposition}[section]
\newtheorem{example}{Example}
\newtheorem*{attention}{Attention}
\newtheorem*{note}{Note}
\newtheorem*{remark}{Remark}
\newtheorem{question}{Question}[section]
\newtheorem{problem}{Problem}
\newtheorem{fact}{Fact}


\begin{document}
\title{Alg\`ebre}
\maketitle
\tableofcontents
\newpage
\textbf{Espace m\'etrique}
On appelle $(E, d)$ un espace m\'etrique si $ E$  est un ensemble et d une distance sur $E$ .

\textbf{Espace complet}
Un espace m\'etrique $ M$  est dit complet si toute suite de Cauchy de $ M$  a une limite dans $ M$  (c'est-\`a-dire qu'elle converge dans $M$ ).\newline
Intuitivement, un espace est complet s'il n'a pas de trou, s'il n'a aucun point manquant. \newline
Par exemple, les nombres rationnels ne forment pas un espace complet, puisque $\sqrt{2}$ n'y figure pas alors qu'il existe une suite de Cauchy de nombres rationnels ayant cette limite.

Il est toujours possible de remplir les trous amenant ainsi \`a la compl\'etion d'un espace donn\'e.

\textbf{Espace euclidien}
 il est d\'efini par la donn\'ee d'un espace vectoriel sur le corps des r\'eels, de dimension finie, muni d'un produit scalaire, qui permet de mesurer distances et angles.

\textbf{Espace hermitien}
En math\'ematiques, un espace hermitien est un espace vectoriel sur le corps commutatif des complexes de dimension finie et muni d'un produit scalaire.

La g\'eom\'etrie d'un tel espace est analogue \`a celle d'un espace euclidien

Une forme hermitienne est une application d\'efini\'e sur E×E \`a valeur dans $\mathbf{C}$ not\'ee $\langle .,.\rangle$, telle que :
\begin{itemize}
		\item pour tout y fix\'e l'application $x \mapsto \langle x,y\rangle $est $\mathbf{C}$-lin\'eaire et
		\item $\forall x,y \in E$,$\langle x,y\rangle=\overline{ \langle y,x\rangle}$.
\end{itemize}
En particulier, $\langle x,x\rangle$ est r\'eel, et $x\mapsto \langle x,x\rangle$ est une forme quadratique sur E vu comme $\mathbf{R}$-espace vectoriel.

\textbf{Espace pr\'ehilbertien}
En math\'ematiques, un espace pr\'ehilbertien est d\'efini comme un espace vectoriel r\'eel ou complexe muni d'un produit scalaire

Un espace pr\'ehilbertien $(E,\langle\cdot,\cdot\rangle)$ est alors un espace vectoriel E muni d'un produit scalaire $\langle\cdot,\cdot\rangle$.

\textbf{Espace de Hilbert}
C'est un espace pr\'ehilbertien complet, c'est-\`a-dire un espace de Banach dont la norme $\parallel\bullet\parallel$d\'ecoule d'un produit scalaire ou hermitien $\langle\cdot,\cdot\rangle$ par la formule
 $$\parallel x\parallel = \sqrt{\langle x,x \rangle}$$
C'est la g\'en\'eralisation en dimension quelconque d'un espace euclidien ou hermitien.
\bigskip

\textbf{Application hermitien's eigenvalues are real values}.
\begin{proof}
Soient A une matrice autoadjointe (r\'eelle ou complexe), $\lambda$ une racine de son polyn\^ome caract\'eristique (il en existe au moins une, mais a priori complexe), et $X$ une matrice colonne complexe non nulle telle que
$AX=\lambda X$. Alors
$$ \overline\lambda X^*.X=(AX)^*.X=X^*.A^*.X=X^*.A.X=\lambda X^*.X $$
or $X*.X$ est non nul, donc $\lambda$ est r\'eel.
\end{proof}

\begin{theorem}
Th\'eor\`eme spectral en dimension finie, pour les endomorphismes —  Tout endomorphisme auto-adjoint d'un espace euclidien ou hermitien est diagonalisable dans une base orthonormale et ses valeurs propres sont toutes r\'eelles.

Th\'eor\`eme spectral pour les matrices —  Soit $A$ une matrice sym\'etrique r\'eelle (resp. hermitienne complexe), alors il existe une matrice $P$ orthogonale (resp. unitaire) et une matrice $D$ diagonale dont tous les coefficients sont r\'eels, telles que la matrice $A = P.D.P^{-1}$
\end{theorem}

\begin{theorem}
Diagonalisation d'un endomorphisme autoadjoint et d'une matrice autoadjointe — \\
Un endomorphisme d'un espace euclidien ou hermitien est autoadjoint si et seulement s'il existe une base orthonormale de vecteurs propres, avec valeurs propres toutes r\'eelles.\\
Une matrice carr\'ee complexe $A$ est autoadjointe si et seulement s'il existe une matrice unitaire $U$ telle que $U.A.U^{-1}$ soit diagonale et r\'eelle.\\
Une matrice carr\'ee r\'eelle $A$ est \textbf{sym\'etrique} si et seulement s'il existe une \textbf{matrice orthogonale} P telle que $P.A.P^{-1}$ soit diagonale et r\'eelle.
\end{theorem}
Rappel:
Une matrice carr\'ee $A$ (n lignes, n colonnes) \`a coefficients r\'eels est dite \textbf{orthogonale}正交矩阵 si $A^t A = I_n$, c'est-\`a-dire que $A^t = A^{-1}$

\textbf{Th\'eor\`eme de Riesz (Fr\'echet-Riesz)}\newline
un th\'eor\`eme qui repr\'esente les \'el\'ements du dual d'un espace de Hilbert comme produit scalaire par un vecteur de l'espace.
Soient :
\begin{itemize}
	\item H un espace de Hilbert (r\'eel ou complexe) muni de son produit scalaire not\'e $<.,.>$
	\item f in H' une forme lin\'eaire continue sur H.
\end{itemize}
Alors il existe un unique $y$ dans $H$ tel que pour tout x de H on ait $f(x) = <y, x>$
$$
\exists\,!\ y \in H\,, \quad \forall x\in H\,, \quad f(x) = \langle y,x\rangle
$$
\underline{Extension aux formes bilin\'eaires}\newline
Si a est une forme bilin\'eaire continue sur un espace de Hilbert r\'eel H (ou une forme sesquilin\'eaire complexe continue sur un Hilbert complexe), alors il existe une unique application A de H dans H telle que, pour tout $(u, v) \in H \times H$, on ait $a(u, v) = <Au, v>$. De plus, A est lin\'eaire et continue, de norme \'egale \`a celle de a.
$$
\exists !\,A\in\mathcal{L}(H),\ \forall (u,v)\in H\times H,\ a(u,v)=\langle Au,v \rangle.
$$
Cela r\'esulte imm\'ediatement de l'isomorphisme canonique (isom\'etrique) entre l'espace norm\'e des formes bilin\'eaires continues sur H × H et celui des applications lin\'eaires continues de H dans son dual, et de l'isomorphisme ci-dessus entre ce dual et H lui-m\^eme.
\bigskip

\textbf{Th\'eor\`eme de Lax-Milgram}\newline
Appliqu\'e \`a certains probl\`emes aux d\'eriv\'ees partielles exprim\'es sous une formulation faible (appel\'ee \'egalement formulation variationnelle). Il est notamment l'un des fondements de la m\'ethode des \'el\'ements finis.
Soient :
\begin{itemize}
		\item $\mathcal{H}$ un espace de Hilbert r\'eel ou complexe muni de son produit scalaire not\'e $\langle.,.\rangle$, de norme associ\'ee not\'ee $\|.\|$
		\item a(.\, ,\,.) une forme bilin\'eaire (ou une forme sesquilin\'eaire si $\mathcal{H}$ est complexe) qui est
				\begin{itemize}
						\item continue sur $\mathcal{H}\times\mathcal{H} : \exists\,c>0, \forall (u,v)\in \mathcal{H}^2\,,\ |a(u,v)|\leq c\|u\|\|v\|$
						\item coercive sur $\mathcal{H}$ (certains auteurs disent plut\^ot $\mathcal{H}$-elliptique) : $\exists\,\alpha>0, \forall u\in\mathcal{H}\,,\ a(u,u) \geq \alpha\|u\|^2$
				\end{itemize}
		\item $L(.)$ une forme lin\'eaire continue sur $\mathcal{H}$
\end{itemize}
Sous ces hypoth\`eses il existe un unique $u$ de $\mathcal{H}$ tel que l'\'equation $a(u,v)=L(v)$ soit v\'erifi\'ee pour tout $v$ de $\mathcal{H}$ :
$$
\quad \exists!\ u \in \mathcal{H},\ \forall v\in\mathcal{H},\quad a(u,v)=L(v)
$$
Si de plus la forme bilin\'eaire a est sym\'etrique, alors  $u$  est l'unique \'el\'ement de $\mathcal{H}$ qui minimise la fonctionnelle $J:\mathcal{H}\rightarrow\R$ d\'efinie par $J(v) = \tfrac{1}{2}a(v,v)-L(v)$ pour tout $v$ de $\mathcal{H}$, c'est-\`a-dire :
$$
\quad \exists!\ u \in \mathcal{H},\quad J(u) = \min_{v\in\mathcal{H}}\ J(v)
$$
\bigskip

$-\laplace $ admet une base de fonctions propres $v_k$, $k \in N$,
orthonormales pour le produit scalaire de $L^2(\Omega)$
\end{document}
