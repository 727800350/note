% !Mode:: "TeX:UTF-8"
\usepackage[english]{babel}
\usepackage{ctex}
\usepackage{amsmath, amsthm, amssymb}

%%%%%%%%%% header & footer %%%%%%%%%%
\setlength\textheight{100.0pt}
\setlength\textwidth{430.0pt}
\usepackage{fancyhdr}
\usepackage{lastpage}
\usepackage{layout}
%\pagestyle{empty}                   %不设置页眉页脚
\footskip = 20pt
\pagestyle{fancy}
\fancyhead{} %clear all fields
%% \lhead{Page \thepage\ of \pageref{LastPage}}
%% \chead{}
\rhead{\small\leftmark}
\fancyfoot{} %clear all fields
%% \lfoot{\href{mailto:wangchaogo1990@gmail.com}{feedback}}
%% \cfoot{}
\rfoot{\thepage}
\renewcommand{\headrulewidth}{1pt}  %页眉线宽,设为0可以去页眉线
\setlength{\skip\footins}{0.5cm}    %脚注与正文的距离
\renewcommand{\footnotesize}{}      %设置脚注字体大小
\renewcommand{\footrulewidth}{1pt}  %脚注线的宽度
%===============
%双线页眉的设置
%\makeatletter %双线页眉
%\def\headrule{{\if@fancyplain\let\headrulewidth\plainheadrulewidth\fi%
%\hrule\@height 1.0pt \@width\headwidth\vskip1pt%上面线为1pt粗
%\hrule\@height 0.5pt\@width\headwidth  %下面0.5pt粗
%\vskip-2\headrulewidth\vskip-1pt}      %两条线的距离1pt
%  \vspace{6mm}}     %双线与下面正文之间的垂直间距
%\makeatother
%===============
%\pagestyle{fancy}
%\fancyhead{} %clear all fields
%\fancyhead[CE]{ \应\用 \数 \学 }
%\fancyhead[CO]{{ huangzh: 数学}}
%\fancyhead[RO]{\thepage} %奇数页眉的右边
%\fancyhead[LE]{\thepage} %偶数页眉的左边
%\fancyhead[RE]{\zihao{-5} 2005 c}
%\fancyfoot[C]{}

%图形
\usepackage{pstricks}
\usepackage{graphicx}
\usepackage{subfigure}
\usepackage{tikz} %pdf figure
\usetikzlibrary{positioning}
\def\pgfsysdriver{pgfsys-dvipdfmx.def}
\pgfsetxvec{\pgfpoint{10pt}{0}}
\pgfsetyvec{\pgfpoint{0}{10pt}}
\tikzset{
box/.style={rectangle, rounded corners=6pt,
minimum width=50pt, minimum height=20pt, inner sep=6pt,
draw=gray,thick, fill=lightgray}
}
\usepackage{ccaption}
\usepackage{epstopdf} %convert eps to pdf
%% bmp, gif not supported
\DeclareGraphicsExtensions{.eps,.mps,.pdf,.jpg,.png}
\DeclareGraphicsRule{*}{eps}{*}{}
\graphicspath{{figure/}{../figure/}}

\usepackage[colorinlistoftodos]{todonotes}


%尺寸
\usepackage[top=2.54cm, bottom=2.54cm, left=3.18cm, right=3.18cm]{geometry} % ms word
%\usepackage[top=0.2cm, bottom=0.2cm, left=0cm, right=0cm,paperwidth=9cm,paperheight=11.7cm]{geometry} % kindle
%字体
%% 主要字体为Times New Roman和宋体
%% 模板某些标题需要华文行楷和32号字
%\setmainfont{Times New Roman}
%% 不需要设置CJKmainfont,ctex 宏包已经很好的处理了
%% 不仅设置了粗体为黑体,斜体为楷体,还兼容了winfonts和adobefonts
%% 直接设置反而会在只有adobefonts的情况下报错
%\setCJKmainfont{宋体}
%\setCJKfamilyfont{hwxingkai}{STXingkai}
%\newcommand{\hwxingkai}{\CJKfamily{hwxingkai}}
%\newcommand{\xiaochuhao}{\fontsize{32pt}{\baselineskip}\selectfont}


\newcommand{\song}{\CJKfamily{song}} %宋体
\newcommand{\fs}{\CJKfamily{fs}} %仿宋体
\newcommand{\kai}{\CJKfamily{kai}} %楷体
\newcommand{\hei}{\CJKfamily{hei}} %黑体
\newcommand{\li}{\CJKfamily{li}} %隶书
\newcommand{\you}{\CJKfamily{you}} %幼圆

% 一号, 1.4 倍行距
\newcommand{\yihao}{\fontsize{26pt}{36pt}\selectfont}
% 二号, 1.25倍行距
\newcommand{\erhao}{\fontsize{22pt}{28pt}\selectfont}
% 小二, 单倍行距
\newcommand{\xiaoer}{\fontsize{18pt}{18pt}\selectfont}
% 三号, 1.5倍行距
\newcommand{\sanhao}{\fontsize{16pt}{24pt}\selectfont}
% 小三, 1.5倍行距
\newcommand{\xiaosan}{\fontsize{15pt}{22pt}\selectfont}
% 四号, 1.5 倍行距
\newcommand{\sihao}{\fontsize{14pt}{21pt}\selectfont}
% 半小四, 1.5倍行距
\newcommand{\banxiaosi}{\fontsize{13pt}{19.5pt}\selectfont}
% 小四, 1.5倍行距
\newcommand{\xiaosi}{\fontsize{12pt}{18pt}\selectfont}
% 大五号, 单倍行距
\newcommand{\dawuhao}{\fontsize{11pt}{11pt}\selectfont}
% 五号, 单倍行距
\newcommand{\wuhao}{\fontsize{10.5pt}{15.75pt}\selectfont}

%% code
%% verbatim 中不能使用\textbf 等格式, alltt可以使用
\usepackage{alltt}

\usepackage{listings}
\lstset{
    backgroundcolor=\color{white},
%%     basicstyle=\wuhao\ttfamily,
    columns=flexible,
    breakatwhitespace=false,
    breaklines=true,
    captionpos=b,
    frame=single,
    numbers=left,
    numbersep=5pt,
    showspaces=false,
    showstringspaces=false,
    showtabs=false,
    stepnumber=1,
    rulecolor=\color{black},
    tabsize=2,
    texcl=true,
    title=\lstname,
    escapeinside={\%*}{*)},
    extendedchars=false,
    mathescape=true,
    xleftmargin=3em,
    xrightmargin=3em,
    numberstyle=\color{gray},
    keywordstyle=\color{blue},
    commentstyle=\color{dkgreen},
    stringstyle=\color{mauve},
}

\newcommand{\fun}[1]{\textit{#1}}

%% hyper reference
%% 使用hyperref包,section,subsection,subsubsection命名中使用中文出现问题
%% 编译通不过, 加了CJKbookmarks=true 之后可以使用中文
\usepackage[colorlinks,linkcolor=blue,anchorcolor=red,citecolor=green,CJKbookmarks=true]{hyperref}

%表格
%allow the use of \toprule, \midrule, and \bottomrule
\usepackage{booktabs}
\usepackage{tabularx}
\usepackage{multirow}
\usepackage{colortbl}
\usepackage{longtable}

%\usepackage{dirtree}  %目录结构图

%颜色
\usepackage{color}
%\definecolor{name}{system}{definition}
\definecolor{Gray}{gray}{0.9}
\definecolor{LightCyan}{rgb}{0.88,1,1}
\definecolor{dkgreen}{rgb}{0,0.6,0}
\definecolor{gray}{rgb}{0.5,0.5,0.5}
\definecolor{mauve}{rgb}{0.58,0,0.82}

\def\red#1{\textcolor[rgb]{1.00,0.00,0.00}{#1}}
\def\yellow#1{\textcolor[rgb]{1.00,1.00,0.00}{#1}}
\def\lime#1{\textcolor[rgb]{0.00,1.00,0.00}{#1}}


\newcommand\warning[1]{\red{#1}}
\newcommand\obey[1]{\thicksim{#1}}

%% 化学
%% \usepackage[version=3]{mhchem}
%% ex: Ammonium sulphate is \ce{(NH4)2SO4}.

\usepackage{fixltx2e}
%% 默认情况下只能使用\textsuperscript,加了这个package之后可以使用\textsubscript

%Math
%% Macro
\def\vecteur#1{(#1_1,~#1_2,~\ldots,~#1_n)}
\def\vector#1{#1_1,~#1_2,~\ldots,~#1_n}
%% \def\vecteur#1{\ensuremath{(#1_1,~#1_2,~\ldots,~#1_n)}}

%随机数
%\usepackage[first=0,last=9]{lcg}
%\newcommand{\ra}{\rand0.\arabic{rand}}

\newcommand{\R}{\mathbb{R}}   %the real number set
%% \newcommand{\C}{\mathbb{C}} %xelatex says \C has already been used.
\newcommand{\N}{\mathbb{N}}
\newcommand{\Z}{\mathbb{Z}}
\newcommand{\Q}{\mathbb{Q}}

\newcommand{\si}{\textrm{ si }}
\newcommand{\sinon}{\textrm{ si non}}
\newcommand{\et}{\textrm{ et }}
\newcommand{\ou}{\textrm{ ou }}
\newcommand{\non}{\textrm{non }}
\newcommand{\ssi}{si et seulement si }
\newcommand{\infinity}{\infty}

\newcommand{\diag}[1]{\textit{diag}(#1)}
%定义运算符
\DeclareMathOperator{\arccot}{arcot}
\DeclareMathOperator{\arcth}{arcth}
\DeclareMathOperator{\arcsh}{arcsh}
\DeclareMathOperator{\arch}{arch}
\DeclareMathOperator{\ch}{ch}
\DeclareMathOperator{\dth}{th} %\th 已经被定义了
\DeclareMathOperator{\sh}{sh}

\newcommand*\laplace{\mathop{}\!\mathbin\bigtriangleup}
\newcommand*\dalambert{\mathop{}\!\mathbin\Box}
\newcommand{\grad}[1]{\nabla #1}
\newcommand{\gradien}[1]{\nabla #1}
\newcommand{\divergence}[1]{\nabla \cdot #1}
\newcommand{\rotationnel}[1]{\nabla \times #1}
\newcommand{\rot}[1]{\nabla \times #1}

\newcommand{\stcomp}[1]{\overline{#1}} % set complement

\usepackage{xspace}
%produit scalaire
\newcommand{\ps}[2]{\ensuremath{\langle #1 , #2\rangle}\xspace}

%use \lasteq to reference the last equation
\newcommand\lasteq{(\theequation)\ }
\newcommand{\eqspace}{\hspace{0.5cm}}
\newcommand{\norm}[1]{\left\Vert #1\right\Vert}
%% insert text in math mode being treated as normal text
\newcommand{\eqnote}[1]{\text{ #1 }}
%% generate a fraction but without the line
\newcommand\mytop[2]{\genfrac{}{}{0pt}{}{#1}{#2}}

%% 特殊符号
%% \usepackage{pifont}
%% \ding{number}, 通过number 来调用不同的符号

%% quantique operators
\newcommand\ket[1]{|#1\rangle}
\newcommand\bra[1]{\langle #1|}
\newcommand\braket[3]{\langle#1|#2|#3\rangle}

