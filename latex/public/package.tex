% !Mode:: "TeX:UTF-8"
\usepackage[english]{babel}
\usepackage{ctex}
\usepackage{amsmath, amsthm, amssymb}

%%%%%%%%%% header & footer %%%%%%%%%%
\setlength\textheight{100.0pt}
\setlength\textwidth{430.0pt}
\usepackage{fancyhdr}
\usepackage{lastpage} %% \pageref{LastPage} 获取最后一页的页码
\usepackage{pdfpages}
\usepackage{layout}
\footskip = 20pt
\pagestyle{fancy}
\fancyhead{} %clear all fields
%% \newsavebox{\headpic}
%% \sbox{\headpic}{\includegraphics[height=1cm]{logo}} %设置页眉logo页眉
%% \lhead{\usebox{\headpic}}
\rhead{\small\leftmark}
\fancyfoot{} %clear all fields
\cfoot{\thepage}
\renewcommand{\headrulewidth}{1pt}  %页眉线宽,设为0可以去页眉线
\setlength{\skip\footins}{0.5cm}    %脚注与正文的距离
\renewcommand{\footnotesize}{}      %设置脚注字体大小
%% \renewcommand{\footrulewidth}{1pt}  %脚注线的宽度
%===============

%图形
\usepackage{graphicx}
\usepackage{float} %% 可以使用H固定图片的位置
\usepackage{caption}
\usepackage[format=hang,singlelinecheck=0,font={sf,small},labelfont=bf]{subfig}
\usepackage[noabbrev]{cleveref}
\captionsetup[subfigure]{subrefformat=simple,labelformat=simple,listofformat=subsimple}
\renewcommand\thesubfigure{(\alph{subfigure})}

\usepackage{epstopdf} %convert eps to pdf
\DeclareGraphicsExtensions{.eps,.mps,.pdf,.jpg,.png} %% bmp, gif not supported
\DeclareGraphicsRule{*}{eps}{*}{}
\graphicspath{{figure/}{../figure/}}

%% \usepackage{pstricks} %% a set of macros that allow the inclusion of PostScript drawings directly inside TeX or LaTeX code
%% \usepackage{wrapfig} %% Wrapping text around figures

%表格
%allow the use of \toprule, \midrule, and \bottomrule
\usepackage{booktabs}
\usepackage{tabularx}
\usepackage{multirow}
\usepackage{colortbl}
\usepackage{longtable}
\usepackage{supertabular}

\usepackage[colorinlistoftodos]{todonotes}

%尺寸
\usepackage[top=2.54cm, bottom=2.54cm, left=3.18cm, right=3.18cm]{geometry} % ms word
%% \usepackage[a4paper]{geometry}
%\usepackage[top=0.2cm, bottom=0.2cm, left=0cm, right=0cm,paperwidth=9cm,paperheight=11.7cm]{geometry} % kindle

%% code
%% verbatim 中不能使用\textbf 等格式, alltt可以使用
%% \usepackage{alltt}

\usepackage{listings}
\lstset{
    backgroundcolor=\color{white},
    columns=flexible,
    breakatwhitespace=false,
    breaklines=true,
    captionpos=tt,
    frame=single, %% Frame: 这个命令能让代码周围显示边框. 它的值可以是 none|leftline|topline|bottomline|lines|single|shadowbox 中的任意一个
    numbers=left, %% numbers 的值可以是 left, right, none, 分别代表了将行号放在代码的左边, 右边和不显示
    numbersep=5pt,
    showspaces=false,
    showstringspaces=false,
    showtabs=false,
    stepnumber=1, %% 设置每多少行显示一次行号
    rulecolor=\color{black},
    tabsize=2,
    texcl=true,
    title=\lstname,
    escapeinside={\%*}{*)},
    extendedchars=false,
    mathescape=true,
    xleftmargin=3em,
    xrightmargin=3em,
    numberstyle=\color{gray},
    keywordstyle=\color{blue},
    commentstyle=\color{dkgreen},
    stringstyle=\color{mauve},
}

\newcommand{\fun}[1]{\textit{#1}}

%% hyper reference
%% 使用hyperref包,section,subsection,subsubsection命名中使用中文出现问题, 编译通不过, 加了CJKbookmarks=true 之后可以使用中文
\usepackage[colorlinks,linkcolor=blue,anchorcolor=red,citecolor=green,CJKbookmarks=true]{hyperref}

\bibliographystyle{plain} % 参考文献格式

%\usepackage{dirtree}  %目录结构图

%颜色
\usepackage{color}
%\definecolor{name}{system}{definition}
\definecolor{Gray}{gray}{0.9}
\definecolor{LightCyan}{rgb}{0.88,1,1}
\definecolor{dkgreen}{rgb}{0,0.6,0}
\definecolor{gray}{rgb}{0.5,0.5,0.5}
\definecolor{mauve}{rgb}{0.58,0,0.82}

\def\red#1{\textcolor[rgb]{1.00,0.00,0.00}{#1}}
\def\yellow#1{\textcolor[rgb]{1.00,1.00,0.00}{#1}}
\def\lime#1{\textcolor[rgb]{0.00,1.00,0.00}{#1}}


\newcommand\warning[1]{\red{#1}}
\newcommand\obey[1]{\thicksim{#1}}

%% \usepackage{fixltx2e}
%% 默认情况下只能使用\textsuperscript,加了这个package之后可以使用\textsubscript

