\documentclass{beamer}
% !Mode:: "TeX:UTF-8"
%% \usepackage{metalogo}
%% \usepackage{doc}
\usepackage{ctex}
\usepackage[english]{babel}
\usepackage{amsmath, amsthm, amssymb}
\usepackage{graphicx}
    
%图形
\usepackage{pstricks}
\usepackage{graphicx}
\usepackage{subfigure}
\usepackage{tikz} %pdf figure
\usetikzlibrary{positioning}
\def\pgfsysdriver{pgfsys-dvipdfmx.def}
\pgfsetxvec{\pgfpoint{10pt}{0}}
\pgfsetyvec{\pgfpoint{0}{10pt}}
\tikzset{
box/.style={rectangle, rounded corners=6pt,
minimum width=50pt, minimum height=20pt, inner sep=6pt,
draw=gray,thick, fill=lightgray}
}
\usepackage{ccaption}
\usepackage{epstopdf} %convert eps to pdf
%% bmp, gif not supported
\DeclareGraphicsExtensions{.eps,.mps,.pdf,.jpg,.png}
\DeclareGraphicsRule{*}{eps}{*}{}
\graphicspath{{figure/}{../figure/}}

\usepackage[colorinlistoftodos]{todonotes}

%字体
%% 主要字体为Times New Roman和宋体
%% 模板某些标题需要华文行楷和32号字
%\setmainfont{Times New Roman}
%% 不需要设置CJKmainfont,ctex 宏包已经很好的处理了
%% 不仅设置了粗体为黑体,斜体为楷体,还兼容了winfonts和adobefonts
%% 直接设置反而会在只有adobefonts的情况下报错
%\setCJKmainfont{宋体}
%\setCJKfamilyfont{hwxingkai}{STXingkai}
%\newcommand{\hwxingkai}{\CJKfamily{hwxingkai}}
%\newcommand{\xiaochuhao}{\fontsize{32pt}{\baselineskip}\selectfont}


\newcommand{\song}{\CJKfamily{song}} %宋体
\newcommand{\fs}{\CJKfamily{fs}} %仿宋体
\newcommand{\kai}{\CJKfamily{kai}} %楷体
\newcommand{\hei}{\CJKfamily{hei}} %黑体
\newcommand{\li}{\CJKfamily{li}} %隶书
\newcommand{\you}{\CJKfamily{you}} %幼圆

% 一号, 1.4 倍行距
\newcommand{\yihao}{\fontsize{26pt}{36pt}\selectfont}
% 二号, 1.25倍行距
\newcommand{\erhao}{\fontsize{22pt}{28pt}\selectfont}
% 小二, 单倍行距
\newcommand{\xiaoer}{\fontsize{18pt}{18pt}\selectfont}
% 三号, 1.5倍行距
\newcommand{\sanhao}{\fontsize{16pt}{24pt}\selectfont}
% 小三, 1.5倍行距
\newcommand{\xiaosan}{\fontsize{15pt}{22pt}\selectfont}
% 四号, 1.5 倍行距
\newcommand{\sihao}{\fontsize{14pt}{21pt}\selectfont}
% 半小四, 1.5倍行距
\newcommand{\banxiaosi}{\fontsize{13pt}{19.5pt}\selectfont}
% 小四, 1.5倍行距
\newcommand{\xiaosi}{\fontsize{12pt}{18pt}\selectfont}
% 大五号, 单倍行距
\newcommand{\dawuhao}{\fontsize{11pt}{11pt}\selectfont}
% 五号, 单倍行距
\newcommand{\wuhao}{\fontsize{10.5pt}{15.75pt}\selectfont}

%% code
%% verbatim 中不能使用\textbf 等格式, alltt可以使用
\usepackage{alltt}

\usepackage{listings}
\lstset{
    backgroundcolor=\color{white},
%%     basicstyle=\wuhao\ttfamily,
    columns=flexible,
    breakatwhitespace=false,
    breaklines=true,
    captionpos=b,
    frame=single,
    numbers=left,
    numbersep=5pt,
    showspaces=false,
    showstringspaces=false,
    showtabs=false,
    stepnumber=1,
    rulecolor=\color{black},
    tabsize=4,
    texcl=true,
    title=\lstname,
    escapeinside={\%*}{*)},
    extendedchars=false,
    mathescape=true,
    xleftmargin=3em,
    xrightmargin=3em,
    numberstyle=\color{gray},
    keywordstyle=\color{blue},
    commentstyle=\color{dkgreen},
    stringstyle=\color{mauve},
}

\newcommand{\fun}[1]{\textit{#1}}

%% 编译通不过, 加了CJKbookmarks=true 之后可以使用中文
\hypersetup{colorlinks,linkcolor=blue,urlcolor=green,anchorcolor=red,citecolor=green,CJKbookmarks=true}

%表格
%allow the use of \toprule, \midrule, and \bottomrule
\usepackage{booktabs}
\usepackage{tabularx}
\usepackage{multirow}
\usepackage{colortbl}
\usepackage{longtable}

%\usepackage{dirtree}  %目录结构图

%颜色
\usepackage{color}
%\definecolor{name}{system}{definition}
\definecolor{Gray}{gray}{0.9}
\definecolor{LightCyan}{rgb}{0.88,1,1}
\definecolor{dkgreen}{rgb}{0,0.6,0}
\definecolor{gray}{rgb}{0.5,0.5,0.5}
\definecolor{mauve}{rgb}{0.58,0,0.82}

\def\red#1{\textcolor[rgb]{1.00,0.00,0.00}{#1}}
\def\yellow#1{\textcolor[rgb]{1.00,1.00,0.00}{#1}}
\def\lime#1{\textcolor[rgb]{0.00,1.00,0.00}{#1}}



%% 化学
\usepackage[version=3]{mhchem}
%% ex: Ammonium sulphate is \ce{(NH4)2SO4}.

\usepackage{fixltx2e}
%% 默认情况下只能使用\textsuperscript,加了这个package之后可以使用\textsubscript

%Math
%% Macro
\def\vecteur#1{(#1_1,~#1_2,~\ldots,~#1_n)}
%% \def\vecteur#1{\ensuremath{(#1_1,~#1_2,~\ldots,~#1_n)}}

%随机数
%\usepackage[first=0,last=9]{lcg}
%\newcommand{\ra}{\rand0.\arabic{rand}}

\newcommand{\R}{\mathbb{R}}   %the real number set
%% \newcommand{\C}{\mathbb{C}} %xelatex says \C has already been used.
\newcommand{\N}{\mathbb{N}}
\newcommand{\Z}{\mathbb{Z}}
\newcommand{\Q}{\mathbb{Q}}

\newcommand{\si}{\textrm{ si }}
\newcommand{\sinon}{\textrm{ si non}}
\newcommand{\et}{\textrm{ et }}
\newcommand{\ou}{\textrm{ ou }}
\newcommand{\non}{\textrm{non }}
\newcommand{\ssi}{si et seulement si }
\newcommand{\infinity}{\infty}

%定义运算符
\DeclareMathOperator{\arccot}{arcot}
\DeclareMathOperator{\arcth}{arcth}
\DeclareMathOperator{\arcsh}{arcsh}
\DeclareMathOperator{\arch}{arch}
\DeclareMathOperator{\ch}{ch}
\DeclareMathOperator{\dth}{th} %\th 已经被定义了
\DeclareMathOperator{\sh}{sh}

\newcommand*\laplace{\mathop{}\!\mathbin\bigtriangleup}
\newcommand*\dalambert{\mathop{}\!\mathbin\Box}
\newcommand{\grad}[1]{\nabla #1}
\newcommand{\gradien}[1]{\nabla #1}
\newcommand{\divergence}[1]{\nabla \cdot #1}
\newcommand{\rotationnel}[1]{\nabla \times #1}
\newcommand{\rot}[1]{\nabla \times #1}

\newcommand{\stcomp}[1]{\overline{#1}} % set complement

\usepackage{xspace}
%produit scalaire
\newcommand{\ps}[2]{\ensuremath{\langle #1 , #2\rangle}\xspace}

%use \lasteq to reference the last equation
\newcommand\lasteq{(\theequation)}
\newcommand{\eqspace}{\hspace{0.5cm}}
\newcommand{\norm}[1]{\left\Vert #1\right\Vert}
%% insert text in math mode being treated as normal text
\newcommand{\eqnote}[1]{\text{ #1 }}
%% generate a fraction but without the line
\newcommand\mytop[2]{\genfrac{}{}{0pt}{}{#1}{#2}}

%% 特殊符号
%% \usepackage{pifont}
%% \ding{number}, 通过number 来调用不同的符号

\mode<presentation> {
%% \usetheme{default}
%% \usetheme{AnnArbor}
%% \usetheme{Antibes}
%% \usetheme{Bergen}
%% \usetheme{Berkeley}
%% \usetheme{Berlin}
%% \usetheme{Boadilla}
%% \usetheme{CambridgeUS}
%% \usetheme{Copenhagen}
\usetheme{Darmstadt}
%% \usetheme{Dresden}
%% \usetheme{Frankfurt}
%% \usetheme{Goettingen}
%% \usetheme{Hannover}
%% \usetheme{Ilmenau}
%% \usetheme{JuanLesPins}
%% \usetheme{Luebeck}
%% \usetheme{Madrid}
%% \usetheme{Malmoe}
%% \usetheme{Marburg}
%% \usetheme{Montpellier}
%% \usetheme{PaloAlto}
%% \usetheme{Pittsburgh}
%% \usetheme{Rochester}
%% \usetheme{Singapore}
%% \usetheme{Szeged}
%% \usetheme{Warsaw}

%% \usecolortheme{albatross}
%% \usecolortheme{beaver}
%% \usecolortheme{beetle}
%% \usecolortheme{crane}
%% \usecolortheme{dolphin}
%% \usecolortheme{dove}
%% \usecolortheme{fly}
%% \usecolortheme{lily}
%% \usecolortheme{orchid}
%% \usecolortheme{rose}
%% \usecolortheme{seagull}
\usecolortheme{seahorse}
%% \usecolortheme{whale}
%% \usecolortheme{wolverine}

\usefonttheme{professionalfonts}
\useinnertheme{rectangles}

%\setbeamertemplate{footline} % To remove the footer line in all slides uncomment this line
%\setbeamertemplate{footline}[page number] % To replace the footer line in all slides with a simple slide count uncomment this line

%\setbeamertemplate{navigation symbols}{} % To remove the navigation symbols from the bottom of all slides uncomment this line
\setbeamertemplate{itemize item}{$\spadesuit$}
\setbeamertemplate{itemize subitem}{$\heartsuit$}
\setbeamertemplate{itemize subsubitem}{$\clubsuit$}
\setbeamertemplate{enumerate item}{\insertenumlabel.}
\setbeamertemplate{enumerate subitem}{\insertenumlabel.\insertsubenumlabel}
\setbeamertemplate{enumerate subsubitem}{\insertenumlabel.\insertsubenumlabel.\insertsubsubenumlabel}
\setbeamertemplate{enumerate mini template}{\insertenumlabel}
\setbeamercolor{footcolor}{fg=black,bg=red!40} % 设置字体和背景颜色
\setbeamertemplate{footline}{%
  \leavevmode%
  \hbox{%
    \begin{beamercolorbox}[wd=.126\paperwidth,ht=2.25ex,dp=1ex,right]{footcolor}%
       \insertframenumber{} / \inserttotalframenumber\hspace*{5ex}
    \end{beamercolorbox}}%
  \vskip0pt%
}
}


\title[Short title]{Full Title of the Talk}
\author{Eric Wang}
\institute[ECPKN]{
Ecole Centrale de P\'ekin\\
\medskip
\textit{wangchaogo1990 dot at gmail.com}
}
\date{\today} % Date, can be changed to a custom date

\begin{document}
\begin{frame}
\titlepage
\end{frame}

\begin{frame}
\frametitle{Overview}
\tableofcontents
\end{frame}

\section{First Section}
\subsection{Subsection Example}

\begin{frame}
\frametitle{Paragraphs of Text}
Sed iaculis dapibus gravida. Morbi sed tortor erat, nec interdum arcu. Sed id lorem lectus. Quisque viverra augue id sem ornare non aliquam nibh tristique. Aenean in ligula nisl. Nulla sed tellus ipsum. Donec vestibulum ligula non lorem vulputate fermentum accumsan neque mollis.\\~\\
\end{frame}


\begin{frame}
\frametitle{Bullet Points}
\begin{itemize}
\item Lorem ipsum dolor sit amet, consectetur adipiscing elit
	\begin{itemize}
	\item test
	\item test
	\end{itemize}
\item Aliquam blandit faucibus nisi, sit amet dapibus enim tempus eu
\end{itemize}
\end{frame}


\begin{frame}
\frametitle{Blocks of Highlighted Text}
\begin{block}{Block 1}
Lorem ipsum dolor sit amet, consectetur adipiscing elit. Integer lectus nisl, ultricies in feugiat rutrum, porttitor sit amet augue. Aliquam ut tortor mauris. Sed volutpat ante purus, quis accumsan dolor.
\end{block}

\begin{block}{Block 2}
Pellentesque sed tellus purus. Class aptent taciti sociosqu ad litora torquent per conubia nostra, per inceptos himenaeos. Vestibulum quis magna at risus dictum tempor eu vitae velit.
\end{block}
\end{frame}


\begin{frame}
\frametitle{Multiple Columns}
\begin{columns}[c] % The "c" option specifies centered vertical alignment while the "t" option is used for top vertical alignment

\column{.45\textwidth} % Left column and width
\textbf{Heading}
\begin{enumerate}
\item Statement
\item Explanation
\item Example
\end{enumerate}

\column{.5\textwidth} % Right column and width
Lorem ipsum dolor sit amet, consectetur adipiscing elit. Integer lectus nisl, ultricies in feugiat rutrum, porttitor sit amet augue. Aliquam ut tortor mauris. Sed volutpat ante purus, quis accumsan dolor.

\end{columns}
\end{frame}

\section{Second Section}
\begin{frame}
\frametitle{Table}
\begin{table}
\begin{tabular}{l l l}
\toprule
\textbf{Treatments} & \textbf{Response 1} & \textbf{Response 2}\\
\midrule
Treatment 1 & 0.0003262 & 0.562 \\
Treatment 2 & 0.0015681 & 0.910 \\
Treatment 3 & 0.0009271 & 0.296 \\
\bottomrule
\end{tabular}
\caption{Table caption}
\end{table}
\end{frame}


\begin{frame}
\frametitle{Theorem}
\begin{theorem}[Mass--energy equivalence]
$E = mc^2$
\end{theorem}
\end{frame}


\begin{frame}[fragile] % Need to use the fragile option when verbatim is used in the slide
\frametitle{Verbatim}
\begin{example}[Theorem Slide Code]
\begin{verbatim}
\begin{frame}
\frametitle{Theorem}
\begin{theorem}[Mass--energy equivalence]
$E = mc^2$
\end{theorem}
\end{frame}\end{verbatim}
\end{example}
\end{frame}


\begin{frame}[fragile] % Need to use the fragile option when verbatim is used in the slide
\frametitle{Verbatim}
\begin{example}[Theorem Slide Code]
\begin{lstlisting}[language = C]
int main(){
	int ret=0;
	return ret;
}
\end{lstlisting}
\end{example}
\end{frame}


\begin{frame}
\frametitle{Figure}
Uncomment the code on this slide to include your own image from the same directory as the template .TeX file.
%\begin{figure}
%\includegraphics[width=0.8\linewidth]{test}
%\end{figure}
\end{frame}


\begin{frame}[fragile] % Need to use the fragile option when verbatim is used in the slide
\frametitle{Citation}
An example of the \verb|\cite| command to cite within the presentation:\\~

This statement requires citation \cite{p1}.
\end{frame}


\begin{frame}
\frametitle{References}
\footnotesize{
\begin{thebibliography}{99} % Beamer does not support BibTeX so references must be inserted manually as below
\bibitem[Smith, 2012]{p1} John Smith (2012)
\newblock Title of the publication
\newblock \emph{Journal Name} 12(3), 45 -- 678.
\end{thebibliography}
}
\end{frame}


\begin{frame}
\Huge{\centerline{The End}}
\end{frame}
\end{document} 
