% !Mode:: "TeX:UTF-8"
\documentclass[openany]{book}
% !Mode:: "TeX:UTF-8"
\usepackage[english]{babel}
\usepackage[UTF8]{ctex}
\usepackage{amsmath, amsthm, amssymb}

% Figure
\usepackage{graphicx}
\usepackage{float} %% H can fix the location
\usepackage{caption}
\usepackage[format=hang,singlelinecheck=0,font={sf,small},labelfont=bf]{subfig}
\usepackage[noabbrev]{cleveref}
\captionsetup[subfigure]{subrefformat=simple,labelformat=simple,listofformat=subsimple}
\renewcommand\thesubfigure{(\alph{subfigure})}

\usepackage{epstopdf} %% convert eps to pdf
\DeclareGraphicsExtensions{.eps,.mps,.pdf,.jpg,.png} %% bmp, gif not supported
\DeclareGraphicsRule{*}{eps}{*}{}
\graphicspath{{figure/}{../figure/}} %% fig directorys

%% \usepackage{pstricks} %% a set of macros that allow the inclusion of PostScript drawings directly inside TeX or LaTeX code
%% \usepackage{wrapfig} %% Wrapping text around figures

% Table
\usepackage{booktabs} %% allow the use of \toprule, \midrule, and \bottomrule
\usepackage{tabularx}
\usepackage{multirow}
\usepackage{colortbl}
\usepackage{longtable}
\usepackage{supertabular}

\usepackage[colorinlistoftodos]{todonotes}

% Geometry
\usepackage[paper=a4paper, top=1.5cm, bottom=1.5cm, left=1cm, right=1cm]{geometry}
%% \usepackage[paper=a4paper, top=2.54cm, bottom=2.54cm, left=3.18cm, right=3.18cm]{geometry} %% ms word
%% \usepackage[top=0.1cm, bottom=0.1cm, left=0.1cm, right=0.1cm, paperwidth=9cm, paperheight=11.7cm]{geometry} %% kindle

% Code
%% \usepackage{alltt} %% \textbf can be used in alltt, but not in verbatim

\usepackage{listings}
\lstset{
    backgroundcolor=\color{white},
    columns=flexible,
    breakatwhitespace=false,
    breaklines=true,
    captionpos=tt,
    frame=single, %% Frame: show a box around, possible values are: none|leftline|topline|bottomline|lines|single|shadowbox
    numbers=left, %% possible values are: left, right, none
    numbersep=5pt,
    showspaces=false,
    showstringspaces=false,
    showtabs=false,
    stepnumber=1, %% interval of lines to display the line number
    rulecolor=\color{black},
    tabsize=2,
    texcl=true,
    title=\lstname,
    escapeinside={\%*}{*)},
    extendedchars=false,
    mathescape=true,
    xleftmargin=3em,
    xrightmargin=3em,
    numberstyle=\color{gray},
    keywordstyle=\color{blue},
    commentstyle=\color{dkgreen},
    stringstyle=\color{mauve},
}

% Reference
%% \bibliographystyle{plain} % reference style

% Color
\usepackage[colorlinks, linkcolor=blue, anchorcolor=red, citecolor=green, CJKbookmarks=true]{hyperref}
\usepackage{color}
\def\red#1{\textcolor[rgb]{1.00,0.00,0.00}{#1}}
\newcommand\warning[1]{\red{#1}}

% Other
%% \usepackage{fixltx2e} %% for use of \textsubscript
%% \usepackage{dirtree}  %% directory structure, like the result of command tree in bash shell

   %导入需要用到的package
% !Mode:: "TeX:UTF-8"

% Chapter
%% \makeatletter\@addtoreset{chapter}{part}\makeatother %% have chapters numbered without interruption (numbering through parts)

% Equation
\makeatletter\@addtoreset{equation}{section}\makeatother 
\renewcommand\theequation{%
\thepart\arabic{chapter}%
-\thepart\arabic{section}%
-\thepart\arabic{equation}%
}

% Theorem
\newtheorem{definition}{D\'efintion}
\newtheorem*{thmwn}{Thm}
\newtheorem{theorem}{Th\'eor\`eme}[chapter]
\newtheorem{corollary}{Corollary}[theorem]
\newtheorem{lemma}{Lemma}
\newtheorem{proposition}{Proposition}[chapter]
\newtheorem*{attention}{Attention}
\newtheorem*{note}{Note}
\newtheorem*{remark}{Remark}
\newtheorem{example}{Example}
\newtheorem{question}{Question}[chapter]
\newtheorem{problem}{Problem}
\theoremstyle{remark}\newtheorem*{answer}{Answer}
\newtheorem{fact}{Fact}

   %导入需要用到的package

\begin{document}
\chapter{The New Quantum Universe}
物理学家, 就像一个好侦探一样, 仔细分析各种证据, 遵循福尔摩斯说过的一句格言:"当你把不可能的事情都排出了以后, 剩下的选择,不管看起来多么不太可能, 一顶是对的."  \newline
光子的本质是量子力学的

\textbf{电子的双缝实验}\\
\fbox{Result: 在没有观察光源时, 电子双缝实验发生干涉, 而有观察光源时, 干涉图案消失.}\\
打开和不打开用来观察电子的光源将导致不同的结果! 这一明显的疑惑来自于光子本身的量子属性. 光, 跟电子一样, 是以确定的叫光子的能量块的方式过来的. 为了看见一个物体, 必须至少有一个光子从这个物体上反弹回来. 这正是问题的症结. 但我们将光照射到子弹上时, 子弹的运动不会发生可以察觉的改变. 因为一个单个的光子的能量和子弹的能量相比太小了. 而电子确实非常细小的量子对象. 光照在点子上会给电子一个猛击, 从而显著改变电子自身的运动状态. 更仔细的分析发现,这一扰动总是正好足够破坏整个干涉图案.
\bigskip

\section{海森堡不确定原理}
$$
\Delta x\times \Delta p \geq \hbar
$$
实验测量的精度存在原则上的极限. 在量子世界里, 扰动(比如光子造成的扰动)是无法忽略的, 这就是海森堡不确定原理的本质.\\
为了精确测量粒子的位置, 就必须使用波长很短的光,因为光的波长决定了无门能将粒子定位在某一定范围的最小长度. 而很短的波长的光有很高的频率, 光子的能力$E=h\nu$, 高频率的
光子达到系统上, 会给量子体系一个很大的推动. 出于同样的原因, 如果想精确测得动量, 我们只能给体系一个很小的推动, 根据普朗克的公式, 这就意味着只能使用低频率的光. 而低频率的光意味着很长的波长, 这样反过来就意味着位置测量的很大不确定性.
\bigskip

\textbf{不确定性与照相术}\\
眼睛就是一个很好的光子探测器: 一个光子就可以激发一个视网膜细胞. 当然, 一般情况下, 很多光子还没有到达视网膜, 就被眼睛吸收了\textit{组成眼睛的很多其他细胞的原子也吸收光子}.
出于这一原因, 大约只有百分之几进入到眼睛的光子真正被眼睛探测到. 显然, 看东西这种化学反应必须是可逆的\textit{否则, 看到一种东西之后, 不回复到之前的状态就看不到新的东西了}---实际上, 大约0.1 秒以后视觉细胞就会回到之前的正常状态. \\
照相术通过把这种化学变化永久保存在感光乳剂里, 克服了眼睛的这种限制.\\
感光胶片里面有很多银化物颗粒, 颗粒里面的银是离子化的. 当光子被乳液吸收的时候, 有时会放出一个电子, 就像光电效应一样. 这个电子会被一个阴离子吸引, 结合成一个电中性的银原子.
游离出来的银原子被含有银离子的化合物包围, 这是不稳定的. 银原子很容易放出电子, 重新变成银离子. 如果在他变成银离子之前, 又有一些光子在他的附近产生了好几个银原子, 这时就会形成一个由几个银原子组成的稳定"显影中心", 而照片就是由很多微小的显影中心形成的.\\
拍照时, 曝光时间极短时, 只有几个光子进入, 那么照片上出现的是一些随机点, 但当曝光时间增加, 进入的光子数也增加, 我们看到显影中心更多的出现在以后图像上很亮的地方. 因此,即使在拍照这样一件非常普通的行动中, 我们也能看到光的量子的, 随机的自然属性.
\bigskip

\textbf{分形: 奇妙的数学}\\
在所有的尺度下看起来都很相似(自相似)\\
当我们观察量子在一段时间内的运动路径时, 发现当我们不断增加放大倍数, 这些量子路径看起来都是锯齿状的,而且都是非常相似的.\\
海岸线是一条分形曲线, 长度取决于我们在什么尺度下测量. 这种现象也是一些百科全书在给出一些国家之间的陆地边界线长度时, 数据不同的部分原因.\\
\textbf{分数维}:描述一条曲线的不规则程度. 一条平滑的曲线的分数维$D=1$, 跟普通的维数一样, 而一条曲线确实不规则, 越是到处有着尖角, 他的分数维就越接近$D=2$, $D=2$ 就是一条非常不规则的曲线, 差不多把一个两维空间都填满了.\\
利用分型理论可以用计算机生成人造风景, 已经是现代电影制造业的一种标准技巧了.

\section{薛定谔和物质波}
$$
E\phi = -\dfrac{\hbar^2}{2m} \frac{d^2 \phi}{dx^2} + V\phi
$$
$$
\lambda = h/p
$$

\section{原子核模型}
电子的数目, 或者等价的说, 质子的数目, 决定了不同元素的化学性质.\par
在通常的温度下, 气体里原子的碰撞一般不足以激发很多的原子, 因为基态与第一激发态之间的能量差很大. 因此在室温下, 大多数原子都处于基态. 基态与任意一个激发态之间的能量差都很大, 相应的光子的频率处于紫外波段, 而不是可见光波段, 因此, 可见光直接穿透很多气体, 而不被吸收. 这就是为什么大多数气体对可见光来说是透明的原因.\par
\textbf{彭宁离子阱}:他能把电子限制在真空中两块带有负电的金属盘中间. 金属盘周围有磁场, 用来防止电子从阱的旁边溢出. 利用一个带负电的金属叉子可以把电子放置到阱中.
通过检测例子在阱中的来回震荡阱, 可以确认粒子的存在. 实际做法是让电子一个一个不断逃出阱外, 知道最后只剩下一个电子.

\section{量子隧道效应}
对时间和能量的测量也存在一个类似的关系
$$
\Delta E \Delta t \geq \hbar
$$
虽然经典意义上, 我们不可能在不违反能量守恒的前提下改变总能量, 但是在量子力学里, 如果时间不确定性是$\Delta t$, 我们无法把能量测量的比$\Delta E = \hbar /\Delta t$精确.
粗略的说, 我们能借到一些能量$\Delta E$来越过壁垒, 只要我们在时间$\Delta t = \hbar / \Delta E$内把能量还回去. 如果势垒太高或者太宽, 隧穿的可能性就会变得非常小. \par

扫描隧道显微镜

\section{泡利与元素}
如果我们让另一个氢原子靠近第一个氢原子, 会发生什么呢? 如果两个电子的自旋都向上, 泡利不相容原理将禁止两个氢原子靠近, 因为两个电子的波函数会重叠, 波函数重叠就意味着两个电子会处于同一个量子态. 如果两个电子的自选方向相反, 他们就可以互相靠近, 并且实际情况是, 两个电子大部分时间都呆在两个氢原子核之间. 这样就在两个氢原子之间产生了一个束缚力, 形成一个稳定的氢分子. 这种化学键, 也就是两个电子被分子中两个原子共享形成的化学键, 叫做共价键. 正是泡利不相容原理, 说明了为什么氢原子是化学活泼的, 为什么两个氢原子能够形成一个氢分子$H_2$. 注意, 同样是泡利不相容原理, 禁止了第三个氢原子与氢分子$H_2$ 再形成共价键, 因为两个能量最低的自旋态已经都被占据了.\par
下一个最简单的原子是氦, 有两个电子, 都填布在最低的1S能级上, 他们的自旋必须相反. 因为1S 能级上已经没有地方容纳更多的电子了, 泡利不相容原理将禁止别的电子靠近氦原子, 就像他对氢分子那样, 因此我们可以预料到氦原子的化学性质很不活泼, 实际上他就是一种惰性气体.

\section{费米子与玻色子}
Toutes les particules \'el\'ementaires d\'ecouvertes \`a ce jour sont soit des bosons, soit des fermions.\par

un grand nombre de particules : les fermions sont des particules qui ob\'eissent \`a la statistique de Fermi-Dirac alors que les bosons ob\'eissent \`a la statistique de Bose-Einstein.\\
la loi statistique Fermi-Dirac : quand on \'echange deux fermions, la fonction d'onde change de signe.
物质粒子都是由费米子构成,而玻色子则是媒介子

\textbf{宏观的玻色子现象}\\
la condensation de Bose-Einstein玻色-爱因斯坦凝聚: 是玻色子原子在冷却到绝对零度附近时所呈现出的一种气态的、超流性的物态,几乎全部原子都聚集到能量最低的量子态,形成一个宏观的量子状态  occupent un unique \'etat quantique de plus basse \'energie (\'etat fondamental)\par
la population macroscopique d'un mode unique de photon dans un laser
大量的光子(玻色子)处于同一个量子状态 \par

玻色子包括:
\begin{itemize}
\item 胶子 - 强相互作用的媒介粒子,自旋为1,有8种
\item 光子 - 电磁相互作用的媒介粒子,自旋为1,只有1种
\item W 及 Z 玻色子 - 弱相互作用的媒介粒子,自旋为1,有3种
\item 引力子 - 引力相互作用的媒介粒子,自旋为2,只有1种,尚未被发现
\item 希格斯玻色子 - 自旋为0,目前只发现1种
\item 介子 - 由两个费米子——夸克组成的强子。
\item 由偶数个核子组成的原子核。因为质子和中子都是费米子,故含偶数个核子的原子核是自旋为整数的玻色子。
\item 声子 - 请参阅固体物理学
\end{itemize}

\chapter{Quantique}
\section{Equation de Schr\"ondinger}
推广到三维情况下,方程为:
$$
\psi \left({\mathbf r}, t \right) = A \cos \left({\mathbf k} \cdot {\mathbf r} - \omega t + \varphi \right)
$$
其中:
r是三维空间中的位置矢量;
$\cdot$ 是矢量点积;
k是波矢。
这一方程描述了平面波。一维情况下,波矢的大小是角波数
$$|{\mathbf k}| = 2\pi/\lambda$$
波矢的方向是平面波行进的方向.

波动方程是双曲形偏微分方程的最典型代表,其最简形式可表示为:关于位置$x$ 和时间$t$ 的标量函数$u$(代表各点偏离平衡位置的距离)满足:
$$
{ \partial^2 u \over \partial t^2 } = c^2 \nabla^2u
$$
这里c通常是一个固定常数,代表波的传播速率.
在针对实际问题的波动方程中,一般都将波速表示成可随波的频率变化的量,这种处理对应真实物理世界中的色散现象。此时,$ c$  应该用波的相速度代替:$v_\mathrm{p} = \frac{\omega}{k}$

上面的波动方程可以化成下面的形式
$$
\left(\nabla^2-\frac{1}{c^2}\frac{\partial^2}{\partial{t}^2}\right)u(\mathbf{r},t)=0.
$$
Separation of variables begins by assuming that the wave function $u(r, t)$ is in fact separable:
$$u(\mathbf{r},t)=A (\mathbf{r}) T(t)$$
Substituting this form into the wave equation, and then simplifying, we obtain the following equation:
$${\nabla^2 A \over A } = {1 \over c^2 T } { d^2 T \over d t^2  }.$$
Notice the expression on the left-hand side depends only on $ r$ , whereas the right-hand expression depends only on $ t$ . As a result, this equation is valid in the general case if and only if both sides of the equation are equal to a constant value. From this observation, we obtain two equations, one for $ A(r)$ , the other for $ T(t)$ :
$${\nabla^2 A \over A } = -k^2$$
and
$$ {1 \over c^2 T } { d^2 T \over dt^2  } = -k^2$$
Rearranging the first equation, we obtain the Helmholtz equation:
$$\nabla^2 A + k^2 A  =  ( \nabla^2 + k^2)  A  =  0$$
Likewise, after making the substitution
$$ \omega  \stackrel{\mathrm{def}}{=}  kc $$
the second equation becomes
$$\frac{d^2{T}}{d{t}^2} + \omega^2T  =  \left( { d^2 \over dt^2 } + \omega^2 \right) T  =  0,$$
where $k$ is the wave vector and $\omega$ is the angular frequency.\par
我们可以解出
$$
T(t) = \exp^{-i \omega t}
$$

回到第一个方程\par
La longueur d'onde dans le milieu est d\'efinie par $\lambda = 2 \pi/k$. L'\'equation de Helmholtz se r\'e\'ecrit :

$$
\left( \, \Delta \ + \ \frac{4\pi^2}{\lambda^2} \, \right) \ \psi(\vec{r}) \ = \ 0
$$

On utilise alors la relation de de Broglie pour une particule non relativiste, pour laquelle la quantit\'e de mouvement $p = m v$  :
$$
\lambda \ = \ \frac{h}{mv} \quad \Longrightarrow \quad \frac{1}{\lambda^2} \ = \ \frac{m^2 \, v^2}{h^2}
$$

Or, l'\'energie cin\'etique s'\'ecrit pour une particule non relativiste :
$$
\frac{1}{2} \, m \, v^2 \ = \ E \ - \ V(\vec{r})
$$

d'où l'\'equation de Schr?dinger stationnaire :
$$
\left[ \, \Delta \ + \ \frac{8\pi^2m}{h^2} \, \left( \, E \ - \ V(\vec{r}) \, \right) \ \right] \ \psi(\vec{r}) \ = \ 0
$$

En introduisant le quantum d'action $\hbar = h/2\pi$, on la met sous la forme habituelle :
$$
- \ \frac{\hbar^2}{2m} \, \Delta \psi(\vec{r}) \ + \ V(\vec{r}) \, \psi(\vec{r}) \ = \ E \, \psi(\vec{r})
$$

Il ne reste plus qu'\`a r\'eintroduire le temps t en explicitant la d\'ependance temporelle pour une onde monochromatique, puis en utilisant la relation de Planck-Einstein $E = \hbar \omega$:

$$
\psi(\vec{r},t) \ = \ \psi(\vec{r}) \ \mathrm{e}^{- \, i \, \omega \, t} \ = \ \psi(\vec{r}) \ \exp \left( - \, \frac{i \, E \, t}{\hbar} \right)
$$

On obtient finalement l'\'equation de Schr\"odinger g\'en\'erale :
$$
- \ \frac{\hbar^2}{2m} \, \Delta \psi(\vec{r},t) \ + \ V(\vec{r}) \, \psi(\vec{r},t) \ = \ i \, \hbar \ \frac{\partial \psi(\vec{r},t)}{\partial t}
$$


L'\'electron est trait\'e comme une onde $\Psi(r,t)$ se d\'epla\c cant dans un puits de potentiel $V$.
La densit\'e de probabilit\'e $\rho(r,t)$ qui lui est associ\'ee est d\'efinie par
$$
\rho(r,t)= R(r,t)^2 = |\Psi(r,t)|^2=\Psi^*(\bold{r},t)\Psi(\bold{r},t)\,\!
$$
(probabilit\'e par unit\'e de volume, * indique un complexe conjugu\'e)
\end{document}
