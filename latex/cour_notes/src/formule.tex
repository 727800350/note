% !Mode:: "TeX:UTF-8"
\chapter{Formule}
\section{Tylor series}
$$
e^x
= 1+x+\frac{ x^2}{2!}
+ \ldots +
\frac{ x^n}{n!}
+0(x^n)
=\sum_{k=0}^{+\infty}\frac{ x^k}{k!}
\quad x \in \R
$$

$$
\sin x=x-\frac{ x^3}{3!}+\frac{ x^5}{5!}+\ldots+ (-1)^n \frac{x^{2n+1}}{(2n+2)!}+
0(x^{2n+1})
=
\sum_{n=0}^{+\infty} \frac{(-1)^n}{(2n+1)!} x^{2n+1}
\quad x \in \R
$$

$$
\cos x =
1 - \frac{ x^2}{2!} + \frac{ x^4}{4!}
+ \ldots +
(-1)^n \frac{ x^{2n}}{(2n)!} + 0(x^{2n+1})
\sum_{n=0}^{+\infty} \frac{(-1)^n}{(2n)!} x^{2n}
\quad x \in \R
$$

$$
\sinh x=x + \frac{ x^3}{3!}+\frac{ x^5}{5!}+\ldots+ \frac{x^{2n+1}}{(2n+1)!}+
0(x^{2n+2})
=
\sum_{n=0}^{+\infty} \frac{x^{2n+1}}{(2n+1)!}
\quad x \in \R
$$

$$
\cosh x =
1 + \frac{ x^2}{2!} + \frac{ x^4}{4!}
+ \ldots +
\frac{ x^{2n}}{(2n)!} + 0(x^{2n+1})
$$

$$
(1+x)^\alpha = 1+\frac{ \alpha}{1!}x + \frac{ \alpha(\alpha-1)}{2!}x^2
+\ldots+
\frac{ \alpha(\alpha-1)\ldots(\alpha-n+1)}{n!}x^n+
0(x^n)
$$

$$
\frac{ 1}{1+x}= 1 -x +x^2-x^3+\ldots +(-1)^n x^n +0(x^n)=\sum_{k=0}^{+\infty}(-1)^kx^k
$$

$$
\frac{ 1}{1-x}= 1 +x +x^2-x^3+\ldots +x^n +0(x^n)=\sum_{k=0}^{+\infty}x^k
$$

$$
\sqrt{1+x}= 1+\frac{ x}{2}-\frac{ 1}{2\cdot 4}x^2+ \frac{ 1\cdot3}{2\cdot4\cdot6}x^3
+\ldots+
(-1)^{n-1} \frac{1\cdot3\ldots(2n-3)}{2\cdot4\ldots(2n)}x^n +0(x^n)
$$

$$
\sqrt{1-x}= 1-\frac{ x}{2}-\frac{ 1}{2\cdot 4}x^2- \frac{ 1\cdot3}{2\cdot4\cdot6}x^3
-\ldots-
(-1)^{n-1} \frac{1\cdot3\ldots(2n-3)}{2\cdot4\ldots(2n)}x^n -0(x^n)
$$

$$
\frac{ 1}{\sqrt{1+x}}
= 1-\frac{ x}{2}+\frac{ 1\cdot3}{2\cdot 4}x^2
+\ldots+
(-1)^{n-1} \frac{1\cdot3\ldots(2n-1)}{2\cdot4\ldots(2n)}x^n +0(x^n)
$$

$$
\frac{ 1}{\sqrt{1-x}}
= 1+\frac{ x}{2}+\frac{ 1\cdot3}{2\cdot 4}x^2
+\ldots+
\frac{1\cdot3\ldots(2n-1)}{2\cdot4\ldots(2n)}x^n +0(x^n)
$$

$$
\log (1+x)=
x-\frac{ x^2}{2}+\frac{ x^3}{3}+\ldots+ (-1)^{n-1} \frac{ x^n}{n}+0(x^n)
=\sum_{k=1}^{+\infty}\frac{(-1)^{n-1}x^n}{n}
\quad x\in(-1,1]
$$

$$
\arctan x=x-\frac{ x^3}{3}+\frac{ x^5}{5}+\ldots+ (-1)^n \frac{x^{2n+1}}{2n+1}+
0(x^{2n+2})
=
\sum_{n=0}^{+\infty} \frac{(-1)^n}{2n+1} x^{2n+1}
\quad x\in[-1,1]
$$

$$
\arcth x=x+\frac{ x^3}{3}+\frac{ x^5}{5}+\ldots+ \frac{x^{2n+1}}{2n+1}+
0(x^{2n+2})
=
\sum_{n=0}^{+\infty} \frac{1}{2n+1} x^{2n+1}
\quad x\in[-1,1]
$$

\red{arcsin arcsh tan待验证}

$$
\arcsin x=
x+ \frac{ 1}{2\cdot3}x^3+\frac{1\cdot3}{2\cdot4\cdot5}x^5
+\ldots+
\frac{ 1\cdot3\cdot(2n-1)}{2\cdot4\cdot(2n)\cdot(2n+1)}x^{2n+1}+0(x^{2n+2})
$$

$$
\arcsh x=
x-\frac{ 1}{2\cdot3}x^3+\frac{1\cdot3}{2\cdot4\cdot5}x^5
+\ldots+
\frac{ (-1)^n \cdot1\cdot3\cdot(2n-1)}{2\cdot4\cdot(2n)\cdot(2n+1)}x^{2n+1}+0(x^{2n+2})
$$

$$
\tan x=
x+\frac{ x^3}{3}+\frac{2}{15}x^5+\frac{ 11}{315}x^7+0(x^8)
$$


\section{常用的初等数学公式}
\subsection{有限项数项级数}
$$
1^2 + 2^2 +3^2+\ldots+n^2=\frac{ n(n+1)(n+2)}{6}
$$

$$
1^3 + 2^3 +3^3+\ldots+n^3=\frac{ n^2(n+1)^2}{4}
$$

$$
1^2+3^2+5^2+\ldots+(2n-1)^2=\frac{n(4n^2-1)}{3}
$$

$$
1^3 + 3^3 +5^3+\ldots+(2n-1)^3=n^2(2n^2-1)
$$

\subsection{双曲函数}
$$
\sh x=\frac{e^x - e^{-x}}{2}, \quad \ch x=\frac{e^x+e^{-x}}{2}, \quad \dth x = \frac{ \sh x}{\ch x}=\frac{e^x-e^{-x}}{e^x + e^{-x}}
$$

\subsection{三角}
$$
\sin ^2 \alpha + \cos^2 \alpha=1
$$

$$
\frac{ \sin x}{\cos x}=\tan x, \quad
\frac{ \cos x}{\sin x}=\cot x, \quad
\csc x=\frac{1}{\sin x}, \quad
\sec x=\frac{ 1}{\cos x}, \quad
1+\tan^2 x=\sec^2 x,\quad
1+\cot^2 x=\csc^2 x
$$

\subsubsection{万能公式}
$$
\tan \frac{x}{2}=t \Rightarrow
\cos x =\frac{ 1-t^2}{1+t^2},\quad
\sin x=\frac{2t}{1+t^2},\quad
dx=\frac{2}{1+t^2}dx
$$
\subsubsection{和差公式}
$$
\sin(x\pm y)=\sin x \cos y \pm \cos x \sin y,\quad \cos(x \pm y)=\cos x \cos y \mp \sin x \sin y
$$

$$
\tan(x \pm y)=\frac{\tan x \pm \tan y}{1 \mp \tan x \tan y}, \quad
\cot (x\pm y)=\frac{\cot x \cot y \mp 1}{\cot y \pm \cot x}
$$

$$
\sin x+\sin y=2\sin \frac{x+y}{2} \cos \frac{ x -y }{2}, \quad
\sin x-\sin y=2\cos \frac{x+y}{2} \sin \frac{ x -y }{2}
$$

$$
\cos x+\cos y=2\cos \frac{x+y}{2} \cos \frac{ x -y }{2}, \quad
\cos x-\cos y=2\sin \frac{x+y}{2} \sin \frac{ x -y }{2}
$$

$$
\cos x \cos y=\frac{ 1}{2}[\cos(x-y)+\cos(x+y)]
$$

$$
\sin x \sin y=\frac{ 1}{2}[\cos(x-y)-\cos(x+y)]
$$

$$
\sin x \cos y=\frac{ 1}{2}[\sin(x-y)+\sin(x+y)]
$$

\subsubsection{被角公式}
$$
\sin 2x=2\sin x \cos x,\quad \cos 2x=\cos^2 x-\sin ^2 x=1-2\sin^2 x=2\cos^2 x -1
$$

\subsubsection{任意三角形的基本关系}
$$
\frac{a}{\sin A}=
\frac{b}{\sin B}=
\frac{c}{\sin C}=
2R(\mbox{正弦定理})
$$

$$
a^2=b^2+c^2-2bc\cos A (\mbox{余弦定理})
$$

$$
S= \frac{ 1}{2}ab\sin C=\sqrt{p(p-a)(p-b)(p-c)}~\mbox{where}~p=\frac{1}{2}(a+b+c)~ (\mbox{面积公式})
$$

\subsection{初等几何}
\textbf{扇形:}
弧长$l=\alpha \cdot R$, \quad
面积$S=\dfrac{1}{2}l\cdot R=\dfrac{1}{2}\alpha R^2$

\textbf{正圆锥}
体积$V=\dfrac{1}{3}\pi r^3 h$ , \quad
侧面积$S=\pi rl$, \quad
全面积$A=\pi r(r+l)$

\textbf{截圆锥}
体积$V=\dfrac{\pi h}{3}(R^2+r^2+Rr)$, \quad
侧面积$S=\pi l(R+r)$

\subsection{中值定理}
拉格朗日中值定理: $f(b)-f(a)=f'(\xi)(b-a)$

柯西中值定理: $$\frac{f(b)-f(a)}{g(b)-g(a)}=\frac{f'(\xi) }{g'(\xi)}$$
\subsection{曲率}
弧微分公式: $ds=\sqrt{1+y'^2}dx$,其中$y'=\tan \alpha$

平均曲率: $K=|\frac{\Delta\alpha}{\Delta s}|$ 其中$\Delta \alpha$:从$M$点到$M'$点,切线斜率的倾角变化量; $\Delta s$:$MM'$弧长

M点的曲率: $K=lim_{\Delta s \to 0}|\frac{\Delta\alpha}{\Delta s}|=|\frac{d \alpha}{ds}|=\frac{|y''|}{\sqrt{(1+y'^2)^3}}$

直线: $K=0$

半径为$a$的圆: $K=\frac{1}{a}$

\section{积分表}
\textbf{导数公式}
$$
(\tan x)'=\sec ^2 x,\quad
(\cot x)'=-\csc^2 x,\quad
(a^x)'=a^x\ln a,\quad
(\log_a x)'=\frac{1}{x \ln a}
$$

$$
(\arcsin x)'=\frac{1}{\sqrt{1-x^2}}, \quad
(\arccos x)'=-\frac{1}{\sqrt{1-x^2}}, \quad
(\arctan x)'=\frac{1}{1+x^2}, \quad
(\arccot x)'=-\frac{1}{1+x^2}
$$

$$
d(\frac{1}{|x|})=-\frac{1}{x|x|}dx, \quad |x|=sgn(x)\cdot x
$$

\textbf{含有$ax+b$的积分}

$$
\int \frac{dx}{ax+b}=\frac{1}{a}\ln|ax+b|+C
$$

$$
\int \frac{dx}{x(ax+b)}=-\frac{1}{b}\ln|\frac{ax+b}{x}|+C
$$

$$
\int \frac{dx}{x^2(ax+b)}=-\frac{ 1}{bx}+\frac{ a}{b^2}\ln |\frac{ax+b}{x}|+C
$$

\textbf{含有$\sqrt{ax+b}$的积分}
$$
\int \sqrt{ax+b}dx=\frac{2}{3a}\sqrt{(ax+b)^3}+C
$$

$$
\int x\sqrt{ax+b}dx=\frac{2}{15a^2}(3ax-2b)\sqrt{(ax+b)^3}+C
$$

$$
\int \frac{x}{\sqrt{ax+b}}dx=\frac{2}{3a^2}(ax-2b)\sqrt{ax+b}+C
$$

\textbf{含有$x^2 \pm a^2$的积分}
$$
\int \frac{dx}{x^2+a^2}=\frac{1}{a}\arctan \frac{x}{a}+C
$$
$$
\int \frac{dx}{x^2-a^2}=\frac{1}{2a}\ln |\frac{x-a}{x+a}|+C
$$
\textbf{含有$ax^2+b(a>0)$的积分}
$$
\int \frac{dx}{ax^2+b}=
\left\{
		\begin{array}{ll}
			\frac{1}{\sqrt{ab}}\arctan \sqrt{\frac{a}{b}}x+C & (b>0) \\
			\frac{1}{2\sqrt{-ab}}\ln|\frac{\sqrt{a}x-\sqrt{-b}}{\sqrt{a}x+\sqrt{-b}}+C & (b<0)
		\end{array}
		\right.
$$

$$
\int \frac{x}{ax^2+b}dx=\frac{1}{2a}\ln |ax^2+b|+C
$$

$$
\int \frac{dx}{x(ax^2+b)}=\frac{1}{2b}\ln \frac{x^2}{|ax^2+b|}+C
$$

\textbf{含有$ax^2+bx+c(a>0)$的积分}
$$
\int \frac{ dx}{ax^2+bx+c}=
\left\{
		\begin{array}{ll}
		 \frac{2}{\sqrt{4ac-b^2}}\arctan \frac{2ax+b}{\sqrt{4ac-b^2}}+C & (b^2<4ac) \\
		 \frac{1}{\sqrt{b^2-4ac}}\ln|\frac{2ax+b-\sqrt{b^2-4ac}}{2ax+b+\sqrt{b^2-4ac}}|+C & (b^2>4ac)
		\end{array}
		\right.
$$

$$
\int \frac{x}{ax^2+bx+c}dx=\frac{1}{2a}\ln|ax^2+bx+c|-\frac{b}{2a}\int \frac{dx}{ax^2+bx+c}
$$

\textbf{含有$\sqrt{x^2+a^2}(a>0)$的积分}
$$
\int \frac{dx}{\sqrt{x^2+a^2}}=\arcsh \frac{x}{a}+C_1=\ln(x+\sqrt{x^2+a^2})+C,\quad \big( \ln(x+\sqrt{1+x^2})\big)'=\frac{1}{\sqrt{1+x^2}}
$$

$$
\int \sqrt{x^2+a^2}dx=\frac{x}{2}\sqrt{x^2+a^2}+\frac{a^2}{2}\ln(x+\sqrt{x^2+a^2})+C
$$

\textbf{含有$\sqrt{x^2-a^2}(a<0)$的积分}
$$
\int \frac{dx}{\sqrt{x^2-a^2}}=\frac{x}{|x|}\arch \frac{|x|}{a}+C_1=
\ln|x+\sqrt{x^2-a^2}|+C
$$

$$
\int \frac{x}{\sqrt{x^2-a^2}}dx=\sqrt{x^2-a^2}+C
$$

$$
\int \frac{dx}{x\sqrt{x^2-a^2}}=\frac{1}{a}\arccos \frac{a}{|x|}+C
$$

\textbf{含有$\sqrt{a^2-x^2}(a>0)$的积分}
$$
\int \frac{dx}{\sqrt{a^2-x^2}}=\arcsin \frac{x}{a}+C
$$

$$
\int \frac{x}{\sqrt{a^2-x^2}}dx=-\sqrt{a^2 - x^2}+C
$$

$$
\int \frac{dx}{x\sqrt{a^2-x^2}}=\frac{1}{a}\ln \frac{a-\sqrt{a^2 - x^2}}{|x|}+C
$$

\textbf{含有$\sqrt{\pm ax^2+bx+c}(a>0)$的积分}
$$
\int \frac{dx}{\sqrt{ax^2+bx+c}}=\frac{1}{\sqrt{a}}
\ln|2ax+b+2\sqrt{a}\sqrt{ax^2+bx+c}+C|
$$
$$
\int \sqrt{ax^2+bx+c}dx=\frac{2ax+b}{4a}\sqrt{ax^2+bx+c}+
\frac{4ac-b^2}{8\sqrt{a^3}}
\ln|2ax+b+2\sqrt{a}\sqrt{ax^2+bx+c}+C|
$$
\textbf{含有$\sqrt{\pm \frac{x-a}{x-b}}$或$\sqrt{(x-a)(x-b)}$的积分}
$$
\int \sqrt{\frac{x-a}{x-b}}dx=(x-b)\sqrt{\frac{x-a}{x-b}}+
(b-a)\ln (\sqrt{|x-a|}+\sqrt{|x-b|})+C
$$

$$
\int \sqrt{\frac{x-a}{b-x}}dx=(x-b)\sqrt{\frac{x-a}{b-x}}+
(b-a)\arcsin \sqrt{\frac{x-a}{b-a}}+C
$$

$$
\int \frac{dx}{\sqrt{(x-a)(b-x)}}=2\arcsin \sqrt{\frac{x-a}{b-a}}+C \quad(a<b)
$$

$$
\int \sqrt{(x-a)(b-x)}=\frac{2x-a-b}{4}\sqrt{(x-a)(b-x)}
+ \frac{(b-a)^2}{4}\arcsin \sqrt{\frac{x-a}{b-a}}+C \quad(a<b)
$$
\textbf{含有三角函数的积分}
$$
\int \sin xdx=-\cos x +C
$$

$$
\int \cos xdx =\sin x +C
$$

$$
\int \tan xdx=-\ln|\cos x|+C
$$

$$
\int \cot xdx=\ln|\sin x|+C
$$

$$
\int \sec xdx=\ln|\tan(\frac{\pi}{4}+\frac{x}{2})|+C=\ln|\sec x+ \tan x|+C
$$

$$
\int \csc xdx=\ln|\tan \frac{x}{2}|=\ln|\csc x-\cot x|+C
$$


$$
\int \sin^n x dx=-\frac{1}{n}\sin^{n-1}x \cos x + \frac{n-1}{n}\int \sin^{n-2}x dx
$$
$$
\int \cos^n x dx=\frac{1}{n}\cos^{n-1}x \sin x + \frac{n-1}{n}\int \cos^{n-2}x dx
$$


$$
\int \frac{dx}{a+b\sin x}=\frac{2}{\sqrt{a^2-b^2}}\arctan \frac{a \tan \frac{x}{2}+b}{\sqrt{a^2-b^2}}+C \quad(a^2>b^2)
$$

$$
\int \frac{dx}{a+b\cos x}=\frac{2}{a+b} \sqrt{\frac{a+b}{a-b}}\arctan(\sqrt{\frac{a-b}{a+b}}\tan \frac{x}{2})+C \quad (a^2>b^2)
$$


$$
\int \frac{ dx}{a^2 \cos^2 x+b^2\sin^2 x}=\frac{ 1}{ab} \arctan(\frac{b}{a}\tan x)+C, \quad
\frac{\frac{1}{\cos^2}dx}{a^2+b^2\tan^2 x}=\frac{d \tan x}{a^2 + b^2 \tan^2 x}
$$

\textbf{含有反三角函数的积分(其中($a>0$))}
$$
\int \arcsin \frac{x}{a}dx=x\arcsin \frac{x}{a}+\sqrt{a^2-x^2}+C
$$

$$
\int x\arcsin \frac{x}{a}dx=(\frac{x^2}{2}-\frac{ a^2}{4})\arcsin \frac{x}{a}+ \frac{x}{4}\sqrt{a^2-x^2}+C
$$

$$
\int \arccos \frac{x}{a}dx=x \arccos \frac{x}{a}-\sqrt{a^2-x^2}+C
$$

$$
\int x\arccos \frac{x}{a}dx=(\frac{x^2}{2}-\frac{ a^2}{4})\arccos \frac{x}{a}- \frac{x}{4}\sqrt{a^2-x^2}+C
$$

$$
\int \arctan \frac{x}{a}dx=x \arctan \frac{x}{a}-\frac{a}{2}\ln(a^2+x^2)+C
$$

$$
\int x \arctan \frac{x}{a}dx=\frac{1}{2}(a^2+x^2)\arctan \frac{x}{a}- \frac{a}{2}x+C
$$
\textbf{含有指数函数的积分}
$$
\int a^xdx=\frac{1}{\ln a}a^x+C
$$
$$
\int e^{ax}dx=\frac{ 1}{a}e^{ax}+C
$$

\textbf{含有对数函数的积分}
$$
\int \ln xdx=x\ln x-x +C
$$

$$
\int (\ln x)^n dx=x(\ln x)^n -n \int (\ln x)^{n-1}dx
$$
\textbf{含有双曲函数的积分}
$$
\int \sh xdx=\ch x+C
$$

$$
\int \ch xdx=\sh x+C
$$

$$
\int \dth xdx=\ln \ch x+C
$$
\textbf{定积分}
$$
\int_{-\pi}^{\pi} \cos nx dx=
\int_{-\pi}^{\pi} \sin nx dx=0
$$
$$
\int_{-\pi}^{\pi} \cos mx\sin nx dx=0
$$

$$
\int_{-\pi}^{\pi} \cos mx\cos nx dx=
\left\{
		\begin{array}{ll}
		 0 & m \neq n \\
		 \pi & m = n
		\end{array}
		\right.
$$

$$
\int_{-\pi}^{\pi} \sin mx\sin nx dx=
\left\{
		\begin{array}{ll}
		 0 & m \neq n \\
		 \pi & m = n
		\end{array}
		\right.
$$

$$
\int_{0}^{\pi} \sin mx\sin nx dx=
\int_{0}^{\pi} \cos mx\cos nx dx=
\left\{
		\begin{array}{ll}
		 0 & m \neq n \\
		 \pi & m = n
		\end{array}
		\right.
$$

$$
I_n=\int_{0}^{\pi/2} \sin^n xdx=\int_{0}^{\pi/2}\cos^n xdx=\frac{n-1}{n}I_{n-2},\quad
I_1=1, I_0=\frac{ \pi}{2}
$$

\section{Transformations}
\subsection{Dirac}
Soit
\begin{equation}
		 \delta _{\varepsilon}(t)=
\left\{
		\begin{array}{ll}
		0 & t<0 \\
		\frac{ 1}{\varepsilon} & 0\leq t \leq \varepsilon \\
		0 & t> \varepsilon
		\end{array}
		\right.
\end{equation}
\begin{definition}
\begin{equation}
\delta (t)=\lim_{\varepsilon \to 0}\delta _{\varepsilon}(t)=
\left\{
		\begin{array}{ll}
		 0 & t\neq 0 \\
		 \infty & t=0
		\end{array}
		\right.
\end{equation}
$$
\int_{-\infty}^{+\infty}\delta (t)dt
=\lim_{\varepsilon \to 0}\int_{-\infty}^{+\infty} \delta_{\varepsilon}(t)dt
=\lim_{\varepsilon \to 0}\int_{0}^{\varepsilon}\frac{1}{\varepsilon}dt
=1
$$
\end{definition}
Propri\'et\'es:

\subsection{Transformation de Fourrier}
\textbf{Serie de Fourrier}
给定一个周期为T的函数x(t),那么它可以表示为无穷级数:
\begin{equation}
		x(t)=\sum _{k=-\infty}^{+\infty}a_k\cdot e^{ik(\frac{2\pi}{T})t}\eqspace (i\text{为虚数单位})
\label{serie.fourrier.fonction}
\end{equation}
其中,$a_k$可以按下式计算:
$$
a_k=\frac{1}{T}\int_{T}x(t)\cdot e^{-ik(\frac{2\pi}{T})t}dt
$$
注意到$f_k(t)=e^{ik(\frac{2\pi}{T})t}$是周期为T的函数,故k 取不同值时的周期信号具有谐波关系(即它们都具有一个共同周期T).\newline
$k=0$时,\eqref{serie.fourrier.fonction} 式中对应的这一项称为直流分量,也就是x(t)在整个周期的平均值.\newline
$k=\pm 1$时具有基波频率$\omega_0=\frac{2\pi}{T}$,称为一次谐波或基波,类似的有二次谐波,三次谐波等等。

\textbf{三角函数族的正交性}
所谓的两个不同向量正交是指它们的内积为0,这也就意味着这两个向量之间没有任何相关性,例如,在三维欧氏空间中,互相垂直的向量之间是正交的。事实上,正交是垂直在数学上的一种抽象化和一般化。一组n个互相正交的向量必然是线性无关的,所以必然可以张成一个n维空间,也就是说,空间中的任何一个向量可以用它们来线性表出。三角函数族的正交性用公式表示出来就是:
$$\int _{0}^{2\pi}\sin (nx)\cos (mx) \,dx=0;$$
$$\int _{0}^{2\pi}\sin (nx)\sin (mx) \,dx=0;(m\ne n)$$
$$\int _{0}^{2\pi}\cos (nx)\cos (mx) \,dx=0;(m\ne n)$$
$$\int _{0}^{2\pi}\sin (nx)\sin (nx) \,dx=\pi;$$
$$\int _{0}^{2\pi}\cos (nx)\cos (nx) \,dx=\pi;$$
\bigskip

$$F(w)=\int_{-\infty}^{+\infty}f(t)e^{-iwt}dt$$
称为$f(t)$的Fourrier变换,记为$F[f(t)]$
$$\frac{ 1}{2\pi}\int_{-\infty}^{\infty}F(w)e^{iwt}dw$$称为$F(w)$的Fourrier逆变换,记为$F^{-1}[F(w)]$
$$f(t) \longleftrightarrow F(w)$$一一对应,称为一组Fourrier变换对,其中$f(t)$称为原像函数,$F(w)$称为像函数.
\begin{example}
\begin{equation}
f(t)=
\left\{
		\begin{array}{ll}
			1 & |t| \leq 1 \\
			0 & |t| > 1
		\end{array}
		\right.
\end{equation}
\begin{eqnarray}
 F(w)=\int_{-\infty}^{+\infty} f(t)e^{-iwt}dt=\int_{-1}^{+1}e^{-iwt}=\frac{2\sin w}{w}
\end{eqnarray}
%% \includegraphics[width=\textwidth]{sinx_x}
\end{example}

\begin{example}
\begin{equation}
f(t)=
\left\{
		\begin{array}{ll}
			0 & t \leq 1 \\
			0 & |t| > 1
		\end{array}
		\right.
\end{equation}
\begin{eqnarray}
 F(w)=\int_{-\infty}^{+\infty} f(t)e^{-iwt}dt=\int_{-1}^{+1}e^{-iwt}=\frac{2\sin w}{w}
\end{eqnarray}
\end{example}

\section{Other}
负数的对数
$$e^{i\theta}=cos\theta+isin\theta$$
当$\theta=\pi$时,得到:$e^{i\pi}=-1\Rightarrow ln(-1)=i\pi$。
这样,我们可以计算任意负数的自然对数,例如:
$$ln(-5)=ln[(-1)*5]=ln(-1)+ln5=i\pi+ln5$$
(注,这里没有考虑$2k\pi$的周期)

