% !Mode:: "TeX:UTF-8"
\documentclass{article}
% !Mode:: "TeX:UTF-8"
\usepackage[english]{babel}
\usepackage[UTF8]{ctex}
\usepackage{amsmath, amsthm, amssymb}

% Figure
\usepackage{graphicx}
\usepackage{float} %% H can fix the location
\usepackage{caption}
\usepackage[format=hang,singlelinecheck=0,font={sf,small},labelfont=bf]{subfig}
\usepackage[noabbrev]{cleveref}
\captionsetup[subfigure]{subrefformat=simple,labelformat=simple,listofformat=subsimple}
\renewcommand\thesubfigure{(\alph{subfigure})}

\usepackage{epstopdf} %% convert eps to pdf
\DeclareGraphicsExtensions{.eps,.mps,.pdf,.jpg,.png} %% bmp, gif not supported
\DeclareGraphicsRule{*}{eps}{*}{}
\graphicspath{{figure/}{../figure/}} %% fig directorys

%% \usepackage{pstricks} %% a set of macros that allow the inclusion of PostScript drawings directly inside TeX or LaTeX code
%% \usepackage{wrapfig} %% Wrapping text around figures

% Table
\usepackage{booktabs} %% allow the use of \toprule, \midrule, and \bottomrule
\usepackage{tabularx}
\usepackage{multirow}
\usepackage{colortbl}
\usepackage{longtable}
\usepackage{supertabular}

\usepackage[colorinlistoftodos]{todonotes}

% Geometry
\usepackage[paper=a4paper, top=1.5cm, bottom=1.5cm, left=1cm, right=1cm]{geometry}
%% \usepackage[paper=a4paper, top=2.54cm, bottom=2.54cm, left=3.18cm, right=3.18cm]{geometry} %% ms word
%% \usepackage[top=0.1cm, bottom=0.1cm, left=0.1cm, right=0.1cm, paperwidth=9cm, paperheight=11.7cm]{geometry} %% kindle

% Code
%% \usepackage{alltt} %% \textbf can be used in alltt, but not in verbatim

\usepackage{listings}
\lstset{
    backgroundcolor=\color{white},
    columns=flexible,
    breakatwhitespace=false,
    breaklines=true,
    captionpos=tt,
    frame=single, %% Frame: show a box around, possible values are: none|leftline|topline|bottomline|lines|single|shadowbox
    numbers=left, %% possible values are: left, right, none
    numbersep=5pt,
    showspaces=false,
    showstringspaces=false,
    showtabs=false,
    stepnumber=1, %% interval of lines to display the line number
    rulecolor=\color{black},
    tabsize=2,
    texcl=true,
    title=\lstname,
    escapeinside={\%*}{*)},
    extendedchars=false,
    mathescape=true,
    xleftmargin=3em,
    xrightmargin=3em,
    numberstyle=\color{gray},
    keywordstyle=\color{blue},
    commentstyle=\color{dkgreen},
    stringstyle=\color{mauve},
}

% Reference
%% \bibliographystyle{plain} % reference style

% Color
\usepackage[colorlinks, linkcolor=blue, anchorcolor=red, citecolor=green, CJKbookmarks=true]{hyperref}
\usepackage{color}
\def\red#1{\textcolor[rgb]{1.00,0.00,0.00}{#1}}
\newcommand\warning[1]{\red{#1}}

% Other
%% \usepackage{fixltx2e} %% for use of \textsubscript
%% \usepackage{dirtree}  %% directory structure, like the result of command tree in bash shell


% !Mode:: "TeX:UTF-8"
% Equation Number
\makeatletter\@addtoreset{equation}{subsection}\makeatother %% reset the equation number in subsection
\renewcommand\theequation{\thepart\arabic{section}-\thepart\arabic{subsection}-\thepart\arabic{equation}} %% section-subsection-equation style

% Theorem
\newtheorem{definition}{D\'efintion} %% document global number
\newtheorem*{thmwn}{Thm} %% without numbers
\newtheorem{theorem}{Th\'eor\`eme}[section] %% section-theorem style
\newtheorem{corollary}{Corollary}[theorem] %% theorem-corollary style
\newtheorem{lemma}{Lemma}
\newtheorem{proposition}{Proposition}[section]
\newtheorem*{attention}{Attention}
\newtheorem*{note}{Note}
\newtheorem*{remark}{Remark}
\newtheorem{example}{Example}
\newtheorem{question}{Question}[section]
\newtheorem{problem}{Problem}
\newtheorem*{answer}{Answer}
\newtheorem{fact}{Fact}

% Header and Footer
\newif\ifheader
\headerfalse
\ifheader
	\setlength\textheight{100.0pt}
	\setlength\textwidth{430.0pt}
	\usepackage{fancyhdr}
	\usepackage{lastpage} %% \pageref{LastPage}: get the last page
	\usepackage{pdfpages}
	\usepackage{layout}
	\footskip = 20pt
	\pagestyle{fancy}
	\fancyhead{} %% clear all fields
%% 	\newsavebox{\headpic}
%% 	\sbox{\headpic}{\includegraphics[height=1cm]{logo}} %% set header logo
%% 	\lhead{\usebox{\headpic}}
	\rhead{\small\leftmark}
	\fancyfoot{} %% clear all fields
	\cfoot{\thepage}
	\renewcommand{\headrulewidth}{1pt}  %% header line width, can be set 0 to get rid of it
	\setlength{\skip\footins}{0.5cm}    %% distance between of footnote and body text
	\renewcommand{\footnotesize}{}      %% footnote size
\fi


\begin{document}
\title{Electromagn\'etisme}
\author{}
\maketitle
\tableofcontents
\newpage

\subsection{Identit\'es vectorielles}
Les identit\'es suivantes peuvent être utiles en analyse vectorielle.
\href{http://fr.wikipedia.org/wiki/Identit%C3%A9s_vectorielles}{wiki}

$$ \mathbf a \cdot (\mathbf b \times \mathbf c) = \mathbf b\cdot(\mathbf c \times \mathbf a) = \mathbf c\cdot(\mathbf a\times \mathbf b) $$
$$ \mathbf a\times (\mathbf b\times \mathbf c) = (\mathbf c \times \mathbf b) \times \mathbf a = \mathbf b (\mathbf a \cdot \mathbf c) - \mathbf c(\mathbf a\cdot \mathbf b) $$
$$ (\mathbf a \times \mathbf b)\cdot(\mathbf c \times \mathbf d) = (\mathbf a \cdot \mathbf c)(\mathbf b \cdot \mathbf d) - (\mathbf a\cdot \mathbf d)(\mathbf b\cdot \mathbf c) (Identit\'e de Binet-Cauchy) $$
$$ \nabla \times (\nabla \psi) = \mathbf 0 $$
$$ \nabla \cdot (\nabla \times \mathbf V) = 0 $$
$$ \nabla\times(\nabla\times\mathbf V) = \nabla(\nabla\cdot \mathbf V)-\nabla^2\mathbf V $$
$$ \nabla(\psi\phi) = (\nabla\psi)\phi + (\nabla\phi)\psi $$
$$ \nabla \cdot (\psi\mathbf V) = (\nabla\psi)\cdot \mathbf V + (\nabla \cdot \mathbf V)\psi $$
$$ \nabla \times (\psi\mathbf V) = (\nabla\psi)\times \mathbf V + (\nabla\times\mathbf V)\psi $$
$$ \nabla(\mathbf A\cdot \mathbf B) = (\mathbf A \cdot \nabla)\mathbf B+(\mathbf B\cdot \nabla)\mathbf A + \mathbf A\times(\nabla\times \mathbf B) + \mathbf B\times(\nabla \times \mathbf A) $$
$$ \nabla\cdot(\mathbf A \times \mathbf B) = (\nabla\times\mathbf A)\cdot \mathbf B - \mathbf A\cdot(\nabla\times\mathbf B) $$
$$ \nabla \times (\mathbf A\times \mathbf B) = (\nabla\cdot \mathbf B)\mathbf A - (\nabla\cdot \mathbf A)\mathbf B + (\mathbf B\cdot \nabla)\mathbf A - (\mathbf A\cdot\nabla)\mathbf B $$
$$ \mathbf V \times (\nabla \times \mathbf V)=\nabla (\mathbf V^2/2) - (\mathbf V\cdot \mathbf \nabla)\mathbf V $$
$$ (\mathbf A \cdot \nabla) \mathbf B - (\mathbf A \times \nabla) \times \mathbf B = (\nabla \cdot \mathbf B) \mathbf A - \mathbf A \times (\nabla \times \mathbf B) $$

Pour l'onde sinuso\"idale: 
$\vec{E} = \vec{E_0} e^{i(\omega t - \vec{k}\cdot \vec{r})}$
%% e^{i(\omega t - \vec{k}\cdot \vec{r})}

$$
\begin{aligned}
\divergence{E} 
& = \divergence{(\vec{E_0} e^{i(\omega t - \vec{k}\cdot \vec{r})})} \\
& = e^{i(\omega t - \vec{k}\cdot \vec{r})} (\divergence{E_0}) + (\gradien{e^{i(\omega t - \vec{k}\cdot \vec{r})}}) \cdot \vec{E_0}\\
& = (\gradien{e^{i(\omega t - \vec{k}\cdot \vec{r})}}) \cdot \vec{E_0}\\
& = (\gradien{e^{i(\omega t - (k_x x + k_y y + k_z z))}}) \cdot \vec{E_0}\\
& = (-i\vec{k}{e^{i(\omega t - \vec{k}\cdot \vec{r})}}) \cdot \vec{E_0}\\
& = -i\vec{k}\cdot \vec{E}
\end{aligned}
$$

$$
\gradien{e^{i(\omega t - \vec{k} \cdot \vec{r})}}
= -i\vec{k}{e^{i(\omega t - \vec{k}\cdot \vec{r})}}
$$

$$
\begin{aligned}
\rot{E} 
& = \rot{(\vec{E_0} e^{i(\omega t - \vec{k}\cdot \vec{r})})} \\
& = e^{i(\omega t - \vec{k}\cdot \vec{r})} (\rot{E_0}) + (\gradien{e^{i(\omega t - \vec{k}\cdot \vec{r})}}) \times \vec{E_0}\\
& = (\gradien{e^{i(\omega t - \vec{k}\cdot \vec{r})}}) \times \vec{E_0}\\
& = (-i\vec{k}{e^{i(\omega t - \vec{k}\cdot \vec{r})}}) \times \vec{E_0}\\
& = -i\vec{k}\times \vec{E}
\end{aligned}
$$

$$
\rot{\vec{E}} = - \dfrac{\partial \vec{B}}{\partial t}
\Rightarrow
-i\vec{k} \times \vec{E} = -i \omega \vec{B}
\Rightarrow
\vec{B} = \dfrac{\vec{k} \times \vec{E}}{\omega}
$$

\subsection{Paquet d'ondes}
\href{http://ressources.univ-lemans.fr/AccesLibre/UM/Pedago/physique/02/divers/paquet2.html}{ref}

Le paquet d'ondes est constitu\'e par la superposition d'ondes de longueurs d'onde diff\'erentes centr\'ees sur une valeur particuli\`ere de $\lambda = \lambda_0$. Le "paquet" est la r\'egion de l'espace o\`u ces ondes interf\`erent de mani\`ere constructive.

On peut \'ecrire la phase d'une onde sinuso\"idale sous la forme $\phi(k) = k.x - \omega(k).t$.\\
Si la phase ne change pas trop dans le domaine de variation de $k$ autour de $k_0$, il y a interf\'erence constructive pour les valeurs de $x$ et $t$ qui rendent la phase constante (i.e pour $d\phi(h) / dk = 0$ donc pour $x = t.d\omega(k) / dk$.\\
Cette perturbation se d\'eplace \`a la vitesse $x / t  = V_g = d\omega(k) / dk$ que l'on nomme vitesse de groupe.

La c\'el\'erit\'e c d'une onde unique de vecteur d'onde $k$ est $c = V_p = \omega / k$ (vitesse de phase). \\
Quand la vitesse de groupe n'est pas constante, il y a dispersion.
\end{document}
