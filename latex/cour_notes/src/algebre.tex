\chapter{Alg\`ebre}
\textbf{Espace m\'etrique}
On appelle $(E, d)$ un espace m\'etrique si $ E$  est un ensemble et d une distance sur $E$ .

\textbf{Espace complet}
Un espace m\'etrique $ M$  est dit complet si toute suite de Cauchy de $ M$  a une limite dans $ M$  (c'est-\`a-dire qu'elle converge dans $M$ ).\newline
Intuitivement, un espace est complet s'il n'a pas de trou, s'il n'a aucun point manquant. \newline
Par exemple, les nombres rationnels ne forment pas un espace complet, puisque $\sqrt{2}$ n'y figure pas alors qu'il existe une suite de Cauchy de nombres rationnels ayant cette limite.

Il est toujours possible de remplir les trous amenant ainsi \`a la compl\'etion d'un espace donn\'e.

\textbf{Espace euclidien}
 il est d\'efini par la donn\'ee d'un espace vectoriel sur le corps des r\'eels, de dimension finie, muni d'un produit scalaire, qui permet de mesurer distances et angles.

\textbf{Espace hermitien}
En math\'ematiques, un espace hermitien est un espace vectoriel sur le corps commutatif des complexes de dimension finie et muni d'un produit scalaire.

La g\'eom\'etrie d'un tel espace est analogue \`a celle d'un espace euclidien

Une forme hermitienne est une application d\'efini\'e sur E×E \`a valeur dans $\mathbf{C}$ not\'ee $\langle .,.\rangle$, telle que :
\begin{itemize}
		\item pour tout y fix\'e l'application $x \mapsto \langle x,y\rangle $est $\mathbf{C}$-lin\'eaire et
		\item $\forall x,y \in E$,$\langle x,y\rangle=\overline{ \langle y,x\rangle}$.
\end{itemize}
En particulier, $\langle x,x\rangle$ est r\'eel, et $x\mapsto \langle x,x\rangle$ est une forme quadratique sur E vu comme $\mathbf{R}$-espace vectoriel.

\textbf{Espace pr\'ehilbertien}
En math\'ematiques, un espace pr\'ehilbertien est d\'efini comme un espace vectoriel r\'eel ou complexe muni d'un produit scalaire

Un espace pr\'ehilbertien $(E,\langle\cdot,\cdot\rangle)$ est alors un espace vectoriel E muni d'un produit scalaire $\langle\cdot,\cdot\rangle$.

\textbf{Espace de Hilbert}
C'est un espace pr\'ehilbertien complet, c'est-\`a-dire un espace de Banach dont la norme $\parallel\bullet\parallel$d\'ecoule d'un produit scalaire ou hermitien $\langle\cdot,\cdot\rangle$ par la formule
 $$\parallel x\parallel = \sqrt{\langle x,x \rangle}$$
C'est la g\'en\'eralisation en dimension quelconque d'un espace euclidien ou hermitien.
\bigskip

\textbf{Th\'eor\`eme de Riesz (Fr\'echet-Riesz)}\newline
un th\'eor\`eme qui repr\'esente les \'el\'ements du dual d'un espace de Hilbert comme produit scalaire par un vecteur de l'espace.
Soient :
\begin{itemize}
		\item H un espace de Hilbert (r\'eel ou complexe) muni de son produit scalaire not\'e $<.,.>$
		\item f ∈ H' une forme lin\'eaire continue sur H.
\end{itemize}
Alors il existe un unique $y$ dans $H$ tel que pour tout x de H on ait $f(x) = <y, x>$
$$
\exists\,!\ y \in H\,, \quad \forall x\in H\,, \quad f(x) = \langle y,x\rangle
$$
\underline{Extension aux formes bilin\'eaires}\newline
Si a est une forme bilin\'eaire continue sur un espace de Hilbert r\'eel H (ou une forme sesquilin\'eaire complexe continue sur un Hilbert complexe), alors il existe une unique application A de H dans H telle que, pour tout $(u, v) \in H \times H$, on ait $a(u, v) = <Au, v>$. De plus, A est lin\'eaire et continue, de norme \'egale \`a celle de a.
$$
\exists !\,A\in\mathcal{L}(H),\ \forall (u,v)\in H\times H,\ a(u,v)=\langle Au,v \rangle.
$$
Cela r\'esulte imm\'ediatement de l'isomorphisme canonique (isom\'etrique) entre l'espace norm\'e des formes bilin\'eaires continues sur H × H et celui des applications lin\'eaires continues de H dans son dual, et de l'isomorphisme ci-dessus entre ce dual et H lui-m\^eme.
\bigskip

\textbf{Th\'eor\`eme de Lax-Milgram}\newline
Appliqu\'e \`a certains probl\`emes aux d\'eriv\'ees partielles exprim\'es sous une formulation faible (appel\'ee \'egalement formulation variationnelle). Il est notamment l'un des fondements de la m\'ethode des \'el\'ements finis.
Soient :
\begin{itemize}
		\item $\mathcal{H}$ un espace de Hilbert r\'eel ou complexe muni de son produit scalaire not\'e $\langle.,.\rangle$, de norme associ\'ee not\'ee $\|.\|$
		\item a(.\, ,\,.) une forme bilin\'eaire (ou une forme sesquilin\'eaire si $\mathcal{H}$ est complexe) qui est
				\begin{itemize}
						\item continue sur $\mathcal{H}\times\mathcal{H} : \exists\,c>0, \forall (u,v)\in \mathcal{H}^2\,,\ |a(u,v)|\leq c\|u\|\|v\|$
						\item coercive sur $\mathcal{H}$ (certains auteurs disent plut\^ot $\mathcal{H}$-elliptique) : $\exists\,\alpha>0, \forall u\in\mathcal{H}\,,\ a(u,u) \geq \alpha\|u\|^2$
				\end{itemize}
		\item $L(.)$ une forme lin\'eaire continue sur $\mathcal{H}$
\end{itemize}
Sous ces hypoth\`eses il existe un unique $u$ de $\mathcal{H}$ tel que l'\'equation $a(u,v)=L(v)$ soit v\'erifi\'ee pour tout $v$ de $\mathcal{H}$ :
$$
\quad \exists!\ u \in \mathcal{H},\ \forall v\in\mathcal{H},\quad a(u,v)=L(v)
$$
Si de plus la forme bilin\'eaire a est sym\'etrique, alors  $u$  est l'unique \'el\'ement de $\mathcal{H}$ qui minimise la fonctionnelle $J:\mathcal{H}\rightarrow\R$ d\'efinie par $J(v) = \tfrac{1}{2}a(v,v)-L(v)$ pour tout $v$ de $\mathcal{H}$, c'est-\`a-dire :
$$
\quad \exists!\ u \in \mathcal{H},\quad J(u) = \min_{v\in\mathcal{H}}\ J(v)
$$
\bigskip

$-\laplace $ admet une base de fonctions propres $v_k$, $k \in N$,
orthonormales pour le produit scalaire de $L^2(\Omega)$
