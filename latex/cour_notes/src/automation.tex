% !Mode:: "TeX:UTF-8"
\chapter{Automation}
\section{自动控制的基本概念}
在自动控制技术中, 把工作的机器称为被控对象
把表征这些机器设备工作状态的物理参量称为被控量
而对这些物理参量的要求值称为给定值或希望值(或参考输入)
则控制的任务可概括为: 使被控对象的被控量等于给定值

多变量系统是现代控制理论研究的主要对象, 在数学上采用状态空间法为基础, 讨论多变量, 变参数, 非线性, 高精度, 高效能等控制系统的分析与设计.

放大元件: 放大倍数越大, 系统的反应越敏感, 一般情况下, 只要系统稳定, 放大倍数应适当大些.

\textbf{稳态响应的含义}: 通常习惯上把不随时间变化的静态称为稳态. 然而, 在控制系统中, 往往响应已达稳态, 但他们可能随时间有规律地变化. 因此, 控制系统中的稳态响应, 简单说来就是指时间趋于无穷大的确定的响应.

对自动控制系统性能的要求在时域中一般可归纳为\textbf{三大性能指标}:
\begin{description}
\item[稳定性]
\item[瞬态质量] 要求系统瞬态响应过程具有一定的快速性与变化的平稳性
\item[稳态误差]
\end{description}
同一个系统, 上述三大性能指标往往相互制约. 提高控制过程的快速性, 可能会引起系统强烈震荡; 改善了平稳性, 控制过程又可能很迟缓, 甚至使得最终精度也很差.

\section{控制系统的数学模型}
零点用$\bigcirc$表示, 极点用$\times$表示 \newline
极点(虚轴左边)离虚轴越远, 相应的模态收敛越快 \newline
当零点不靠近任何极点时, 诸极点相对来说, 距离零点远一些的极点其模态所占比重较大, 若零点靠近某极点, 则对应模态的比重就减小, 所以离零点很近的极点比重会被大大削弱(\textbf{偶极子}). 当零极点相重, 产生零极点对消时, 相应的模态就消失了.

\textbf{串联}: 传递函数相乘 \newline
\textbf{并联}: 传递函数相加 \newline
结构图变换: 比较点前移, 引出点后移


闭环系统的常用传递函数\newline
开环传递函数: 将回环从反馈的末端$B(s)$处断开, 沿回环从误差信号$E(s)$开始至末端反馈信号$B(s)$终止, 期间经过的的通道传递函数的乘积称为开环传递函数\newline
\textbf{前向传递函数}:有输入到输出的直接通过传递函数的乘积
\begin{equation}
		\text{闭环传递函数} = \frac{\text{前向传递函数}}{1+\text{开环传递函数}}
\end{equation}

无论外部输入信号取何种形式和作用与系统任一输入端(如控制输入端或扰动输入端), 也不论输出信号选择哪个变量(如$C(s)$或$E(s)$), 所对应的闭环传递函数都具有相同的特征方程. \newline
这就是说, 系统的闭环极点与外部输入信号的形式和作用点无关, 同时也与输出信号的选取无关, 仅取决于闭环特征方程的根.\newline
由此可进一步说明, 系统响应的自由运动模态是系统的固有属性, 与外部激发信号无关.


Mason梅逊增益公式
\begin{equation}
		G = \frac{\sum_{k=1}^n P_k * \Delta_k}{\Delta}
\end{equation}
G:总增益\newline
$P_k$: 前向传递函数\newline
$\Delta$: 信号流图的特征式

\section{时域分析法}
\subsection{典型输入信号}
\begin{itemize}
	\item 阶跃信号
	\item 斜坡信号
	\item 抛物线信号
	\item 脉冲信号
	\item 正弦信号
\end{itemize}

\subsection{一阶系统时域分析}
一阶系统的单位阶阶跃响应
\subsection{二阶系统}
典型二阶系统

阻尼(欠阻尼, 临界阻尼, 过阻尼, 负阻尼)
\newline 重点是欠阻尼的分析, 包括调节时间, 超调量

\subsection{稳定性分析}
劳斯稳定判据

误差函数(0,I,II型系统)

\section{根轨迹法}
如开环传递函数
$$
G(s) = \frac{ K}{s(0.5s+1)} = \frac{K_g}{s(s+2)}
$$
\begin{itemize}
	\item n阶系统有n条分支
	\item 根轨迹上每只分支起始于开环极点处, 结束于有限零点处或者无线点处
	\item 开环传递函数分子多项式阶数m小于分母多项式阶数n, 因此有$(n-m)$条根轨迹的终点在无穷远处
\end{itemize}

增加开环零极点对根轨迹的影响\newline
\textbf{增加开环零点}: 在复平面内的共轭复根的根轨迹向左弯曲, 而且分离点左移, 故输出响应的动态过程衰减较快, 超调量减小, 系统的相对稳定性较好\newline
\textbf{增加开环极点}: 在复平面内的共轭复根的根轨迹向右弯曲, 而且分离点右移,这时, 其快速性已大为下降, 相对稳定性也变差

\section{线性系统的频率响应法}
$G(j\omega)$ 就是系统的频率特性, 就是相当于把传递函数$G(s)$中的$s$换成$j\omega$

\textbf{极坐标图}: 当$\omega$ 从 $0 \to \infty$, 频率特性$G(j\omega)$的矢端轨迹

\textbf{伯德图}(对数幅频与对数相频两条曲线)\newline
横轴按频率的对数$\lg \omega$标尺刻度, 但标出的是频率$\omega$ 本身的数值, 因此横轴的刻度是不均匀的

典型环节
\begin{itemize}
  \item 比例环节
  \item 积分环节
  \item 微分环节
  \item 惯性环节 (交接频率)
  \item 一阶微分环节
  \item 震荡环节
  \item 二阶微分环节
  \item 延迟环节
\end{itemize}

频段
\begin{description}
	\item[低频段] 指所有交接频率之前的区段, 这一段特性完全由积分环节和开环增益决定
	\item[中频段] 通常是指在幅值穿越频率$\omega_c$(截止频率)附近的区段, 这一段特性集中反映了闭环系统动态响应的平稳性和快速性
	\item[高频段] 指幅频曲线在中频段以后($\omega > 10\omega_c$)的区段, 这部分特性是由系统中时间常数很小, 频带很宽的部件决定的. 由于远离$\omega_c$, 一般分贝值又较低, 故对系统的动态影响不大. 另外, 从系统抗干扰的角度看, 高频段特性是有其意义的. 高频段的幅值, 直接反映了系统对输入端高频信号的抑制能力. 这部分特性分贝值越低, 系统抗干扰能力越强.
\end{description}

最小相位系统: 开环零点和开环极点全部位于s 左半平面的系统
\subsection{Nyquist稳定判据}
当$G(s)H(s)$ 包含积分环节时, 在对数相频曲线$\omega$为$0_+$的地方, 应该补画一条从相角$\angle G(j0_+)H(j0_+)+N*90^\circ$ 到$\angle G(j0_+)H(j0_+)$
的虚线, 这里N 是积分环节数.\newline
计算正负穿越时, 应将补画的虚线看成对数相频特性曲线的一部分.

\subsection{开环频域指标}
稳定裕度: 赋值裕度和相角裕度

