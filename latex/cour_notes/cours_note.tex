% !Mode:: "TeX:UTF-8"
\documentclass[openany]{book}
% !Mode:: "TeX:UTF-8"
\usepackage[english]{babel}
\usepackage[UTF8]{ctex}
\usepackage{amsmath, amsthm, amssymb}

% Figure
\usepackage{graphicx}
\usepackage{float} %% H can fix the location
\usepackage{caption}
\usepackage[format=hang,singlelinecheck=0,font={sf,small},labelfont=bf]{subfig}
\usepackage[noabbrev]{cleveref}
\captionsetup[subfigure]{subrefformat=simple,labelformat=simple,listofformat=subsimple}
\renewcommand\thesubfigure{(\alph{subfigure})}

\usepackage{epstopdf} %% convert eps to pdf
\DeclareGraphicsExtensions{.eps,.mps,.pdf,.jpg,.png} %% bmp, gif not supported
\DeclareGraphicsRule{*}{eps}{*}{}
\graphicspath{{img/}{figure/}{../figure/}} %% fig directorys

%% \usepackage{pstricks} %% a set of macros that allow the inclusion of PostScript drawings directly inside TeX or LaTeX code
%% \usepackage{wrapfig} %% Wrapping text around figures

% Table
\usepackage{booktabs} %% allow the use of \toprule, \midrule, and \bottomrule
\usepackage{tabularx}
\usepackage{multirow}
\usepackage{colortbl}
\usepackage{longtable}
\usepackage{supertabular}

\usepackage[colorinlistoftodos]{todonotes}

% Geometry
\usepackage[paper=a4paper, top=1.5cm, bottom=1.5cm, left=1cm, right=1cm]{geometry}
%% \usepackage[paper=a4paper, top=2.54cm, bottom=2.54cm, left=3.18cm, right=3.18cm]{geometry} %% ms word
%% \usepackage[top=0.1cm, bottom=0.1cm, left=0.1cm, right=0.1cm, paperwidth=9cm, paperheight=11.7cm]{geometry} %% kindle

% Code
%% \usepackage{alltt} %% \textbf can be used in alltt, but not in verbatim

\usepackage{listings}
\lstset{
    backgroundcolor=\color{white},
    columns=flexible,
    breakatwhitespace=false,
    breaklines=true,
    captionpos=tt,
    frame=single, %% Frame: show a box around, possible values are: none|leftline|topline|bottomline|lines|single|shadowbox
    numbers=left, %% possible values are: left, right, none
    numbersep=5pt,
    showspaces=false,
    showstringspaces=false,
    showtabs=false,
    stepnumber=1, %% interval of lines to display the line number
    rulecolor=\color{black},
    tabsize=2,
    texcl=true,
    title=\lstname,
    escapeinside={\%*}{*)},
    extendedchars=false,
    mathescape=true,
    xleftmargin=3em,
    xrightmargin=3em,
    numberstyle=\color{gray},
    keywordstyle=\color{blue},
    commentstyle=\color{green},
    stringstyle=\color{red},
}

% Reference
%% \bibliographystyle{plain} % reference style

% Color
\usepackage[colorlinks, linkcolor=blue, anchorcolor=red, citecolor=green, CJKbookmarks=true]{hyperref}
\usepackage{color}
\def\red#1{\textcolor[rgb]{1.00,0.00,0.00}{#1}}
\newcommand\warning[1]{\red{#1}}

% Other
%% \usepackage{fixltx2e} %% for use of \textsubscript
%% \usepackage{dirtree}  %% directory structure, like the result of command tree in bash shell

   %导入需要用到的package
% !Mode:: "TeX:UTF-8"

% Chapter
%% \makeatletter\@addtoreset{chapter}{part}\makeatother %% have chapters numbered without interruption (numbering through parts)

% Equation
\makeatletter\@addtoreset{equation}{section}\makeatother 
\renewcommand\theequation{%
\thepart\arabic{chapter}%
-\thepart\arabic{section}%
-\thepart\arabic{equation}%
}

% Theorem
\newtheorem{definition}{D\'efintion}
\newtheorem*{thmwn}{Thm}
\newtheorem{theorem}{Th\'eor\`eme}[chapter]
\newtheorem{corollary}{Corollary}[theorem]
\newtheorem{lemma}{Lemma}
\newtheorem{proposition}{Proposition}[chapter]
\newtheorem*{attention}{Attention}
\newtheorem*{note}{Note}
\newtheorem*{remark}{Remark}
\newtheorem{example}{Example}
\newtheorem{question}{Question}[chapter]
\newtheorem{problem}{Problem}
\newtheorem*{answer}{Answer}
\newtheorem{fact}{Fact}

   %导入需要用到的package


\begin{document}
\title{MES NOTES DE COURS}
\author{Eric}
\maketitle
\tableofcontents

\frontmatter
\chapter*{Preface}
This book the notebook edited by Eric for his courses \`a L'Ecole Centrale de P\'ekin. \par
In this book, I have used French, English and Chinese three languages. \par
If you find any error, please feel free to contact me by E-mail to \href{mailto:wangchaogo1990@gmail.com}{Eric}. \par
When you send mail, please specify the error position.\par
Your feedback is a great contribution for the success of this book!\par
Thank you!
\par
(The content in this book reference those textbooks$^\copyright$ and wikipedia$^\copyright$)
\newpage
\begin{center}
\  %一个空格, 如果不加的话, 后面的\vfill 不起作用
\vfill
\large{\`A MOI DANS LE FUTURE!}
\vskip 1cm
\large{TO THOSE WHO WILL REVIEW THE COURSE!}
\vfill
\end{center}
\mainmatter

\part{MATH\'EMATIQUES}
\chapter{Alg\`ebre}
\textbf{Espace m\'etrique}
On appelle $(E, d)$ un espace m\'etrique si $ E$  est un ensemble et d une distance sur $E$ .

\textbf{Espace complet}
Un espace m\'etrique $ M$  est dit complet si toute suite de Cauchy de $ M$  a une limite dans $ M$  (c'est-\`a-dire qu'elle converge dans $M$ ).\newline
Intuitivement, un espace est complet s'il n'a pas de trou, s'il n'a aucun point manquant. \newline
Par exemple, les nombres rationnels ne forment pas un espace complet, puisque $\sqrt{2}$ n'y figure pas alors qu'il existe une suite de Cauchy de nombres rationnels ayant cette limite.

Il est toujours possible de remplir les trous amenant ainsi \`a la compl\'etion d'un espace donn\'e.

\textbf{Espace euclidien}
 il est d\'efini par la donn\'ee d'un espace vectoriel sur le corps des r\'eels, de dimension finie, muni d'un produit scalaire, qui permet de mesurer distances et angles.

\textbf{Espace hermitien}
En math\'ematiques, un espace hermitien est un espace vectoriel sur le corps commutatif des complexes de dimension finie et muni d'un produit scalaire.

La g\'eom\'etrie d'un tel espace est analogue \`a celle d'un espace euclidien

Une forme hermitienne est une application d\'efini\'e sur E×E \`a valeur dans $\mathbf{C}$ not\'ee $\langle .,.\rangle$, telle que :
\begin{itemize}
		\item pour tout y fix\'e l'application $x \mapsto \langle x,y\rangle $est $\mathbf{C}$-lin\'eaire et
		\item $\forall x,y \in E$,$\langle x,y\rangle=\overline{ \langle y,x\rangle}$.
\end{itemize}
En particulier, $\langle x,x\rangle$ est r\'eel, et $x\mapsto \langle x,x\rangle$ est une forme quadratique sur E vu comme $\mathbf{R}$-espace vectoriel.

\textbf{Espace pr\'ehilbertien}
En math\'ematiques, un espace pr\'ehilbertien est d\'efini comme un espace vectoriel r\'eel ou complexe muni d'un produit scalaire

Un espace pr\'ehilbertien $(E,\langle\cdot,\cdot\rangle)$ est alors un espace vectoriel E muni d'un produit scalaire $\langle\cdot,\cdot\rangle$.

\textbf{Espace de Hilbert}
C'est un espace pr\'ehilbertien complet, c'est-\`a-dire un espace de Banach dont la norme $\parallel\bullet\parallel$d\'ecoule d'un produit scalaire ou hermitien $\langle\cdot,\cdot\rangle$ par la formule
 $$\parallel x\parallel = \sqrt{\langle x,x \rangle}$$
C'est la g\'en\'eralisation en dimension quelconque d'un espace euclidien ou hermitien.
\bigskip

\textbf{Th\'eor\`eme de Riesz (Fr\'echet-Riesz)}\newline
un th\'eor\`eme qui repr\'esente les \'el\'ements du dual d'un espace de Hilbert comme produit scalaire par un vecteur de l'espace.
Soient :
\begin{itemize}
		\item H un espace de Hilbert (r\'eel ou complexe) muni de son produit scalaire not\'e $<.,.>$
		\item f ∈ H' une forme lin\'eaire continue sur H.
\end{itemize}
Alors il existe un unique $y$ dans $H$ tel que pour tout x de H on ait $f(x) = <y, x>$
$$
\exists\,!\ y \in H\,, \quad \forall x\in H\,, \quad f(x) = \langle y,x\rangle
$$
\underline{Extension aux formes bilin\'eaires}\newline
Si a est une forme bilin\'eaire continue sur un espace de Hilbert r\'eel H (ou une forme sesquilin\'eaire complexe continue sur un Hilbert complexe), alors il existe une unique application A de H dans H telle que, pour tout $(u, v) \in H \times H$, on ait $a(u, v) = <Au, v>$. De plus, A est lin\'eaire et continue, de norme \'egale \`a celle de a.
$$
\exists !\,A\in\mathcal{L}(H),\ \forall (u,v)\in H\times H,\ a(u,v)=\langle Au,v \rangle.
$$
Cela r\'esulte imm\'ediatement de l'isomorphisme canonique (isom\'etrique) entre l'espace norm\'e des formes bilin\'eaires continues sur H × H et celui des applications lin\'eaires continues de H dans son dual, et de l'isomorphisme ci-dessus entre ce dual et H lui-m\^eme.
\bigskip

\textbf{Th\'eor\`eme de Lax-Milgram}\newline
Appliqu\'e \`a certains probl\`emes aux d\'eriv\'ees partielles exprim\'es sous une formulation faible (appel\'ee \'egalement formulation variationnelle). Il est notamment l'un des fondements de la m\'ethode des \'el\'ements finis.
Soient :
\begin{itemize}
		\item $\mathcal{H}$ un espace de Hilbert r\'eel ou complexe muni de son produit scalaire not\'e $\langle.,.\rangle$, de norme associ\'ee not\'ee $\|.\|$
		\item a(.\, ,\,.) une forme bilin\'eaire (ou une forme sesquilin\'eaire si $\mathcal{H}$ est complexe) qui est
				\begin{itemize}
						\item continue sur $\mathcal{H}\times\mathcal{H} : \exists\,c>0, \forall (u,v)\in \mathcal{H}^2\,,\ |a(u,v)|\leq c\|u\|\|v\|$
						\item coercive sur $\mathcal{H}$ (certains auteurs disent plut\^ot $\mathcal{H}$-elliptique) : $\exists\,\alpha>0, \forall u\in\mathcal{H}\,,\ a(u,u) \geq \alpha\|u\|^2$
				\end{itemize}
		\item $L(.)$ une forme lin\'eaire continue sur $\mathcal{H}$
\end{itemize}
Sous ces hypoth\`eses il existe un unique $u$ de $\mathcal{H}$ tel que l'\'equation $a(u,v)=L(v)$ soit v\'erifi\'ee pour tout $v$ de $\mathcal{H}$ :
$$
\quad \exists!\ u \in \mathcal{H},\ \forall v\in\mathcal{H},\quad a(u,v)=L(v)
$$
Si de plus la forme bilin\'eaire a est sym\'etrique, alors  $u$  est l'unique \'el\'ement de $\mathcal{H}$ qui minimise la fonctionnelle $J:\mathcal{H}\rightarrow\R$ d\'efinie par $J(v) = \tfrac{1}{2}a(v,v)-L(v)$ pour tout $v$ de $\mathcal{H}$, c'est-\`a-dire :
$$
\quad \exists!\ u \in \mathcal{H},\quad J(u) = \min_{v\in\mathcal{H}}\ J(v)
$$
\bigskip

$-\laplace $ admet une base de fonctions propres $v_k$, $k \in N$,
orthonormales pour le produit scalaire de $L^2(\Omega)$
%=========================================================================%
%=========================================================================%
%=========================================================================%
\chapter{Analyse}

\textbf{COERVIT\'E}

Une fonction f d\'efinie sur un espace norm\'e X \`a valeurs dans $ \bar{\R}:=\R\cup\{-\infty,+\infty\}$ est dite coercive sur une partie non born\'ee $ P$ de $X$ si
$$
\lim_{\| x\|\to+\infty\atop x\in P}f(x)=+\infty
$$

ou de mani\`ere plus pr\'ecise
$$
\forall\,\nu\in\R,\quad \exists\,\rho\geqslant0:\quad (x\in X ~\mbox{et}~ \|x\|\geqslant\rho) \quad\Longrightarrow\quad f(x)\geqslant\nu.
$$

Il revient au m\^eme de dire que les intersections avec P des ensembles de sous-niveau de la fonction sont born\'ees :
$$
\forall\,\nu\in\R,\qquad\{x\in P: f(x)\leqslant\nu\}~\mbox{est born\'e.}
$$
Si l'on ne sp\'ecifie pas la partie P, il est sous-entendu que P=X.

\underline{Cas d'une forme bilin\'eaire} \newline
Plus sp\'ecifiquement, une forme bilin\'eaire $a:X\times X\to\R$ est dite coercive si elle v\'erifie :
$$
\exists\,\alpha>0,\quad\forall\,x\in X:\qquad a(x,x) \geqslant \alpha\|x\|^2.
$$
Certains auteurs pr\'ef\`erent utiliser l'appellation X-elliptique pour cette derni\`ere d\'efinition. Celle-ci intervient entre autres dans le th\'eor\`eme de Lax-Milgram et la th\'eorie des op\'erateurs elliptiques, accessoirement dans la m\'ethode des \'el\'ements finis.

\underline{Lien entre les d\'efinitions}\newline
Dans le cas où a est une forme bilin\'eaire, en posant $f(u)=a(u,u)$ on a \'equivalence entre la coercivit\'e de a et celle de $f$. En effet, $\scriptstyle\lim_{\| x\|\to\infty}f(x)=+\infty $implique qu'il existe $R>0 $tel que $\scriptstyle\|x\|\geqslant R\Rightarrow f(x)\geqslant 1$. Ainsi:
$$
\left(\frac{R}{\|u\|}\right)^2a(u,u)=a\left(\frac{R}{\|u\|}u,\frac{R}{\|u\|}u\right)=f\left(\frac{R}{\|u\|}u\right)\geqslant 1
$$
et
$$
a(u,u)\geqslant\left(\frac{\|u\|}{R}\right)^2.
$$
维基百科上没有直接给出coercivit\'e的定义, 但是按照维基百科的解释, 我觉得coercivit\'e的定义是
$$
 \alpha=\left(\frac{1}{R}\right)^2
$$

\bigskip
\textbf{Application contractante}\newline
Une application contractante, ou contraction, est une application k-lipschitzienne avec $0 \geq k \le 1$
\bigskip

\textbf{Th\'eor\`eme du point fixe pour une application contractante}\newline
Soient $ E$  un espace m\'etrique complet (non vide) et $ f$  une application $ k$ -contractante de $ E$  dans $ E$ . Il existe un point fixe unique $x^*$ de f (c'est-\`a-dire un $x^*$ dans $ E$  tel que $ f(x^* ) = x^*)$ . De plus, toute suite d'\'el\'ements de E v\'erifiant la r\'ecurrence
$$x_{n+1}=f(x_n)$$
v\'erifie la majoration
$$d(x_n,x^*) \le \frac {k^n}{1-k} d(x_0,x_1)$$
donc converge vers $x^*$
\bigskip

\textbf{In\'egalit\'e de Poincar\'e}
un r\'esultat de la th\'eorie des espaces de Sobolev\\
Cette in\'egalit\'e permet de borner une fonction \`a partir d'une estimation sur ses d\'eriv\'ees et de la g\'eom\'etrie de son domaine de d\'efinition\\
L'in\'egalit\'e de Poincar\'e classique
Soit $p$, tel que $ 1 \leq p < \infinity$  et $\Omega$ un ouvert de largeur finie (born\'e dans une direction). Alors il existe une constante  $  C$  , d\'ependant uniquement de $\Omega$ et $p$, telle que, pour toute fonction $u$   de l'espace de Sobolev $W_0^{1,p(\Omega)}$
$$
\| u \|_{L^{p} (\Omega)} \leq C \| \nabla u \|_{L^{p} (\Omega)}
$$

\subsection{Formule}
\textbf{拉普拉斯算子: 梯度的散度}
$$\laplace f = \mbox{div} (\vec{\mbox{grad}}f)$$

Cart\'esien: $$\laplace f = \sum_{i=1}^n \frac{\partial^2 f }{\partial x_i^2} $$

函数的拉普拉斯算子也是该函数的黑塞矩阵的迹 $\laplace f = tr(H(f))$

极坐标下的拉普拉斯算子表示法
$$\laplace f = \frac{ 1}{\rho}\frac{\partial  }{\partial \rho}(\rho \frac{\partial  f}{\partial \rho}) + \frac{ 1}{\rho^2}\frac{\partial ^2 f}{\partial\theta^2} + \frac{\partial^2 f}{\partial z^2}$$

\bigskip
\textbf{Equation de poisson}
$$\laplace \varphi = f$$
si $f=0$, 那么泊松方程就会变成一个齐次方程, 称为拉普拉斯方程

\bigskip
\textbf{高斯}

高斯公式用散度表示为:
$$
\iiint_{\Omega}\mathrm{div}\mathbf{A}dv=
\int\!\!\!\!\int_{\Sigma}\!\!\!\!\!\!\!\!\!\!\!\!\!\!\;\;\;\bigcirc\,\,A_{n}dS
=
\int\!\!\!\!\int_{\Sigma}\!\!\!\!\!\!\!\!\!\!\!\!\!\!\;\;\;\bigcirc\,\,\mathbf{A}\cdot d\mathbf{S}
$$
其中Σ是空间闭区域Ω的边界曲面,而
$$
A_n=\mathbf{A}\cdot\mathbf{n}=P\cos\alpha+Q\cos\beta+R\cos\gamma
$$
$$
d\mathbf{S}=\mathbf{n}\cdot S
$$
$n$是向量$A$在曲面$\Sigma$的外侧法向量上的投影。

\bigskip
%\vskip 0.5cm
\textbf{斯托克斯公式}

$\mathbf{R}^3$上的斯托克斯公式
设$S$是分片光滑的有向曲面,$S$的边界为有向闭曲线$Γ$,即$\Gamma=\partial S$,且$Γ$的正向与$S$的侧符合右手规则: 函数$P(x,y,z),Q(x,y,z),R(x,y,z)$都是定义在"曲面$S$连同其边界$Γ$"上且都具有一阶连续偏导数的函数,则有
$$\iint\limits_{S}(\frac{\partial R}{\partial y}-\frac{\partial Q}{\partial z})dydz+(\frac{\partial P}{\partial z}-\frac{\partial R}{\partial x})dzdx+(\frac{\partial Q}{\partial x}-\frac{\partial P}{\partial y})dxdy
=\oint\limits_{\Gamma}Pdx+Qdy+Rdz$$

这个公式叫做$\mathbf{R}^3$上的斯托克斯公式或开尔文-斯托克斯定理、旋度定理。这和函数的旋度有关,用梯度算符可写成
$$
 \int_{S} \nabla \times \mathbf{F} \cdot d\mathbf{S} = \oint_{\partial S} \mathbf{F} \cdot d \mathbf{r}
$$

通过以下公式可以在"对坐标的"曲线积分""和对面积的"面积积分""之间相互转换:
$$
\iint\limits_{\Sigma}\begin{vmatrix} \cos \alpha & \cos \beta & \cos \gamma \\ \frac{\partial}{\partial x} & \frac{\partial}{\partial y} & \frac{\partial}{\partial z} \\ P & Q & R \end{vmatrix}dS=\oint\limits_{\Gamma}Pdx+Qdy+Rdz
$$

\bigskip
\textbf{格林公式}
设闭区域$D$由分段光滑的曲线$\partial D$($\partial D$是$D$取正向的边界曲线)围成,函数$P(x,y)$及$Q(x,y)$在$D$上具有一阶连续偏导数,则有
$$\oint_{\partial D} (Pdx+Qdy) = \iint_D (\frac{\partial Q}{\partial x} - \frac{\partial  P}{\partial y}dxdy)$$

\bigskip
\textbf{格林第一公式}
设函数$u(x,y,z)$和$v(x,y,z)$在闭区域$\Omega$上具有一阶及二阶连续偏导数,则有
$$\iiint_{\Omega} u\laplace v dxdydz
=
\oint\!\oint_{\Sigma} u \frac{\partial v}{\partial n}dS -
\iiint_{\Omega}(
\frac{\partial  u}{\partial x}\frac{\partial v}{\partial x}+
\frac{\partial  u}{\partial y}\frac{\partial v}{\partial y}+
\frac{\partial  u}{\partial z}\frac{\partial v}{\partial z}+
)dxdydz$$
其中$\Sigma$是闭区域$\Omega$的整个边界曲面,$\frac{\partial v}{\partial n}$为函数$v(x,y,z)$沿$\Sigma$的外法线方向的方向导数

更加简洁的写法:
$$
\int_{\Omega}u \laplace v d\Omega = \int_{\Sigma}u \frac{\partial v}{\partial n}d\Sigma - \int_{\Omega} \vec{\mbox{grad}}u \cdot \vec{\mbox{grad}} v d\Omega
$$
也可以用int\'egration par partie 来理解:
div是二次微分, grad是一次微分, div可以看成是在grad的基础上再来一次微分, 而
$ \frac{\partial  v}{\partial n} = <\vec{\mbox{grad}} v,\vec{n}>$, 所以
$\frac{\partial  v}{\partial n}$ 可以理解成一次微分

\bigskip
\textbf{格林第二公式}
设$u(x,y,z)$、$v(x,y,z)$是两个定义在闭区域$\Omega$上的具有二阶连续偏导数的函数,$\frac{\partial u}{\partial n},\frac{\partial v}{\partial n}$依次表示$u(x,y,z)$、$v(x,y,z)$沿$\Sigma$的外法线方向的方向导数,则有
$$\iiint\limits_{\Omega}(u\Delta v - v\Delta u)dxdydz=\oint\!\oint_{\Sigma}(u \frac{\partial v}{\partial n}-v\frac{\partial u}{\partial n})dS$$

\bigskip
\textbf{Serie de Fourrier}
给定一个周期为T的函数x(t),那么它可以表示为无穷级数:
\begin{equation}
		x(t)=\sum _{k=-\infty}^{+\infty}a_k\cdot e^{ik(\frac{2\pi}{T})t}\eqspace (i\text{为虚数单位})
\label{serie.fourrier.fonction}
\end{equation}
其中,$a_k$可以按下式计算:
$$
a_k=\frac{1}{T}\int_{T}x(t)\cdot e^{-ik(\frac{2\pi}{T})t}dt
$$
注意到$f_k(t)=e^{ik(\frac{2\pi}{T})t}$是周期为T的函数,故k 取不同值时的周期信号具有谐波关系(即它们都具有一个共同周期T).\newline
$k=0$时,\eqref{serie.fourrier.fonction} 式中对应的这一项称为直流分量,也就是x(t)在整个周期的平均值.\newline
$k=\pm 1$时具有基波频率$\omega_0=\frac{2\pi}{T}$,称为一次谐波或基波,类似的有二次谐波,三次谐波等等。

\textbf{三角函数族的正交性}
所谓的两个不同向量正交是指它们的内积为0,这也就意味着这两个向量之间没有任何相关性,例如,在三维欧氏空间中,互相垂直的向量之间是正交的。事实上,正交是垂直在数学上的一种抽象化和一般化。一组n个互相正交的向量必然是线性无关的,所以必然可以张成一个n维空间,也就是说,空间中的任何一个向量可以用它们来线性表出。三角函数族的正交性用公式表示出来就是:
$$\int _{0}^{2\pi}\sin (nx)\cos (mx) \,dx=0;$$
$$\int _{0}^{2\pi}\sin (nx)\sin (mx) \,dx=0;(m\ne n)$$
$$\int _{0}^{2\pi}\cos (nx)\cos (mx) \,dx=0;(m\ne n)$$
$$\int _{0}^{2\pi}\sin (nx)\sin (nx) \,dx=\pi;$$
$$\int _{0}^{2\pi}\cos (nx)\cos (nx) \,dx=\pi;$$
\bigskip

\textbf{Autres}
Une matrice
$$
P=
\left(
             \begin{array}{ccc}
               x_1 & y_1 & 1 \\
               x_2 & y_2 & 1 \\
               x_3 & y_3 & 1 \\
             \end{array}
          \right)
$$
由 $(x_1,y_1)$, $(x_2,y_2)$, $(x_3,y_3)$ 三个点组成的三角形的面积:$surf = \frac{1}{2}\det{P}$

%==========================================================================%
%==========================================================================%
%==========================================================================%
\part{PHYSIQIUE}
\chapter{Simulation}

\section{Calcule diff\'erentiel}
\textbf{Diff\'erentiabilit\'e au sens de G\^ateaux}
\begin{definition}
Une fonction $F(x)$ d'un espace vectoriel $ V$  dans $\R$ est
une fonction diff\'erentiable au sens de Gateaux en $ x\in V$  si pour tout $ h \in V$  la fonction $ g(t) = F(x+th)$  est d\'erivable sur $\R$ en$
t = 0 $ et si il existe une forme lin\'eaire sur $ V$  , not\'ee $ DF(x)$ , telle que
$$
\forall h \in V \eqspace \frac{d}{d t}F(x+th)|_{t = 0} = DF(x).h
$$
\end{definition}

\textbf{Diff\'erentielle au sens de Fr\'echet}
\begin{definition}
		Soit $V$ et $W$ deux espaces vectoriels norm\'es.
Une application $ f(x)$ de $V$  dans $W$  est diff\'erentiable (au sens de Fr\'echet, pr\'ecision souvent omise) en $x \in V$  si il existe
une application lin\'eaire continue de $V$ dans $W$ , not\'ee $Df(x)$ telle que
$$
\lim_{\norm{h}\to 0}  \frac{ f(x+h) - f(x) - Df(x).h}{\norm{h}} = 0
$$
\end{definition}
Pour une fonction num\'erique $f:\R \to \R$
$$
\lim_{h \to 0} \frac{ f(x+h) - f(x) - f'(x).h}{h} = 0 \Rightarrow
Df(x).h = f'(x).h
$$

Normes \'equivalentes \\
si 
$\norm{x}_1$
et 
sont deux normes \'equivalentes, alors il existe $c_1,c_2 \in \R^+$ tells que
$$
c_1 \norm{x}_1 \leq \norm{x}_2 \leq c_2 \norm{x}_1
$$

$$
\frac{d}{dt}F(x(t))|_{t = 0} 
= \sum_i \frac{\partial F(x(0))}{\partial x_i}
= \ps{\grad{F(x(0))}}{x'(0)}
$$
une droite passant par le point $x$ et de direction $ h \in \R^n$ , donc d'\'equation param\'etrique $ x(t) = x + th$ 

$$
\frac{d}{dt}F(x+th)|_{t = 0} 
= \ps{\grad{F(x)}}{h}
= DF(x).h
$$

\subsection{Formulaire}
$$ D(g\circ f)(x) = Dg(f(x)) \circ Df(x) $$
$$ D(f(x),g(x)).h = (Df(x).h, g(x)) + (f(x), Dg(x).h)$$
L'ensemble des fonctions diff\'erentiables (au sens de Gateaux ou de Fr\'echet) est stable par addition, multiplication, composition

\subsection{Jacobienne et Hessien}
$$
F :
\begin{pmatrix}
	x_1\\\vdots\\x_n
\end{pmatrix} 
\longmapsto 
\begin{pmatrix}
f_1(x_1,\dots,x_n)\\
\vdots\\
f_m(x_1,\dots,x_n)
\end{pmatrix}
$$

matrice jacobienne de F:
$$
J_F\left(M\right)=
\begin{pmatrix} 
\dfrac{\partial f_1}{\partial x_1} & \cdots & \dfrac{\partial f_1}{\partial x_n} \\
\vdots & \ddots & \vdots \\
\dfrac{\partial f_m}{\partial x_1} & \cdots & \dfrac{\partial f_m}{\partial x_n}
\end{pmatrix}
$$

$$
JF(x)_{i,j} = \frac{\partial  f_i(x_1, \cdots, x_n)}{\partial x_j}
$$

$$
HF(x) = Jf(x) \eqspace f(x) = F'(x)
$$
$$
HF(x)_{i,j} = \frac{\partial ^2 F(x)}{\partial x_i \partial  x_j}
$$
$$
\frac{d^2}{d t^2}F(x+th)|_{t=0} = \ps{HF(x).h}{h}
$$

\begin{proposition}
Une fonction F(x) deux fois diff\'erentiable, admet un d\'eveloppement limit\'e au second ordre
$$ \forall h \in V, F(x+h) = F(x) + \ps{\grad{F(x)}}{h} + \frac{1}{2}\ps{HF(x)h}{h} + \epsilon(h)\norm{h}^2 $$
\end{proposition}

\subsection{Examples}
\begin{example}
Soit $F(x) = \dfrac{1}{2}\ps{Ax}{x} - \ps{b}{x}$
$$DF(x).h = \ps{Ax-b}{h}$$
et pour le produit scalaire canonique
$$
F'(x) = \grad{F(x)} = Ax - b
$$
$$JF'(x) = HF(x) = A$$
\end{example}

\begin{example}
$L$ une forme lin\'eaire continue sur $V$  et $a$  une forme bilin\'eaire continue sym\'etrique et d\'efinie positive sur $V$ \\
Soit $J(v) = \dfrac{1}{2}a(u,v) - L(v)$
$$
DJ(u).v = a(u,v) - L(v)
$$
\end{example}

\begin{example}
$$J(u) = \int_0^1 g(x,u(x),u'(x))dx$$
Calculons $DJ(u)$
$$
DJ(u).v 
= \frac{d}{dt}J(u+tv)|_{t=0}
=\frac{d}{dt}(\int_0^1 g(x,u+tv,u'+tv')dx)|_{t=0}
$$
$$
DJ(u).v 
=\int_0^1 \frac{\partial g}{\partial u_1}(x,u,u')v + \frac{\partial g}{\partial u_2}(x,u,u')v' dx
$$
这里的$u_1$ 对应$u$,$u_2$对应$u'$
$$
DJ(u).v 
=\int_0^1 \frac{\partial g}{\partial u}(x,u,u')v + \frac{\partial g}{\partial u'}(x,u,u')v' dx
$$
对$v'$ 进行分部积分, 同时考虑到$v(0)=v(1)=0$, 得到
$$
DJ(u).v 
=\int_0^1 (\frac{\partial g}{\partial u}(x,u,u') - \frac{d}{dx}\frac{\partial g}{\partial u'}(x,u,u'))v(x) dx
$$
En introduisant le produit scalaire de $L^2([0,1])$
$$
DJ(u).v 
= \ps{-\frac{d}{dx}\frac{\partial g}{\partial u'}(x,u,u')+\frac{\partial g}{\partial u}(x,u,u')}{v}
$$

Soit $u1 \in V_0$ un extremum local de la fonction $J(u)$, il v\'erifie $DJ(u1) = 0$, Donc
$$
\forall v \in V_0, DJ(u).v = \int_0^1 (\frac{\partial  g}{\partial u}(x,u,u') - \frac{d}{dx}\frac{\partial g}{\partial u'}(x,u,u'))v(x) dx = 0
$$
然后得到
$$
\frac{\partial g}{\partial u} - \frac{d}{dx}\frac{\partial g}{\partial u'} = 0
$$
这就是变分原理中的欧拉公式
\end{example}

\section{Convecivit\'e}
\begin{definition}
		On suppose que $V$ est un espace de Hilbert. Une application $f(x)$ de $V$ dans $V$ est monotone si
		$$
		\forall x,y \in V, \ps{f(y)-f(x)}{y-x} \geq 0
		$$
\end{definition}
Note que si $V=\R$, cela revient \`a dire que $f(x)$ est croissante.

\begin{theorem}
		Soit $F(x)$ une fonction diff\'erentiable sur un espace de Hilbert $H$. $F(x)$ est convexe \ssi l'application de $H$ dans $H$ d\'efinie par $f(x) = \grad{F(x)}$ est monotone.
\end{theorem}

\begin{theorem}
		Sur $\R^n$ une fonction deux fois didd\'erentiable est convexe \ssi le Hessien $HF(x)$ est une matrice semi-d\'efinie positive.\\
		R\'eciproquement si $\forall x,h \in V, \ps{HF(x)h}{h}>0$, alros $F$ est strictement convexe.
\end{theorem}

\section{Autre}
Norme dans $H^1$: $\norm{u} = \int_0^1 u'^2 + u^2 dx$\newline
其中$H^1$ 是指espace de Herbert 上的一阶可微(还是可导还不太肯定)的函数构成的空间


L'EDP(sigle pour equations aux d\'eriv\'ees partielles) du second ordre lin\'eaire \`a cofficients constans, en dimension deux:
\begin{equation}
		a \frac{\partial^2 u}{\partial x^2} +
		2b \frac{\partial^2 u}{\partial x \partial y} +
		c \frac{\partial^2 u}{\partial y^2} +
		d \frac{\partial u}{\partial x} +
		e \frac{\partial u}{\partial y} +
		fu
		=0
\end{equation}
L'equation caract\'eristique de cette EDP:
\begin{equation}
		ax^2 + 2bxy + cy^2 + dx + ey + f = 0 \eqspace \text{equation conique}
\end{equation}
方程中的$x$,$y$ 不一定指空间, 也可以是其他未知量, 比如时间$t$

\begin{itemize}
		\item Elliptique si $ac - b^2 > 0$ (la solution minimise une fonctionnelle d'\'energie)
		\item Parabolique si $ac - b^2 = 0$ (\'evolution dissipatif)
		\item Hyperbolique si $ac - b^2 > 0$ (ph\'enom\`enes physiques conservatifs)
\end{itemize}

\textbf{Formulation faible(variationnelle)}

\'etant donn\'e un op\'erateur diff\'erentiel $ \displaystyle R(.) $ et une fonction $ \displaystyle f $ d\'efinie sur un domaine ouvert $ \Omega$ , la formulation forte du probl\`eme est la suivante :

Trouver $\displaystyle u$ d\'efinie sur $ \Omega$  v\'erifiant $ \displaystyle R(u)=f $ en tout point de $ \Omega$ .

Une solution $ \displaystyle u$  est naturellement solution du probl\`eme suivant (formulation faible) :

Trouver $ \displaystyle u$  d\'efinie sur$  \Omega $ v\'erifiant $ \int_\Omega R(u)\ v = \int_\Omega f\ v $ pour toute fonction $ \displaystyle v$  d\'efinie sur $ \Omega$ .

la formulation variationnelle d'un probl\`eme r\'egi par des \'equations aux d\'eriv\'ees partielles correspond \`a une formulation faible de ces \'equations qui s'exprime en termes d'alg\`ebre lin\'eaire dans le cadre d'un \emph{espace de Hilbert}. A l'aide du \emph{th\'eor\`eme de Lax-Milgram}, elle permet de discuter de l'existence et de l'unicit\'e de solutions. La m\'ethode des \'el\'ements finis se fonde sur une formulation variationnelle pour d\'eterminer des solutions num\'eriques approch\'ees du probl\`eme d'origine.

\underline{\'equation de Poisson}\newline
Pour un ouvert $ \Omega$  de $ \R^n$ , consid\'erons l'espace $ L^2(\Omega) $ des fonctions de carr\'e int\'egrable et l'espace de Sobolev $ H^k(\Omega)$  des fonctions dont les d\'eriv\'ees partielles jusqu'\`a l'ordre k sont dans $ L^2(\Omega)$ .

\'etant donn\'e une fonction $ f \in L^2(\Omega)$ , on cherche une solution du probl\`eme suivant (formulation forte):
\begin{equation}
    \begin{cases} u \in H^2(\Omega) \\ -\Delta u = f \text{ dans } \Omega \\ u = 0 \text{ sur } \partial\Omega \end{cases}
\end{equation}

La formulation variationnelle correspondante est la suivante :
\begin{equation}
    \begin{cases} u \in H^1(\Omega) \\ A(u,v) = F(v) \; \forall v \in H^1(\Omega) \, | \, v = 0 \text{ sur } \partial\Omega \\ u = 0 \text{ sur } \partial\Omega \end{cases}
\end{equation}
où
$$
A(u,v) = \int_\Omega \nabla u \cdot \nabla v \eqspace \text{et} \eqspace F(v) = \int_\Omega fv.~
$$
我们可以看到在formulation forte中$u \in H^2(\Omega)$, 而在formulation faible 中$u \in H^1(\Omega)$, 也就是变弱了

Le th\'eor\`eme de Lax-Milgram permet ensuite de conclure \`a l'existence et \`a l'unicit\'e d'une solution de la formulation variationnelle.

\`a noter qu'une solution du premier probl\`eme est toujours solution du second, alors que la r\'eciproque n'est pas vraie (une solution dans $  H^1(\Omega)$   peut ne pas \^etre assez r\'eguli\`ere pour \^etre dans $  H^2(\Omega)) $  : c'est d'ailleurs pour cette raison qu'une solution de la formulation variationnelle est parfois appel\'ee solution faible (ou encore semi-faible).

Avec des conditions de bord plus g\'en\'erales que celles pr\'esent\'ees ici, ce probl\`eme est plus amplement d\'evelopp\'e ici.

\section{Phenom\`enes}
\textbf{Equation d'advection}
平流\\
cherche une fonction $u(x,t)$ du point d'abscisse $ x$ , au temps $ t$ , $ u \in C^1([0;1] \times [0,T])$ solution du probl\`eme
\begin{equation}
\left\{
  \begin{array}{ll}
		  \frac{\partial  u}{\partial t} + a \frac{\partial u}{\partial x } = 0 \\
		  u(x,0) = u_0{x} \\
		  u(0,t) = g(t)
  \end{array}
\right.
\end{equation}
\begin{definition}
		Les droites caract\'eristiques dans le plan $(x,t)$ de l'\'equation \lasteq  sont les droites$
x - at = Cte$
\end{definition}

\begin{proposition}
		Une fonction $ u(x,t) \in C^1$  est solution de \lasteq  si et seulement si $ u$  est une
fonction constante sur les droites caract\'eristiques.
\end{proposition}

Les solutions de l\'equation \lasteq est:
\begin{equation}
u(x,t)=
\left\{
  \begin{array}{ll}
		  u_0(x-at) \si x >= at \\
		  g(t-\frac{x}{a}) \si x \leq at
  \end{array}
\right.
\end{equation}
\bigskip

\textbf{Equation de la diffusion}
\begin{equation}
\left\{
  \begin{array}{ll}
		  u \in C^2(\Omega \times [0,T]) \\
		  C \frac{\partial  u}{\partial t} - k \laplace u + cu =0 \si x \in \Omega \\
		  -k \frac{\partial u}{\partial n} = 0 \si x \in \Gamma \\
		  u(x,0) = u_0{x} \si x \in \Omega\\
  \end{array}
\right.
\end{equation}
\bigskip

\textbf{Equation des ondes}
Soit $u(x,t), x \in [0,L], t \in [0,T] $ solution du probl\`eme
\begin{equation}
\left\{
  \begin{array}{ll}
		  \frac{\partial^2  u}{\partial t^2} = c^2 \frac{\partial^2 u}{\partial x^2 } \eqspace  x \in (0,L), t \in (0,T) \\
		  u(x,0) = u_0{x} \eqspace x \in (0,L) \eqspace u_0(x)\in C([0,L])\  \text{la position initial}\\
		  \frac{\partial  u}{\partial t}(x,0) = 0 \eqspace x \in (0,L)\ \text{la vitess initial est suppos\'ee null}\\
		  u(0,t) = u(L,t) = 0 \eqspace t \in (0,T)
  \end{array}
\right.
\end{equation}
\bigskip

\textbf{Probl\`eme de Dirichlet non-homog\`ene}
边界上定值\\
\begin{equation}
\left\{
  \begin{array}{ll}
		 -\laplace u + cu = f \eqspace \forall x \in \Omega \\
		 u_{\Gamma} = u_d
  \end{array}
\right.
\end{equation}
\bigskip

\textbf{Probl\`eme de Neumann en dimension deux}\\
les conditions aux limites sur $\Gamma$ portent sur la d\'eriv\'ee normale de la solution\\
\begin{equation}
\left\{
  \begin{array}{ll}
		 -\laplace u + cu = f \eqspace \forall x \in \Omega \\
		 \frac{\partial u}{\partial n}_{\Gamma} = g
  \end{array}
\right.
\end{equation}
\bigskip

\section{Discr\'etisation en temps}
\begin{equation}
\left\{
  \begin{array}{ll}
		  \frac{d X}{dt} = A X \\
		  X(0) = X_0
  \end{array}
\right.
\end{equation}
On note $X^n = X(n\tau) \in \R^N$
$$
\frac{X^{n+1} - X^n}{\tau} = (1-\theta)AX^n + \theta A X^{n+1}
$$
\begin{itemize}
		\item Euler explicite $\theta = 0$
		\item Euler implicite $\theta = 1$
		\item Trap\`ezes $\theta = 1/2$
\end{itemize}
\bigskip

\end{document}
