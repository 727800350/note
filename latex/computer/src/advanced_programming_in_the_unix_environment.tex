% !Mode:: "TeX:UTF-8"
\documentclass{article}
% !Mode:: "TeX:UTF-8"
\usepackage[english]{babel}
\usepackage[UTF8]{ctex}
\usepackage{amsmath, amsthm, amssymb}

% Figure
\usepackage{graphicx}
\usepackage{float} %% H can fix the location
\usepackage{caption}
\usepackage[format=hang,singlelinecheck=0,font={sf,small},labelfont=bf]{subfig}
\usepackage[noabbrev]{cleveref}
\captionsetup[subfigure]{subrefformat=simple,labelformat=simple,listofformat=subsimple}
\renewcommand\thesubfigure{(\alph{subfigure})}

\usepackage{epstopdf} %% convert eps to pdf
\DeclareGraphicsExtensions{.eps,.mps,.pdf,.jpg,.png} %% bmp, gif not supported
\DeclareGraphicsRule{*}{eps}{*}{}
\graphicspath{{img/}{figure/}{../figure/}} %% fig directorys

%% \usepackage{pstricks} %% a set of macros that allow the inclusion of PostScript drawings directly inside TeX or LaTeX code
%% \usepackage{wrapfig} %% Wrapping text around figures

% Table
\usepackage{booktabs} %% allow the use of \toprule, \midrule, and \bottomrule
\usepackage{tabularx}
\usepackage{multirow}
\usepackage{colortbl}
\usepackage{longtable}
\usepackage{supertabular}

\usepackage[colorinlistoftodos]{todonotes}

% Geometry
\usepackage[paper=a4paper, top=1.5cm, bottom=1.5cm, left=1cm, right=1cm]{geometry}
%% \usepackage[paper=a4paper, top=2.54cm, bottom=2.54cm, left=3.18cm, right=3.18cm]{geometry} %% ms word
%% \usepackage[top=0.1cm, bottom=0.1cm, left=0.1cm, right=0.1cm, paperwidth=9cm, paperheight=11.7cm]{geometry} %% kindle

% Code
%% \usepackage{alltt} %% \textbf can be used in alltt, but not in verbatim

\usepackage{listings}
\lstset{
    backgroundcolor=\color{white},
    columns=flexible,
    breakatwhitespace=false,
    breaklines=true,
    captionpos=tt,
    frame=single, %% Frame: show a box around, possible values are: none|leftline|topline|bottomline|lines|single|shadowbox
    numbers=left, %% possible values are: left, right, none
    numbersep=5pt,
    showspaces=false,
    showstringspaces=false,
    showtabs=false,
    stepnumber=1, %% interval of lines to display the line number
    rulecolor=\color{black},
    tabsize=2,
    texcl=true,
    title=\lstname,
    escapeinside={\%*}{*)},
    extendedchars=false,
    mathescape=true,
    xleftmargin=3em,
    xrightmargin=3em,
    numberstyle=\color{gray},
    keywordstyle=\color{blue},
    commentstyle=\color{green},
    stringstyle=\color{red},
}

% Reference
%% \bibliographystyle{plain} % reference style

% Color
\usepackage[colorlinks, linkcolor=blue, anchorcolor=red, citecolor=green, CJKbookmarks=true]{hyperref}
\usepackage{color}
\def\red#1{\textcolor[rgb]{1.00,0.00,0.00}{#1}}
\newcommand\warning[1]{\red{#1}}

% Other
%% \usepackage{fixltx2e} %% for use of \textsubscript
%% \usepackage{dirtree}  %% directory structure, like the result of command tree in bash shell

   %导入需要用到的package
% !Mode:: "TeX:UTF-8"
%+++++++++++++++++++++++++++++++++++article+++++++++++++++++++++++++++++++++
%customize the numbering of equation, to make it like section-subsection-equation style, for example,1-2-3
\makeatletter\@addtoreset{equation}{subsection}\makeatother
\renewcommand\theequation{%
\thepart\arabic{section}%
-\thepart\arabic{subsection}%
-\thepart\arabic{equation}%
}
%theorem
\newtheorem{definition}{D\'efintion} %% 整篇文章的全局编号
\newtheorem*{thmwn}{Thm} %% without numbers
\newtheorem{theorem}{Th\'eor\`eme}[section] %% 从属于section编号
\newtheorem{corollary}{Corollary}[theorem] %% 从属于theorem编号
\newtheorem{lemma}{Lemma}
\newtheorem{proposition}{Proposition}[section]
\newtheorem{example}{Example}
\newtheorem*{attention}{Attention}
\newtheorem*{note}{Note}
\newtheorem*{remark}{Remark}
\newtheorem{question}{Question}[section]
\newtheorem{problem}{Problem}
\newtheorem{fact}{Fact}

   %导入需要用到的package
\begin{document}
\tableofcontents
\newpage
\section{IO}
\fun{open, read, write, lseek, close}\\
Unbufferd IO, means that each \fun{read} or \fun{write} invokes a system call in the kernel.\\
These unbufferd I/O functions are not part of ISO C, but are part of POSIX.1 and the single UNIX Specification.

\section{signal}
\textbf{Program Start-Up}\todo{Not understood}\\
When a program is executed, the status of all signals is either default or ignore. Normally, all signals are set to their default action, 
unless the process that calls exec is ignoring the signal. Specifically, the exec functions change the disposition of 
any signals being caught to their default action and leave the status of all other signals alone. 
(Naturally, a signal that is being caught by a process that calls exec cannot be caught by the same function in the new program, 
since the address of the signal- catching function in the caller probably has no meaning in the new program file that is executed.)

One specific example is how an interactive shell treats the interrupt and quit signals for a background process. 
With a shell that doesn't support job control, when we execute a process in the background, as in
\begin{lstlisting}[language=sh]
    cc main.c &
\end{lstlisting}
the shell automatically sets the disposition of the interrupt and quit signals in the background process to be ignored. 
This is so that if we type the interrupt character, it doesn't affect the background process. 
If this weren't done and we typed the interrupt character, it would terminate not only the foreground process, but also all the background processes.

Many interactive programs that catch these two signals have code that looks like
\begin{lstlisting}[language=C]
         void sig_int(int), sig_quit(int);
         if (signal(SIGINT, SIG_IGN) != SIG_IGN)
             signal(SIGINT, sig_int);
         if (signal(SIGQUIT, SIG_IGN) != SIG_IGN)
             signal(SIGQUIT, sig_quit);
\end{lstlisting}
Doing this, the process catches the signal only if the signal is not currently being ignored.

These two calls to signal also show a limitation of the signal function: we are not able to determine the current disposition of a signal 
without changing the disposition. We'll see later in this chapter how the sigaction function allows us to determine a signal's disposition without changing it.
\end{document}
