% !Mode:: "TeX:UTF-8"
\documentclass{article}
% !Mode:: "TeX:UTF-8"
\usepackage[english]{babel}
\usepackage[UTF8]{ctex}
\usepackage{amsmath, amsthm, amssymb}

% Figure
\usepackage{graphicx}
\usepackage{float} %% H can fix the location
\usepackage{caption}
\usepackage[format=hang,singlelinecheck=0,font={sf,small},labelfont=bf]{subfig}
\usepackage[noabbrev]{cleveref}
\captionsetup[subfigure]{subrefformat=simple,labelformat=simple,listofformat=subsimple}
\renewcommand\thesubfigure{(\alph{subfigure})}

\usepackage{epstopdf} %% convert eps to pdf
\DeclareGraphicsExtensions{.eps,.mps,.pdf,.jpg,.png} %% bmp, gif not supported
\DeclareGraphicsRule{*}{eps}{*}{}
\graphicspath{{img/}{figure/}{../figure/}} %% fig directorys

%% \usepackage{pstricks} %% a set of macros that allow the inclusion of PostScript drawings directly inside TeX or LaTeX code
%% \usepackage{wrapfig} %% Wrapping text around figures

% Table
\usepackage{booktabs} %% allow the use of \toprule, \midrule, and \bottomrule
\usepackage{tabularx}
\usepackage{multirow}
\usepackage{colortbl}
\usepackage{longtable}
\usepackage{supertabular}

\usepackage[colorinlistoftodos]{todonotes}

% Geometry
\usepackage[paper=a4paper, top=1.5cm, bottom=1.5cm, left=1cm, right=1cm]{geometry}
%% \usepackage[paper=a4paper, top=2.54cm, bottom=2.54cm, left=3.18cm, right=3.18cm]{geometry} %% ms word
%% \usepackage[top=0.1cm, bottom=0.1cm, left=0.1cm, right=0.1cm, paperwidth=9cm, paperheight=11.7cm]{geometry} %% kindle

% Code
%% \usepackage{alltt} %% \textbf can be used in alltt, but not in verbatim

\usepackage{listings}
\lstset{
    backgroundcolor=\color{white},
    columns=flexible,
    breakatwhitespace=false,
    breaklines=true,
    captionpos=tt,
    frame=single, %% Frame: show a box around, possible values are: none|leftline|topline|bottomline|lines|single|shadowbox
    numbers=left, %% possible values are: left, right, none
    numbersep=5pt,
    showspaces=false,
    showstringspaces=false,
    showtabs=false,
    stepnumber=1, %% interval of lines to display the line number
    rulecolor=\color{black},
    tabsize=2,
    texcl=true,
    title=\lstname,
    escapeinside={\%*}{*)},
    extendedchars=false,
    mathescape=true,
    xleftmargin=3em,
    xrightmargin=3em,
    numberstyle=\color{gray},
    keywordstyle=\color{blue},
    commentstyle=\color{green},
    stringstyle=\color{red},
}

% Reference
%% \bibliographystyle{plain} % reference style

% Color
\usepackage[colorlinks, linkcolor=blue, anchorcolor=red, citecolor=green, CJKbookmarks=true]{hyperref}
\usepackage{color}
\def\red#1{\textcolor[rgb]{1.00,0.00,0.00}{#1}}
\newcommand\warning[1]{\red{#1}}

% Other
%% \usepackage{fixltx2e} %% for use of \textsubscript
%% \usepackage{dirtree}  %% directory structure, like the result of command tree in bash shell


% !Mode:: "TeX:UTF-8"
%+++++++++++++++++++++++++++++++++++article+++++++++++++++++++++++++++++++++
%customize the numbering of equation, to make it like section-subsection-equation style, for example,1-2-3
\makeatletter\@addtoreset{equation}{subsection}\makeatother
\renewcommand\theequation{%
\thepart\arabic{section}%
-\thepart\arabic{subsection}%
-\thepart\arabic{equation}%
}
%theorem
\newtheorem{definition}{D\'efintion} %% 整篇文章的全局编号
\newtheorem*{thmwn}{Thm} %% without numbers
\newtheorem{theorem}{Th\'eor\`eme}[section] %% 从属于section编号
\newtheorem{corollary}{Corollary}[theorem] %% 从属于theorem编号
\newtheorem{lemma}{Lemma}
\newtheorem{proposition}{Proposition}[section]
\newtheorem{example}{Example}
\newtheorem*{attention}{Attention}
\newtheorem*{note}{Note}
\newtheorem*{remark}{Remark}
\newtheorem{question}{Question}[section]
\newtheorem{problem}{Problem}
\newtheorem{fact}{Fact}


\begin{document}
\title{Machine Learning}
\author{}
\maketitle
%% \newpage
\tableofcontents
\newpage
\section{Introduction}
algorithms for inferring unkowns from knows.

\textbf{Applications}
\begin{itemize}
\item Spam mail
\item Handwriting
\item Google Streetview
\item Speech Recognition
\item Netflix, 视频推荐
\item Navigation of robot
\item Climate modelling
\end{itemize}

\textbf{Classes of ML problems}:
\begin{itemize}
\item Supervised: Given $((x_1, y_1), (x_2, y_2), \ldots, (x_n, y_n))$
	\begin{itemize}
	\item classification
	\item regression
	\end{itemize}
\item unsupervised: Given $(\vector{x})$, find patterns in the data
	\begin{itemize}
	\item Clustering
	\item density estimation: Given data comes from a unkown probability distribution
	\item dimensional reduction: 降维(but perserve the structure)
	\end{itemize}
\item Semi-supervised: Given $((x_1, y_1), (x_2, y_2), \ldots, (x_k, y_k), x_{k+1}, \ldots, x_n)$, predict $y_{k+1}, \ldots, y_n$
\item Active learning
\item Decision theory
\item Reinforcement learning: rewards/losses, maximize lifetime reward
\end{itemize}
\href{http://i.imgbox.com/sVuBzaqV.png}{unsupervised ML}

\textbf{Generative Discriminative(区别对待) approaches}\\
Given $((x_1, y_1), (x_2, y_2), \ldots, (x_k, y_k), x_{k+1}, \ldots, x_n)$\\
discriminative: $P(y|x)$\\
geneerative: $P(x,y) = f(x|y)P(y) = P(y|x)f(x)$
It is called generative, because it use the density distribution $f(x)$ or $f(y)$

Examples of generative models include:
\begin{itemize}
\item Gaussian mixture model and other types of mixture model
\item Hidden Markov model
\item Probabilistic context-free grammar
\item Naive Bayes
\item Averaged one-dependence estimators
\item Latent Dirichlet allocation
\item Restricted Boltzmann machine
\end{itemize}

Examples of discriminative models used in machine learning include:
\begin{itemize}
\item Logistic regression, a type of generalized linear regression used for predicting binary or categorical outputs (also known as maximum entropy classifiers)
\item Linear discriminant analysis
\item Support vector machines
\item Boosting (meta-algorithm)
\item Conditional random fields
\item Linear regression
\item Neural networks
\end{itemize}

\section{K-Nearest Neighbor(KNN)}
Given $((x_1, y_1), (x_2, y_2), \ldots, (x_k, y_k), x_{k+1}, \ldots, x_n), x$\\
$x_i \in R^d, y_i \in \{0, 1\}$

\textbf{Majority vote of $k$ nearest points}\\
\href{http://upload.wikimedia.org/wikipedia/commons/thumb/e/e7/KnnClassification.svg/220px-KnnClassification.svg.png}{KNN demo}

$k=1$, 找到最近的1个点\\
$d(x_i, x_j) = \norm{x_i - x_j}$

$k=3$, 找到最近的3个点, 看label 1多还是label 0多.

Prob interpretation. Fix $k$\\
$P(y) =$ fraction of points $x_i$ in $N_k(x)$ s.t $y_i = y$\\
有时也记为$P(y|x, D)$, this is discriminative

How to select parameter $k$?\\
Cross validation, etc.

\section{Tree: CART approach}
\subsection{Decision Trees(CART Approach)}
Given $((x_1, y_1), (x_2, y_2), \ldots, (x_n, y_n)), x$\\

\textbf{Binary tree, minimize error in each leaf}.

see note in paper

\subsection{Regression Tree}
see note in paper

\subsubsection{Growing a regression tree}
Greedy algo

$x_i \in R^d$

1. First split: choose $j$ and $s$ to minimize the quantity:
$$
\min_y (\sum_{i: x_{ij} > s} (y - y_i)^2 + \sum_{i: x_{ij} \leq s} (y - y_i)^2)
$$
As the given data is finite, for a choosen $j$, there is only a finite possibilities for $s$, \\
so we can split the whole region into two regions, we note one of them as $R$.

2. Spliting region $R$: choose $j$ and $s$ to minimize
$$
\min_y (\sum_{i: x_{ij} > s \et x_i \in R} (y - y_i)^2 + \sum_{i: x_{ij} \leq s \et x_i \in R} (y - y_i)^2)
$$

3. Stop when one point

4. only consider splits resulting in regions with $\geq 5$ points in per region

\subsection{classification tree}
\subsubsection{Growing a classification tree}
see note in paper

\subsection{Generalization}
Impurity(不纯, 杂质) measures:
\begin{itemize}
\item misclassfication rate
\item entropy $H_R$
\item $F_R$: Gini index, $G_R = \sum_{y \in Y} P_R(y)(1 - P_R(y))$
\end{itemize}

Other issues about this kind of model
\begin{itemize}
\item Categorical predictiors
\item Loss matrix
\item Missing values, $(x_1, y_1), \ldots, (x_i, y_i)$, and for some $i,j$, $x_{ij}$ is missing, it is noted N/A
\item Linear combinations
\item Instability, means sensitive to the data, that is high variance, but we can use aggregation to deal with it.
\end{itemize}

\section{Bootstrap aggregation(Bagging)}
Ofen prove the performance

Given $(x_1, y_1), (x_2, y_2), \ldots (x_n, y_n) \obey P \ iid$\\
Given $x$, predict $y$

我们知道, 如果我们有很多组的given data, 那么, 我们可以在每组数据中, 求得$x$对应的$y$, 然后对这些求得的$y$ 取平均值, 我们有理由确信, 如果given data的组数
足够多, 计算得到的平均值是很接近与真实值的.\\
bootstrap 就是类似于这样操作的一种方法.\\
由于我们知道数据的分布规律, 那么我们可以sampling 多次, 得到多组数据, 如下:\\
$(x_1^{(1)}, y_1^{(1)}), (x_2^{(1)}, y_2^{(1)}), \ldots (x_n^{(1)}, y_n^{(1)})$ \\
$(x_1^{(2)}, y_1^{(2)}), (x_2^{(2)}, y_2^{(2)}), \ldots (x_n^{(2)}, y_n^{(2)})$ \\
$\vdots$\\
$(x_m^{(m)}, y_m^{(m)}), (x_2^{(m)}, y_2^{(m)}), \ldots (x_n^{(m)}, y_n^{(m)})$ \\

\smallskip
$EY = y = f(x)$, unbiased estimation

\section{K-means Cluestering}
无监督, greedy algo, a special case of EM algo

Given $D = (\vector{x}), x_i \in R^d$

Intuition: points are near to the center of their cluster

Assume $K$ clusters, with unknown centers $\mu_1, \mu_2, \ldots, \mu_k$
$$
L
= \sum_{j = 1}^k \sum_{i | x_i \in\ cluster\ j}\norm{x_i - \mu_j}^2
= \sum_{j = 1}^k \sum_i^n a_{ij}\norm{x_i - \mu_j}^2
$$
where 
$$
a_{ij} =
\left\{
  \begin{array}{ll}
    1, & \si x_i \in\ cluster\ j\\
    0, & \sinon
  \end{array}
\right.
$$
We want to choose $a_{ij}$ and $\mu_i$ to minimize $L$.

K-means: try to minimize $L$ with repect to(wrt) $a$ and $\mu$
\begin{enumerate}
\item Init $\mu_1, \mu_2, \ldots, \mu_k$
\item Choose optimal $a$ for fixed $\mu$
\item Choose optimal $\mu$ for fixed $a$
\item Repeat 2 and 3 until convergence
\end{enumerate}

For step 2:\\
对于固定的$\mu$, 也就是说我们知道了所有clusters的centers, 需要将$x_i$分配到这些clusters中, 直觉告诉我们应该\\
assign $x_i$ to the nearset $\mu_j$. i.e.
$$
a_{ij} =
\left\{
  \begin{array}{ll}
  1, & \si j = \arg \min_l \norm{x_i - \mu_l}^2\\
    0, & \sinon
  \end{array}
\right.
$$

For step 3:\\
对于固定$a$, 也就是说我们知道了$x_i$都是属于哪个cluster的, 然后需要找出最佳的cluster's center.\\
首先我们的直觉告诉我们, cluster center应该为属于这个cluster的点的中心, 也就是平均值.

$$
\begin{aligned}
\end{aligned}
L 
= \sum_{j = 1}^k \sum_i^n a_{ij}\norm{x_i - \mu_j}^2
= \sum_{j = 1}^k \sum_i^n a_{ij}(x_i - \mu_j)^T (x_i - \mu_j)
$$

$$
0 = \grad_{\mu_j}{L} = \sum_i \sum_l a_{ij} \grad{(x_i - \mu_l)^T (x_i - \mu_l)}
$$

例如: $n=3, k=2, d=2$, 也就是说把3个二维的数据分到两个clusters中.\\
$x_i = (x_{i1}, x_{i2})^T$
$\mu_i = (\mu_{i1}, \mu_{i2})^T$
$a_{ij}$ 表示第$i$个点属于第$j$个cluster.

$$
\begin{aligned}
L
= & \sum_{j = 1}^2 \sum_{i=1}^3 a_{ij}(x_i - \mu_j)^T (x_i - \mu_j) \\
= &  a_{11}(x_1 - \mu_1)^T (x_1 - \mu_1) + a_{12}(x_1 - \mu_2)^T (x_1 - \mu_2) \\
& + a_{21}(x_2 - \mu_1)^T (x_2 - \mu_1) + a_{22}(x_2 - \mu_2)^T (x_2 - \mu_2) \\
& + a_{31}(x_3 - \mu_1)^T (x_3 - \mu_1) + a_{32}(x_3 - \mu_2)^T (x_3 - \mu_2) \\
\end{aligned}
$$
如果我对$\mu_1$求gradient, 那么不含有$\mu_1$(这里就是含有$\mu_2$)的项就为0,
$$
\begin{aligned}
\grad_{\mu_1}{L} 
& = \sum_{i = 1}^3 a_{i1}\grad{(x_i - \mu_1)^T (x_i - \mu_1)} \\
& = \sum_{i = 1}^3 a_{i1} \times (-2(x_i - \mu_1)) \\
& = \sum_{i = 1}^3  -2a_{i1}(x_i - \mu_1) \\
\end{aligned}
$$
上式用到了simulation et optimisation中学到的一个公式:
$$
F(x)= \frac{1}{2} \ps{Ax}{x} - \ps{b}{x}
$$
\`ou, A une matrice de dimension $n$, sym\'trique d\'efinie positive. 那么
$$DF(x).h = \ps{Ax - b}{h}$$
$$\grad{F(x)} = Ax - b$$
且$F(x)$ est une fontion convexe, 所有有一个minimum

所以
$$
\begin{aligned}
0 = \grad_{\mu_j}{L} 
&= \sum_i \sum_l a_{il} \grad{(x_i - \mu_l)^T (x_i - \mu_l)} \\
&= \sum_i a_{ij} \grad{(x_i - \mu_j)^T (x_i - \mu_j)} \\
&= \sum_i -2a_{ij} (x_i - \mu_j) \\
&= -2(\sum_i a_{ij}x_i - \mu_j \sum_i a_{ij})
\end{aligned}
$$

$$
\Rightarrow 
\mu_j = \dfrac{\sum_i a_{ij}x_i}{\sum_i a_{ij}}
$$
$\sum_i a_{ij}$  表示属于cluster $j$个点的个数\\
所以这确实是cluster的重心

为了确认L在这个$\mu_j$取到的是一个minimum, 而不是maximum, 我们求一下二阶导.

$$
\frac{\partial }{\partial \mu_{jk}} \grad_{\mu_{j} L}
= 2(\sum_i a_{ij}) e_k
\Rightarrow 
\grad^2_{\mu_j} L = 2(\sum_i a_{ij})I > 0
$$
所以这是一个minimum, 而不是一个maximum

\href{https://www.youtube.com/watch?v=zHbxbb2ye3E}{K-means demo}

\section{EM(expectation-maximization) algo}
A class of algos, 
it's more of an approach to derive an aglo for approximately obtaining (MLE)maximum-likelihood estimation or (MAP)maximum a posteriori estimates of parameters
when some of the data is missing.

Given: $x = (\vector(x))$\\
Model: $(X, Z) \obey p_\theta$ for some (unkown) $\theta \in \Theta$

Motivation of EM:

\end{document}
