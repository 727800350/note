% !Mode:: "TeX:UTF-8"
\section{Improving Deep Neural Networks}

\subsection{Data Sets}
train / dev / test set

Make sure train / dev and test come from the same distribution.

\subsection{Bias and Variance}
eg: Cat classification

Human error $\simeq 0$
\begin{itemize}
\item Train set error $1\%$ and dev set error $11\%$ => high variance
\item Train set error $15\%$ and dev set error $16\%$ => high bias
\item Train set error $15\%$ and dev set error $30\%$ => high bias and high variance
\item Train set error $0.5\%$ and dev set error $1\%$ => low bias and low variance
\end{itemize}

Basic recipe
\begin{itemize}
\item High bias(training set problem)? -> bigger network / train longer / NN architecture search
\item High variance? -> more data / regularization / NN architecture search
\end{itemize}

\subsection{Regularization}
If you suspect your neural network is over fitting your data, that is you have a high variance problem.
One of the first thing you should try is probably regularization.
The other way to address high variance is to get more training data, that is also quite reliable.
But you can not always get more training data, or it could be expensive to get more data.

\subsubsection{Logistic Regression}
$\min_{w, b} J(w, b)$

$$J(w, b) = \dfrac{1}{m} \sum_{i = 1}^m L(\hat{y}^{(i)}, y^{(i)}) + \dfrac{\lambda}{2m} ||w||_2^2$$

L2 regularization: $\dfrac{\lambda}{2m} ||w||_2^2 = \dfrac{\lambda}{2m} \sum_{i = 1}^{n_x} w_i^2 = w^T w$

L1 regularization: $\dfrac{\lambda}{2m} ||w||_1 = \dfrac{\lambda}{2m} \sum_{i = 1}^{n_x} |w|$, $w$ will be sparse.
\todo{难道用L2就不是稀疏的了? 一个平方求和, 一个绝对值求和, 差不多啊.}

$\lambda$ is regularization parameter

\subsubsection{Neural Network}
\begin{equation}
J(w^{[1]}, b^{[1]}, \ldots, w^{[L]}, b^{[L]}) = \dfrac{1}{m} \sum_{i = 1}^m L(\hat{y}^{(i)}, y^{(i)}) + \dfrac{\lambda}{2m} \sum_{l = 1}^L ||W^{[l]}||_F^2
\label{eq.regu.nn}
\end{equation}

Frobenius norm: $||W^{[l]}||_F^2 = \sum_{i = 1}^{n^{[l - 1]}} \sum_{j = 1}^{n^{[l]}} (W_{ij}^{[l]})^2$

$$
\left\{
\begin{aligned}
& dW^{[l]} = (\text{form back prop}) \\
& W^{[l]} := W^{[l]} - \alpha dW^{[l]}
\end{aligned}
\right.
$$
becomes:
$$
\left\{
\begin{aligned}
& dW^{[l]} = (\text{form back prop}) + \dfrac{\lambda}{m} W^{[l]} \\
& W^{[l]} := W^{[l]} - \alpha dW^{[l]} = (1 - \alpha \dfrac{\lambda}{m}) W^{[l]} - \alpha (\text{form back prop})
\end{aligned}
\right.
$$
For this reason, L2 regularization is sometimes called "weight decay".

\subsubsection{How does regularization prevent overfitting?}
In the regularization function \eqref{eq.regu.nn} of neural network, if we set the regularization parameter $\lambda$ very large, as the object is to minimize $J$, so the weight matrixs $W$ will be reasonably close to zero.
So one piece of intuition is maybe it set the weight to be so close to zero for a lot of hidden units that's basically zeroing out a lot of impact of these hidden units.
And if that is the case, this much simplified neural network becomse a much smaller neural network. Then less overfitting.

过拟合的时候,拟合函数的系数往往非常大.
因为过拟合意味着拟合函数需要顾忌每一个点,最终形成的拟合函数波动很大.在某些很小的区间里,函数值的变化很剧烈.
这就意味着函数在某些小区间里的导数值(绝对值)非常大, 由于自变量值可大可小, 所以只有系数足够大, 才能保证导数值很大.
而正则化是通过约束参数的范数使其不要太大, 所以可以在一定程度上减少过拟合情况.

$\lambda \uparrow \Longrightarrow W^{[l]} \downarrow$, in the case of $\tanh$ as the activation function, every layer is rougly linear, then the whole network is linear.
So it is not able to overfit(as overfit functions are normally non linear functions).

\subsubsection{Dropout regularization}
illustrate with layer $l = 3$, keep prob $= 0.8$.

$$
\left\{
\begin{aligned}
& d^{[3]} = np.random.rand(shape) < keep\_prob \\
& a^{[3]} = np.multiply(a^{[3]}, d^{[3]}) \\
& a^{[3]} /= keep\_prob
\end{aligned}
\right.
$$
最后再除以keep prob 是为了保持输出的scale 一致, 不管有没有采用drop out.

You should not use drop out when making predictions at test time. Because we do not want our prediction to be random.

\subsubsection{Data augmentation}
有时候不是因为算法好赢了,而是因为拥有更多的数据才赢了.

不记得原话是哪位大牛说的了,hinton?从中可见训练数据有多么重要,特别是在深度学习方法中,更多的训练数据,意味着可以用更深的网络,训练出更好的模型.

既然这样,收集更多的数据不就行啦?如果能够收集更多可以用的数据,当然好.但是很多时候,收集更多的数据意味着需要耗费更多的人力物力,有弄过人工标注的同学就知道,效率特别低,简直是粗活.

所以,可以在原始数据上做些改动,得到更多的数据,以图片数据集举例,可以做各种变换,如:
\begin{itemize}
\item 将原始图片旋转一个小角度
\item 添加随机噪声
\item 一些有弹性的畸变(elastic distortions),论文<Best practices for convolutional neural networks applied to visual document analysis>对MNIST做了各种变种扩增.
\item 截取(crop)原始图片的一部分.比如DeepID中,从一副人脸图中,截取出了100个小patch作为训练数据,极大地增加了数据集.感兴趣的可以看<Deep learning face representation from predicting 10,000 classes>.
\end{itemize}

\subsubsection{Early stopping}

\subsubsection{Normalize input}
zero mean and normalize variance.

use the same normalization strategy to dev and test set.

\subsection{Vanishing/exploding gradients}
suppose $g(z) = z$ and $b^{[l]} = 0$, we have:
$$\hat{y} = W^{[l]} W^{[l - 1]} \cdots W^{[2]} W^{[1]} x$$
\begin{itemize}
\item if we suppse $W^{[l]} = \begin{bmatrix}1.5 & 0 \\ 0 & 1.5\end{bmatrix}$, the gradient will explode.
\item if we suppse $W^{[l]} = \begin{bmatrix}0.5 & 0 \\ 0 & 0.5\end{bmatrix}$, the gradient will vanish.
\end{itemize}

partial solution in Logistic Regression:

large $n \longrightarrow$ small $w_i$

For relu activation function, we want $\var{w_i} = \dfrac{2}{n}$
$$W^{[l]} = np.random.randn(shape) \times np.sqrt(\dfrac{2}{n^{[l - 1]}})$$

for $\tanh$ activation function, use xaiver initialization $np.sqrt(\dfrac{1}{n^{l - 1}})$

