\chapter{Branch and Bound}
\section{Th\'eorie}
分支界定法是求解整数线性规划最优解的经典方法.

\textbf{定义}:
对有约束条件的最优化问题(其可行解为有限数)的所有可行解空间恰当地进行系统搜索,这就是分支与界定的内容.
通常把全部解空间反复地分割为越来越小的子集,称为分枝,并对每个子集内的解集计算一个目标下界(对于最小值问题),这称为定界.
在每次分枝后,若某个已知可行解集的目标值不能达到当前的界限,则将这个子集舍去.这样,许多子集不予考虑,这称为剪枝.
这就是分枝界限法的思路.

\textbf{背景}:
分枝界限法可以用于求解纯整数或混合的整数规划问题.在上世纪六十年代由Land Doig和Dakin等人提出.这种方法灵活且便于用计算机求解,目前已经成功运用于求解生产进度问题,旅行推销员问题,工厂选址问题,背包问题及分配问题等.

\textbf{思路}:
设有最大化的整数规划问题$A$, 与它相应的线性规划问题时$B$.
从解问题$B$开始, 若其最优解不符合$A$的整数条件,那么$B$的最优目标函数必是$A$的最优目标函数$z^*$的上界,记作$\overline{z}$, 而$z$的任意可行解的目标函数值将是$z$的一个下界$\underline{z}$.
分枝界定法就是把$B$的可行域分成子区域的方法, 逐步减小$\overline{z}$ 和增大$\underline{z}$, 最终求到$z^*$.

\section{Example}
求解整数规划
$$ \max z = 40 x_1 + 90 x_2 $$
$$
s.t.
\begin{cases}
9 x_1 + 7 x_2 <= 56 \\
7 x_1 + 20 x_2 <= 70 \\
x_1, x_2 \in \mathbb{Z}^+
\end{cases}
$$

\immediate\write18{pic="branch_bound.png" && [ ! -f $pic ] && wget https://images2.imgbox.com/48/c2/ade6stUA_o.png -O $pic}
\begin{figure}[htbp]
	\centering
	\includegraphics[scale = 0.5]{branch_bound}\\
	\caption{Branch and bound}\label{fig.branch_bound}
\end{figure}

\textbf{layer 0}\par
先不考虑整数的限制, 即解相应的线性规划$B$, 得到最优解为
$$ x_1 = 4.809, x_2 = 1.817, z = 355.89$$
但是它不符合整数条件. 这时$z$ 是问题的问题A的最优目标函数值$z^*$的上界, 记为$\overline{z}$.
而$x_1 = 0, x_2 = 0$ 显然是问题$A$的一个整数可行解, 这时$z = 0$, 是$z^*$的一个下界, 记为$\underline{z}$, 因此$0 <= z^* <= 355$.

\textbf{layer 1}\par
因为$ x_1 = 4.809, x_2 = 1.817$ 当前为非整数, 故不满足$A$的要求, 任选一个进行分支. 设选$x_1$ 进行分支, 把可行集分成$2$ 个子集.
$$x_1 <= [4.809] = 4 \et x_1 >= [4.809] + 1 = 5$$
因为$4$ 与 $5$之间无整数, 故两个子集的整数解的并必与原可行集合整数解一致. 这一步称为分支.
这两个子集的规划及求解如下:

问题$B_1$
$$ \max z = 40 x_1 + 90 x_2 $$
$$
s.t.
\begin{cases}
9 x_1 + 7 x_2 <= 56 \\
7 x_1 + 20 x_2 <= 70 \\
0 <= x_1 <= 4, x_2 >= 0
\end{cases}
$$
最优解为$$x_1 = 4, x_2 = 2.1, z_1 = 349$$

问题$B_2$
$$ \max z = 40 x_1 + 90 x_2 $$
$$
s.t.
\begin{cases}
9 x_1 + 7 x_2 <= 56 \\
7 x_1 + 20 x_2 <= 70 \\
x_1 >= 5, x_2 >= 0
\end{cases}
$$
最优解为$$x_1 = 5, x_2 = 1.571, z_1 = 341.390$$
再定界: $0 <= z^* <= 349$

\textbf{layer 2}\par
问题$B_1$ 的最优解$x_1 = 4, x_2 = 2.1$ 不是整数, 按照$x_2$再进行分支.

问题$B_{11}$
$$ \max z = 40 x_1 + 90 x_2 $$
$$
s.t.
\begin{cases}
9 x_1 + 7 x_2 <= 56 \\
7 x_1 + 20 x_2 <= 70 \\
0 <= x_1 <= 4, 0 <= x_2 <= 2
\end{cases}
$$
最优解为$$x_1 = 4, x_2 = 2, z_1 = 340$$
\textbf{这里刚好为整数解, 因此可以把$z^*$的下界提高到$340$}

问题$B_{12}$
$$ \max z = 40 x_1 + 90 x_2 $$
$$
s.t.
\begin{cases}
9 x_1 + 7 x_2 <= 56 \\
7 x_1 + 20 x_2 <= 70 \\
0 <= x_1 <= 4, x_2 >= 3
\end{cases}
$$
最优解为$$x_1 = 1.428, x_2 = 3, z_1 = 327.12$$

\textbf{再定界}: $340 <= z^* <= 349$

对问题$B_2$再进行分支得到问题$B_{21}$ 和 $B_{22}$

问题$B_{21}$
$$ \max z = 40 x_1 + 90 x_2 $$
$$
s.t.
\begin{cases}
9 x_1 + 7 x_2 <= 56 \\
7 x_1 + 20 x_2 <= 70 \\
x_1 >= 5, 0 <= x_2 <= 1 
\end{cases}
$$
最优解为$$x_1 = 5.4, x_2 = 1, z_1 = 307.760$$

问题$B_{22}$
$$ \max z = 40 x_1 + 90 x_2 $$
$$
s.t.
\begin{cases}
9 x_1 + 7 x_2 <= 56 \\
7 x_1 + 20 x_2 <= 70 \\
x_1 >= 5, x_2 >= 2 
\end{cases}
$$
无可行解

再继续往下分支, 得到的子集的最优解会越来越小, 因为 $x_1 = 4, x_2 = 2, z = 340$ 就是整个问题的最优解.

