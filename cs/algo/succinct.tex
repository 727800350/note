\section{Succinct}
\href{http://succinct.cs.berkeley.edu/}{berkeley succinct}
\href{https://www.youtube.com/watch?v=O1a6YxFMp0g}{Succinct Spark: Fast Interactive Queries on Compressed RDDs}

Succinct is a data store that enables efficient queries directly on a compressed representation of the input data.
Succinct uses a compression technique that achieves compression close to that of gzip and yet allows random access into the input data.

Succinct natively supports point querys including
\begin{itemize}
\item search
\item range queries
\item regular expressions
\item random access
\end{itemize}

What differentiates Succinct from previous data stores is that Succinct supports these queries without storing any secondary indexes, without requiring data scans and without decompressing the data.
\red{All the required information is embedded within the compressed representation and queries are executed directly on the compressed representation}.

As a base API, Succinct exposes a simple interface that supports above queries on flat files.
Applications that perform queries on semi-structured data can extend this API to build higher-level data representations.

On real-world and benchmark datasets, Succinct requires as much as an order of magnitude lower storage compared to state-of-the-art systems with similar functionality.
As a result, Succinct executes more queries in faster storage, leading to lower query latency than existing systems for a much larger range of input sizes.

What do we lose?
\begin{itemize}
\item Preprocessing time
\item Extra CPU for data access: When you are doing random access, succinct requires extract CPU cycles because you have to decompress the at least for the range of data that you are accessing.
\item Low Sequential scan throughput: succinct is primarily designed to enable random access, which means that sequential throughput is going to be lower.
\item Do not support in-place updates: succinct supportes updates by append
\end{itemize}

question:
succinct 说压缩后可以把更多的数据放到内存里, 但是需要preprocess, 还挺耗时的.
他是把压缩后的数据都加载到内存中吗?
那他和加索引有什么本质区别, 生成索引需要时间, 索引文件肯定比数据文件小很多, 我觉得比succinct 压缩后的数据还是小很多, 那用索引文件可以把更多的meta data 放在内存里面, query 不是更快吗?

视频中有C++ 的lib, 但是没有找到

