% !Mode:: "TeX:UTF-8"
\documentclass{article}
% !Mode:: "TeX:UTF-8"
\usepackage[english]{babel}
\usepackage[UTF8]{ctex}
\usepackage{amsmath, amsthm, amssymb}

% Figure
\usepackage{graphicx}
\usepackage{float} %% H can fix the location
\usepackage{caption}
\usepackage[format=hang,singlelinecheck=0,font={sf,small},labelfont=bf]{subfig}
\usepackage[noabbrev]{cleveref}
\captionsetup[subfigure]{subrefformat=simple,labelformat=simple,listofformat=subsimple}
\renewcommand\thesubfigure{(\alph{subfigure})}

\usepackage{epstopdf} %% convert eps to pdf
\DeclareGraphicsExtensions{.eps,.mps,.pdf,.jpg,.png} %% bmp, gif not supported
\DeclareGraphicsRule{*}{eps}{*}{}
\graphicspath{{img/}{figure/}{../figure/}} %% fig directorys

%% \usepackage{pstricks} %% a set of macros that allow the inclusion of PostScript drawings directly inside TeX or LaTeX code
%% \usepackage{wrapfig} %% Wrapping text around figures

% Table
\usepackage{booktabs} %% allow the use of \toprule, \midrule, and \bottomrule
\usepackage{tabularx}
\usepackage{multirow}
\usepackage{colortbl}
\usepackage{longtable}
\usepackage{supertabular}

\usepackage[colorinlistoftodos]{todonotes}

% Geometry
\usepackage[paper=a4paper, top=1.5cm, bottom=1.5cm, left=1cm, right=1cm]{geometry}
%% \usepackage[paper=a4paper, top=2.54cm, bottom=2.54cm, left=3.18cm, right=3.18cm]{geometry} %% ms word
%% \usepackage[top=0.1cm, bottom=0.1cm, left=0.1cm, right=0.1cm, paperwidth=9cm, paperheight=11.7cm]{geometry} %% kindle

% Code
%% \usepackage{alltt} %% \textbf can be used in alltt, but not in verbatim

\usepackage{listings}
\lstset{
    backgroundcolor=\color{white},
    columns=flexible,
    breakatwhitespace=false,
    breaklines=true,
    captionpos=tt,
    frame=single, %% Frame: show a box around, possible values are: none|leftline|topline|bottomline|lines|single|shadowbox
    numbers=left, %% possible values are: left, right, none
    numbersep=5pt,
    showspaces=false,
    showstringspaces=false,
    showtabs=false,
    stepnumber=1, %% interval of lines to display the line number
    rulecolor=\color{black},
    tabsize=2,
    texcl=true,
    title=\lstname,
    escapeinside={\%*}{*)},
    extendedchars=false,
    mathescape=true,
    xleftmargin=3em,
    xrightmargin=3em,
    numberstyle=\color{gray},
    keywordstyle=\color{blue},
    commentstyle=\color{green},
    stringstyle=\color{red},
}

% Reference
%% \bibliographystyle{plain} % reference style

% Color
\usepackage[colorlinks, linkcolor=blue, anchorcolor=red, citecolor=green, CJKbookmarks=true]{hyperref}
\usepackage{color}
\def\red#1{\textcolor[rgb]{1.00,0.00,0.00}{#1}}
\newcommand\warning[1]{\red{#1}}

% Other
%% \usepackage{fixltx2e} %% for use of \textsubscript
%% \usepackage{dirtree}  %% directory structure, like the result of command tree in bash shell


% !Mode:: "TeX:UTF-8"
%+++++++++++++++++++++++++++++++++++article+++++++++++++++++++++++++++++++++
%customize the numbering of equation, to make it like section-subsection-equation style, for example,1-2-3
\makeatletter\@addtoreset{equation}{subsection}\makeatother
\renewcommand\theequation{%
\thepart\arabic{section}%
-\thepart\arabic{subsection}%
-\thepart\arabic{equation}%
}
%theorem
\newtheorem{definition}{D\'efintion} %% 整篇文章的全局编号
\newtheorem*{thmwn}{Thm} %% without numbers
\newtheorem{theorem}{Th\'eor\`eme}[section] %% 从属于section编号
\newtheorem{corollary}{Corollary}[theorem] %% 从属于theorem编号
\newtheorem{lemma}{Lemma}
\newtheorem{proposition}{Proposition}[section]
\newtheorem{example}{Example}
\newtheorem*{attention}{Attention}
\newtheorem*{note}{Note}
\newtheorem*{remark}{Remark}
\newtheorem{question}{Question}[section]
\newtheorem{problem}{Problem}
\newtheorem{fact}{Fact}


\begin{document}
\title{Automatique}
\maketitle
\tableofcontents
\newpage
%% \chapter{Automation}
\section{自动控制的基本概念}
在自动控制技术中, 把工作的机器称为被控对象
把表征这些机器设备工作状态的物理参量称为被控量
而对这些物理参量的要求值称为给定值或希望值(或参考输入)
则控制的任务可概括为: 使被控对象的被控量等于给定值

多变量系统是现代控制理论研究的主要对象, 在数学上采用状态空间法为基础, 讨论多变量, 变参数, 非线性, 高精度, 高效能等控制系统的分析与设计.

放大元件: 放大倍数越大, 系统的反应越敏感, 一般情况下, 只要系统稳定, 放大倍数应适当大些.

\textbf{稳态响应的含义}: 通常习惯上把不随时间变化的静态称为稳态. 然而, 在控制系统中, 往往响应已达稳态, 但他们可能随时间有规律地变化. 因此, 控制系统中的稳态响应, 简单说来就是指时间趋于无穷大的确定的响应.

对自动控制系统性能的要求在时域中一般可归纳为\textbf{三大性能指标}:
\begin{description}
\item[稳定性]
\item[瞬态质量] 要求系统瞬态响应过程具有一定的快速性与变化的平稳性
\item[稳态误差]
\end{description}
同一个系统, 上述三大性能指标往往相互制约. 提高控制过程的快速性, 可能会引起系统强烈震荡; 改善了平稳性, 控制过程又可能很迟缓, 甚至使得最终精度也很差.

\section{控制系统的数学模型}
零点用$\bigcirc$表示, 极点用$\times$表示 \newline
极点(虚轴左边)离虚轴越远, 相应的模态收敛越快 \newline
当零点不靠近任何极点时, 诸极点相对来说, 距离零点远一些的极点其模态所占比重较大, 若零点靠近某极点, 则对应模态的比重就减小, 所以离零点很近的极点比重会被大大削弱(\textbf{偶极子}). 当零极点相重, 产生零极点对消时, 相应的模态就消失了.

\textbf{串联}: 传递函数相乘 \newline
\textbf{并联}: 传递函数相加 \newline
结构图变换: 比较点前移, 引出点后移


闭环系统的常用传递函数\newline
开环传递函数: 将回环从反馈的末端$B(s)$处断开, 沿回环从误差信号$E(s)$开始至末端反馈信号$B(s)$终止, 期间经过的的通道传递函数的乘积称为开环传递函数\newline
\textbf{前向传递函数}:有输入到输出的直接通过传递函数的乘积
\begin{equation}
		\text{闭环传递函数} = \frac{\text{前向传递函数}}{1+\text{开环传递函数}}
\end{equation}

无论外部输入信号取何种形式和作用与系统任一输入端(如控制输入端或扰动输入端), 也不论输出信号选择哪个变量(如$C(s)$或$E(s)$), 所对应的闭环传递函数都具有相同的特征方程. \newline
这就是说, 系统的闭环极点与外部输入信号的形式和作用点无关, 同时也与输出信号的选取无关, 仅取决于闭环特征方程的根.\newline
由此可进一步说明, 系统响应的自由运动模态是系统的固有属性, 与外部激发信号无关.


Mason梅逊增益公式
\begin{equation}
		G = \frac{\sum_{k=1}^n P_k * \Delta_k}{\Delta}
\end{equation}
G:总增益\newline
$P_k$: 前向传递函数\newline
$\Delta$: 信号流图的特征式

\section{时域分析法}
\subsection{典型输入信号}
\begin{itemize}
	\item 阶跃信号
	\item 斜坡信号
	\item 抛物线信号
	\item 脉冲信号
	\item 正弦信号
\end{itemize}

\subsection{一阶系统时域分析}
一阶系统的单位阶阶跃响应
\subsection{二阶系统}
典型二阶系统

阻尼(欠阻尼, 临界阻尼, 过阻尼, 负阻尼)
\newline 重点是欠阻尼的分析, 包括调节时间, 超调量

\subsection{稳定性分析}
劳斯稳定判据

误差函数(0,I,II型系统)

\section{根轨迹法}
如开环传递函数
$$
G(s) = \frac{ K}{s(0.5s+1)} = \frac{K_g}{s(s+2)}
$$
\begin{itemize}
	\item n阶系统有n条分支
	\item 根轨迹上每只分支起始于开环极点处, 结束于有限零点处或者无线点处
	\item 开环传递函数分子多项式阶数m小于分母多项式阶数n, 因此有$(n-m)$条根轨迹的终点在无穷远处
\end{itemize}

增加开环零极点对根轨迹的影响\newline
\textbf{增加开环零点}: 在复平面内的共轭复根的根轨迹向左弯曲, 而且分离点左移, 故输出响应的动态过程衰减较快, 超调量减小, 系统的相对稳定性较好\newline
\textbf{增加开环极点}: 在复平面内的共轭复根的根轨迹向右弯曲, 而且分离点右移,这时, 其快速性已大为下降, 相对稳定性也变差

\section{线性系统的频率响应法}
$G(j\omega)$ 就是系统的频率特性, 就是相当于把传递函数$G(s)$中的$s$换成$j\omega$

\textbf{极坐标图}: 当$\omega$ 从 $0 \to \infty$, 频率特性$G(j\omega)$的矢端轨迹

\textbf{伯德图}(对数幅频与对数相频两条曲线)\newline
横轴按频率的对数$\lg \omega$标尺刻度, 但标出的是频率$\omega$ 本身的数值, 因此横轴的刻度是不均匀的

典型环节
\begin{itemize}
  \item 比例环节
  \item 积分环节
  \item 微分环节
  \item 惯性环节 (交接频率)
  \item 一阶微分环节
  \item 震荡环节
  \item 二阶微分环节
  \item 延迟环节
\end{itemize}

频段
\begin{description}
	\item[低频段] 指所有交接频率之前的区段, 这一段特性完全由积分环节和开环增益决定
	\item[中频段] 通常是指在幅值穿越频率$\omega_c$(截止频率)附近的区段, 这一段特性集中反映了闭环系统动态响应的平稳性和快速性
	\item[高频段] 指幅频曲线在中频段以后($\omega > 10\omega_c$)的区段, 这部分特性是由系统中时间常数很小, 频带很宽的部件决定的. 由于远离$\omega_c$, 一般分贝值又较低, 故对系统的动态影响不大. 另外, 从系统抗干扰的角度看, 高频段特性是有其意义的. 高频段的幅值, 直接反映了系统对输入端高频信号的抑制能力. 这部分特性分贝值越低, 系统抗干扰能力越强.
\end{description}

最小相位系统: 开环零点和开环极点全部位于s 左半平面的系统
\subsection{Nyquist稳定判据}
当$G(s)H(s)$ 包含积分环节时, 在对数相频曲线$\omega$为$0_+$的地方, 应该补画一条从相角$\angle G(j0_+)H(j0_+)+N*90^\circ$ 到$\angle G(j0_+)H(j0_+)$
的虚线, 这里N 是积分环节数.\newline
计算正负穿越时, 应将补画的虚线看成对数相频特性曲线的一部分.

\subsection{开环频域指标}
稳定裕度: 赋值裕度和相角裕度

\section{线性系统的可控性和可观测性}
在用传递函数描述的经典控制系统中,输出量一般是可控的和可以被测量的,因而不需要特别地提及可控性及可观测性的概念.\\
现代控制理论用状态方程和输出方程描述系统,输出和输入构成系统的外部变量,而状态为系 统的内部变量.
系统就好比是一块集成电路芯片,内部结构可能十分复杂,物理量很多,而外部只有少数几个引脚,对电路内部物理量的控制和观测都只能通过这为数不多的几个引脚 进行.\\
这就存在着\textbf{系统内的所有状态是否都受输入控制和所有状态是否都可以从输出反映出 来的问题,这就是可控性和可观测性问题}.

如果系统所有状态变量的运动都可以通过有限的控制点的输入来使其由任意的初态达到任意设定的终态,则称系统是\textbf{可控的},更确切地说是状态可控的;否则,就称系统是不完全可控的,简称系统不可控.

可控的充分必要条件:$(B, AB, A^2B, \ldots, A^{n-1}B)$ 满秩

相应地,如果系统所有状态变量的任意形式的运动均可由有限测量点的输出完全确定出来,则称系统是\textbf{可观测的},简称系统可观测;反之,则称系统是不完全可观测的,简称系统不可观测.

可观测的充分必要条件:$(C, CA, CA^2, \ldots, CA^{n-1})^t$ 满秩

可控性与可观测性的概念,是用状态空间描述系统引申出来的新概念,在现代控制理论中起着重要的作用。可控性、可观测性与稳定性是现代控制系统的三大基本特性。

如果某个状态变量可直接用仪器测量,它必然是可观测的.在多变量系统中,能直接测量的状态一般不多,大多数状态变量往往只能通过对输出量的测量间接得到,有些状态变量甚至根本就不可观测.\\
需要注意的是,出现在输出方程中的状态变量不一定可观测,不出现在输出方程中的状态变量也不一定就不可观测.

\bigskip
传递函数可约时,传递函数分母阶次将低于系统特征方程阶次.若对消掉的是系统的一个不稳定特征值,便可能掩盖了系统固有的不稳定性而误认为系统稳定.\\
通常说用传递函数描述系统特性不完全,就是指它可能掩盖系统的不可控性,不可观测性及不稳定性.只有当系统是可控又可观测时,传递函数描述与状态空间描述才是等价的. 

\href{http://www.bulingfei.com/Control/ChineseText/Chapter2Section6.aspx?ChineseChapter2Section6}{传递矩阵}
\end{document}
