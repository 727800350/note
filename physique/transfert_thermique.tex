% !Mode:: "TeX:UTF-8"
\documentclass{article}
% !Mode:: "TeX:UTF-8"
\usepackage[english]{babel}
\usepackage[UTF8]{ctex}
\usepackage{amsmath, amsthm, amssymb}

% Figure
\usepackage{graphicx}
\usepackage{float} %% H can fix the location
\usepackage{caption}
\usepackage[format=hang,singlelinecheck=0,font={sf,small},labelfont=bf]{subfig}
\usepackage[noabbrev]{cleveref}
\captionsetup[subfigure]{subrefformat=simple,labelformat=simple,listofformat=subsimple}
\renewcommand\thesubfigure{(\alph{subfigure})}

\usepackage{epstopdf} %% convert eps to pdf
\DeclareGraphicsExtensions{.eps,.mps,.pdf,.jpg,.png} %% bmp, gif not supported
\DeclareGraphicsRule{*}{eps}{*}{}
\graphicspath{{figure/}{../figure/}} %% fig directorys

%% \usepackage{pstricks} %% a set of macros that allow the inclusion of PostScript drawings directly inside TeX or LaTeX code
%% \usepackage{wrapfig} %% Wrapping text around figures

% Table
\usepackage{booktabs} %% allow the use of \toprule, \midrule, and \bottomrule
\usepackage{tabularx}
\usepackage{multirow}
\usepackage{colortbl}
\usepackage{longtable}
\usepackage{supertabular}

\usepackage[colorinlistoftodos]{todonotes}

% Geometry
\usepackage[paper=a4paper, top=1.5cm, bottom=1.5cm, left=1cm, right=1cm]{geometry}
%% \usepackage[paper=a4paper, top=2.54cm, bottom=2.54cm, left=3.18cm, right=3.18cm]{geometry} %% ms word
%% \usepackage[top=0.1cm, bottom=0.1cm, left=0.1cm, right=0.1cm, paperwidth=9cm, paperheight=11.7cm]{geometry} %% kindle

% Code
%% \usepackage{alltt} %% \textbf can be used in alltt, but not in verbatim

\usepackage{listings}
\lstset{
    backgroundcolor=\color{white},
    columns=flexible,
    breakatwhitespace=false,
    breaklines=true,
    captionpos=tt,
    frame=single, %% Frame: show a box around, possible values are: none|leftline|topline|bottomline|lines|single|shadowbox
    numbers=left, %% possible values are: left, right, none
    numbersep=5pt,
    showspaces=false,
    showstringspaces=false,
    showtabs=false,
    stepnumber=1, %% interval of lines to display the line number
    rulecolor=\color{black},
    tabsize=2,
    texcl=true,
    title=\lstname,
    escapeinside={\%*}{*)},
    extendedchars=false,
    mathescape=true,
    xleftmargin=3em,
    xrightmargin=3em,
    numberstyle=\color{gray},
    keywordstyle=\color{blue},
    commentstyle=\color{dkgreen},
    stringstyle=\color{mauve},
}

% Reference
%% \bibliographystyle{plain} % reference style

% Color
\usepackage[colorlinks, linkcolor=blue, anchorcolor=red, citecolor=green, CJKbookmarks=true]{hyperref}
\usepackage{color}
\def\red#1{\textcolor[rgb]{1.00,0.00,0.00}{#1}}
\newcommand\warning[1]{\red{#1}}

% Other
%% \usepackage{fixltx2e} %% for use of \textsubscript
%% \usepackage{dirtree}  %% directory structure, like the result of command tree in bash shell


% !Mode:: "TeX:UTF-8"
% Equation Number
\makeatletter\@addtoreset{equation}{subsection}\makeatother %% reset the equation number in subsection
\renewcommand\theequation{\thepart\arabic{section}-\thepart\arabic{subsection}-\thepart\arabic{equation}} %% section-subsection-equation style

% Theorem
\newtheorem{definition}{D\'efintion} %% document global number
\newtheorem*{thmwn}{Thm} %% without numbers
\newtheorem{theorem}{Th\'eor\`eme}[section] %% section-theorem style
\newtheorem{corollary}{Corollary}[theorem] %% theorem-corollary style
\newtheorem{lemma}{Lemma}
\newtheorem{proposition}{Proposition}[section]
\newtheorem*{attention}{Attention}
\newtheorem*{note}{Note}
\newtheorem*{remark}{Remark}
\newtheorem{example}{Example}
\newtheorem{question}{Question}[section]
\newtheorem{problem}{Problem}
\newtheorem*{answer}{Answer}
\newtheorem{fact}{Fact}

% Header and Footer
\newif\ifheader
\headerfalse
\ifheader
	\setlength\textheight{100.0pt}
	\setlength\textwidth{430.0pt}
	\usepackage{fancyhdr}
	\usepackage{lastpage} %% \pageref{LastPage}: get the last page
	\usepackage{pdfpages}
	\usepackage{layout}
	\footskip = 20pt
	\pagestyle{fancy}
	\fancyhead{} %% clear all fields
%% 	\newsavebox{\headpic}
%% 	\sbox{\headpic}{\includegraphics[height=1cm]{logo}} %% set header logo
%% 	\lhead{\usebox{\headpic}}
	\rhead{\small\leftmark}
	\fancyfoot{} %% clear all fields
	\cfoot{\thepage}
	\renewcommand{\headrulewidth}{1pt}  %% header line width, can be set 0 to get rid of it
	\setlength{\skip\footins}{0.5cm}    %% distance between of footnote and body text
	\renewcommand{\footnotesize}{}      %% footnote size
\fi


% Equation
\newcommand\lasteq{(\theequation)\ } %% use \lasteq to reference the last equation
\newcommand{\eqspace}{\hspace{0.5cm}}
\newcommand{\eqnote}[1]{\text{ #1 }} %% insert text in math mode being treated as normal text
\newcommand\mytop[2]{\genfrac{}{}{0pt}{}{#1}{#2}} %% generate a fraction but without the line

% Vecteur
\def\vecteur#1{(#1_1,~#1_2,~\ldots,~#1_n)}
\def\vector#1{#1_1,~#1_2,~\ldots,~#1_n}

% Set
\newcommand{\R}{\mathbb{R}} %% the real number set
\newcommand{\N}{\mathbb{N}}
\newcommand{\Z}{\mathbb{Z}}
\newcommand{\Q}{\mathbb{Q}}
\newcommand{\set}[1]{\{#1\}}
\newcommand{\stcomp}[1]{\overline{#1}} %% set complement

% Logic
\newcommand{\si}{\textrm{\ if }}
\newcommand{\sinon}{\textrm{ si non}}
\newcommand{\then}{\textrm{ then }}
\newcommand{\et}{\textrm{\ et }}
\newcommand{\ou}{\textrm{ ou }}
\newcommand{\non}{\textrm{non }}
\newcommand{\ssi}{si et seulement si }

% Math Operator
\newcommand{\fun}[1]{\textit{#1}}
\DeclareMathOperator{\arccot}{arcot}
\DeclareMathOperator{\arcth}{arcth}
\DeclareMathOperator{\arcsh}{arcsh}
\DeclareMathOperator{\arch}{arch}
\DeclareMathOperator{\ch}{ch}
\DeclareMathOperator{\dth}{th} %% \th already used
\DeclareMathOperator{\sh}{sh}
\DeclareMathOperator{\var}{var}
\DeclareMathOperator{\Ker}{Ker}
\DeclareMathOperator{\Img}{Img}

\newcommand*\laplace{\mathop{}\!\mathbin\bigtriangleup}
\newcommand*\dalambert{\mathop{}\!\mathbin\Box}
\newcommand{\grad}[1]{\nabla #1}
\newcommand{\gradien}[1]{\nabla #1}
\newcommand{\divergence}[1]{\nabla \cdot #1}
\newcommand{\rotationnel}[1]{\nabla \times #1}
\newcommand{\rot}[1]{\nabla \times #1}

\newcommand{\diag}[1]{\textit{diag}(#1)}
\newcommand{\mean}[1]{\overline{#1}}
\newcommand{\estimate}[1]{\hat{#1}}
\newcommand{\indep}{\!\perp\!\!\!\perp}
\newcommand{\nindep}{\not\!\perp\!\!\!\perp}
\newcommand{\norm}[1]{\left\Vert #1\right\Vert}
\newcommand{\obey}[1]{\thicksim{#1}}

\usepackage{xspace}
\newcommand{\ps}[2]{\ensuremath{\langle #1 , #2\rangle}\xspace} %% produit scalaire

%% quantique operators
\newcommand\ket[1]{|#1\rangle}
\newcommand\bra[1]{\langle #1|}
\newcommand\braket[3]{\langle#1|#2|#3\rangle}

% Symbol
\newcommand{\infinity}{\infty}


\begin{document}
\title{Transfert thermique}
\author{整理: \href{mailto:wangchaogo1990@gmail.com}{Eric Wang}}
\maketitle
\tableofcontents
\newpage
%% \chapter{Transfert thermique}
\section{Introduction}
经典热力学只研究在系统的初, 终状态之间参数发生了什么变化, 而且认为系统与外界间的热量交换是在无限小温差下发生的一系列无限缓慢的平衡过程.\\
然而任何实际的热量传递过程都必须存在着温度梯度. 热量传递是典型的非平衡过程, 即不可逆过程, 他所研究的恰恰是热力学没有给出答案, 但又是工程实践中无法回避的现实问题-
系统内外的热量交换是以何种方式, 多大的速率进行的.\\
换一个角度看待这个问题, 也可以认为热力学中可逆方式交换的热量必定是传热学(即实际工程)意义上可能达到的最高极限.

\bigskip
\textbf{传热问题的研究方法}\\
从某种角度讲, 一切传热问题所遵循的基本规律只有两个, 一个是特定传热问题的特殊规律, 第二个就是能量守恒(在对流换热情况下还要涉及质量守恒和动量守恒等基本法则).

在运用能量守恒法则时, 必须首先明确定义分析的对象是一个控制容积(control volume), 还是一个控制面(control surface)? 这两者之间有原则性区别.\\
对控制容积还应进一步区分微元容积与有限容积, 用前者得到的一般是微分方程, 用后者得到积分方程或代数方程.
\begin{enumerate}
\item 针对控制容积的能量守恒方程
	\begin{enumerate}
	\item 能量平衡式中各项的单位都是$W$瓦, 即单位时间传递的热流量
	\item 进入控制容积的热流量, 指以各种方式进入的热功率的总和. 涉及对流时, 则指流体在流入控制容积的同时代入的热流量. 当涉及辐射方式时, 一般换热仅在物体的表面上进行(气体辐射及透热介质除外)
	\item 内热源. 可以为负, 即为实际上的冷热源
	\item 在非相变条件下, 蓄存热量的增加意味着温度升高. 增量为负值表示蓄存减少, 温度降低. 显然稳态传热时, 该项一定为零.
	\item 除非特殊情况, 否则研究传热问题时一般忽略动能和势能的变化
	\item 上述能量守恒关系还有一种表达形式, 既不针对瞬时值, 而是就某一有限时间间隔而言. 这只要对上述平衡关系中的每一项作时间积分即可.
	\end{enumerate}
\item 针对表面的能量守恒方程\\
	任何物体或系统都存在表面, 物体通过表面才得以和外界进行热量的传递和交换. 针对表面的能量守恒关系与针对给定容积时有原则性区别. 原因是表面既没有体积, 也没有质量,
	因此内热源项和蓄热项将不复存在, 也就是说, 针对表面的能量守恒关系只有进与出两项, 不管是瞬时功率还是一段时间内的累积热量, 都必须随时保持相等.
	\begin{center}
	$\Phi_{in} = \Phi_{out}$ 或者 $Q_{in} = Q_{out}$
	\end{center}
	针对表面的这种能量守恒原则无论对稳态或非稳态, 瞬时或任意时间间隔都是正确的
\end{enumerate}

Toute la physique des transferts dans les milieux continues repose sur l'hypoth\`ese de \textbf{l'\'equilibre thermodynamique local(E.T.L)}\\
Cela signifie qu'il est toujours possible de d\'efinir, \`a chaque instant $t$, en tout point $r$, les variables physiques usuelles, eg. t\'emp\'erature $T(r,t)$.

\bigskip
r\'ef\'erentiel d'un \'el\'ement du syst\`eme mat\'eriel: transport par des particules $\Rightarrow$ diffusion
\begin{itemize}
\item conduction \'electrique
\item diffusion d'une esp\`ese dans un milieu
\item conduction thermique
\end{itemize}

r\'ef\'erentiel quelconque: convection.\\
les ph\'enom\`enes de diffusion se couplent aux ph\'enom\`enes de convection dans un r\'ef\'erentiel quelconque.

\bigskip
$$ d\Phi = \vec{q} \cdot \vec{n} dS $$
$d\Phi$: la puissance \'el\'ementaire [$W$]; flux \'el\'ementaire d'\'energie \`a travers $dS$\\
$\vec{q}$: vecteur flux surfacique d'\'energie [$W/m^2$]

$$\varphi = \vec{q} \cdot \vec{n}$$
$\varphi$: flux surfacique, $>0$ dans le sens des axes

\section{Transfert conduction et conducto-convetif}
\textbf{Conduction et conducto-convection}\\
(CD)Le transfert thermique dans le fluide \`a le paroi:
$$\varphi^{cd} |_{pf} = -\\\lambda _f \frac{\partial T_f}{\partial y}|_{pf}$$

(CC)Le transfert conduto-convectif(根据正方向):
$$\varphi ^{cc}=h(T_p - T_c) \quad \text{由p到c为正方向}$$

Transferts conductifs stationnaires au sein d'un milieu immobile.

$\lambda, h$ 与 $T$ 无关 $\Rightarrow$ transferts conductifs lin\'eaires
\subsection{热阻}
\subsubsection{无限长侧表面绝热套筒的热阻}
左视剖面如图所示:
\href{http://i.imgbox.com/X9xXlBML.png}{示意图}\\
截取的套筒微元长度为dz(图中未标出,即为灰色方块的长度),沿径向方向的温度场分布记为T(r),热量由中心向外扩展

$$
d\Phi = -\lambda \dfrac{dT}{dr}\cdot dS = -\lambda \dfrac{dT}{dr} \cdot 2\pi rdz
$$
$$-dT = \dfrac{d\Phi}{2\pi \lambda dz} \cdot \dfrac{dr}{r}$$
$$ T_1 - T_2 = \dfrac{d\Phi}{2\pi \lambda dz} \ln{\dfrac{R_2}{R_1}} $$
($R_1 \leftrightarrow r_1 et R_2 \leftrightarrow r_2$)

显然这个模型的径向热阻为 $R_t = \dfrac{1}{2\pi \lambda dz} \ln{\dfrac{R_2}{R_1}}$

\subsubsection{有限长套筒热阻}
在上一题的基础上,将套筒的长度限定为L,则将上式积分可得,总的热阻为(en coordonn\'ees cart\'esiennes):
$$ R_t = \dfrac{L}{\pi \lambda(R_2^2 - R_1^2)} $$

\subsubsection{对流热阻}
考虑流体-固体界面处的热阻模型,固体壁的温度为$T_p$,接触面积为$\Sigma$ ,液体的温度为$T_c$,对流换热系数为$h$.\\
则有$\Phi^{cc} = h\Sigma(T_p - T_c)$ , 所以热阻
$$R_t = \dfrac{1}{h\Sigma}$$
注:以后的很多实际传热问题中,换热系数h并不是单纯的对流换热系数,而是"对流换热与辐射换热系数之和"!!! $h=h^{cc} + h^R$

\subsection{Ailette}
为了提高换热效率,从经济性的角度考量,大多会想办法增大"固-液-气"的接触面积.
所以聪明的工程师们设计出了"肋片"这一装置.
此种设计充分体现了\textbf{分形会增大系统表面积}这一精妙的数学思想~~

所谓肋片,就是粘在主换热片/管上的小薄片装置,剖视图如图所示
\href{http://i.imgbox.com/UBJtc7TX.png}{肋片的示意图}

肋片横截剖面上的温度场分析可以近似采用"线性均匀分布"假设,即:温度场的梯度仅仅沿z轴方向.

线性模型的建立,还需要牢记一个基本条件:材料导热系数不随温度的变化而产生变化,这就导致系统的焓变为零.
\todo{Why}
这一假设将广泛运用于以下章节的分析中.

$B_i \leq 0.1 \Rightarrow$ approximation ailette valide en terme de champ de t\'emp\'erature, et en terme de flux.

对于一般图形来说,没有一个确定的量能够单独表示特征尺寸,所以一般用$\dfrac{\text{面积}}{\text{周长}}$来表示这个特征尺寸\\

$$D_t = \frac{A}{P}$$

对于圆柱体: $D_t = \dfrac{A}{P} = \dfrac{\pi R^2}{2 \pi R} = \dfrac{R}{2}$

对于正方体: $D_t = \dfrac{A}{P} = \dfrac{a^2}{4a} = \dfrac{a}{4}$

$$ B_i = \frac{h D_t}{\lambda} = \frac{h}{\lambda} \cdot \frac{A}{P} = \frac{hA}{\lambda P}$$

$$ m^2 = \frac{hP}{\lambda A}$$
\textbf{\red{$B_i$ 与 $m$中, $h$ 都在分子部分}}

$$ D_t^2 m^2 = B_i$$

\subsubsection{Ailette id\'eale}
理想肋片由铝或者铜制成,其导热性能非常强.假设其长度为L,则应有
$$ m\cdot l \leq 1 \quad \frac{T(z)-T_f}{T_p - T_f} \simeq 1 $$

La temp\'erature de l'ailette rest \textbf{en tout point} tr\`es voisine de $T_p  \Rightarrow \dfrac{ T - T_f}{T_p - T_f}\simeq 1$

传导传热的时间效应消失掉了.这一模型的传热效率(根据前两节的公式求极限即可得到)为:
$$
\eta_{isotherme} = \dfrac{L}{D_t} = \dfrac{LP}{A}
$$

\subsubsection{Ailette infinie}
无线长的肋片中,总能将肋片的温度场调整到与环境达到平衡(温度场形成一个稳定的分布, 不随时间变化).
而在现实应用中,一般将符合如下条件的肋片按照无限长肋片模型处理,即
$$ m\cdot l \geq 5 $$
此时有:
$$ \frac{T(z)-T_f}{T_p - T_f}=exp(-mz) $$
传热效率: $\eta_{\infty} = \dfrac{\lambda m}{h} = \dfrac{1}{\sqrt{B_i}}$

Aucun flux n'est plus dissip\'e en bout d'ailette. En effet, au bout d'une certaine longueur, la temp\'erature de l'ailette est proche de la temp\'erature de r\'ef\'erence du fluide.Le flux \'echang\'e tend vers $0$.

肋片热场的近似,最吸引工程师们的就是在Bi远小于1时的情况(一般要求小于等于0.1).
这时候才可以说如果$B_i \leq 0.1$,物体最大与最小过余温度之差小于5\%,对于一般工程计算,此时已经足够精确的可以认为整个物体温度均匀

\subsubsection{D\'emonstration}
\href{http://i.imgbox.com/IttTCHOi.png}{Ailette 的模型图}

向上为 $z$ 正方向

\textbf{进入的符号取正, 出去的为负, 在稳态条件下, 进入的与出去的代数和为零(对于下面的control volume)}
\begin{eqnarray}
\text{en z:} & \Phi_1=\varphi(z)\cdot A \\
\text{en z+dz:} &\Phi_2=-\varphi(z+dz) \cdot A\\
\text{en surface lat\'eral:} & \Phi_3=h(T_a - t(z)) \cdot P dz \\
\text{三个flux相加为零:}  &\Phi_1 + \Phi_2 + \Phi_3 = 0
\end{eqnarray}

边界条件(control surface)
\begin{equation}
\begin{aligned}
\eqnote{en} z = 0, \eqspace T(0) = T_0 \\
\eqnote{en} z = l \eqnote{bout d'ailette}, \eqspace - \lambda \frac{dT}{dz} = h[T(l) - T_a]
\end{aligned}
\end{equation}

R\'esultat:
$$ \varphi(z)\cdot A -\varphi(z+dz)\cdot A + h(T_a - t(z)) \cdot P dz = 0 $$

得到
$$
\frac{T(z) - T_a}{T_0 - T_a}
=\frac
{\exp[m(l-z)](1 + m\lambda/h) - \exp[-m(l-z)](1 - m\lambda/h)}
{\exp(ml)(1+m\lambda/h) - \exp(-ml)(1-m\lambda/h)}
$$

用到的公式(泰勒展开):
$$ \frac{ d}{dx}(y+\frac{dy}{dx})  = \frac{ dy}{dx}+\frac{ d^2y}{dx^2}dx $$

\begin{equation}
	\begin{split}
	\frac{ d^2T}{dz^2}=\frac{ hP_m}{\lambda A}(T(z)-T_f)=m^2(T(z)-T_f) \\
	 T-T_f=A e^{mz} + B e^{-mz}
	\end{split}
\end{equation}
Pour l'ailette infinie
\begin{equation}
	\begin{split}
	  \text{温度不能无穷大}\Rightarrow A=0 \\
	  \eqnote{en } z=0 \quad T_p - T_f =B \Rightarrow \frac{T-T_f}{T_p - T_f}=\exp(-mz)
	\end{split}
\end{equation}


温度梯度大的地方,传热效率不一定高,相反温度梯度低的地方也有可能传热效率很高,例如ailette,由于横向传热很快,
导致横向几乎没有温度差,没有温度差也就没有温度梯度,但是横向传热效率很高

\section{Rayonnement}
\textbf{Milieu transparant et corps opaque}
\begin{description}
\item[Un milieu transparant]: il n'int\'egit pas de champ de rayonnement; il n'\'emet pas, n'absorbe pas, ne r\'efl\'echit pas, ni ne diffuse de rayonnement; tout rayonnment incident est transmis quelles que soient sa direction et sa fr\'equence
\item[Un corps opaque] ne transmet aucunne fraction d'un rayonnement incident(i); le rayonnement incident est soit absorb\'e(a), soit r\'efl\'echit(r)
\end{description}

\subsection{Classfication}
Les grandeurs physiques seront distingu\'ees selon :
\begin{itemize}
\item La composition spectrale du rayonnement
    \begin{itemize}
    \item Si la grandeur est relative \`a l'ensemble du spectre elle est dite \textit{totale}.
    \item Si elle concerne un intervalle spectral \'etroit $d \lambda$  autour d'une longueur d'onde $\lambda$  elle est dite \textit{monochromatique} : $G_{\lambda }$.
    \end{itemize}

\item La distribution spatiale du rayonnement
    \begin{itemize}
    \item Si la grandeur est relative \`a l'ensemble des directions de l'espace, elle est dite \textit{h\'emisph\'erique}.
    \item Si elle caract\'erise une direction donn\'ee de propagation, elle est dite \textit{directionnelle} : $G_x$.
    \end{itemize}
\end{itemize}
温度为$T$的物体的:
L'\'energie \'emise est maximale pour une certaine longueur d'onde $\lambda _m$\\
$\lambda_m$ est le plus repr\'esent\'ee, 其中的下标$m$ 不是表示波长最大, 而是指当波长取这个值时, luminance 最大(根据la loi de Planck)
$$\lambda _m=\frac{3000k\cdot \mu m}{T} \eqnote{书中使用的是}2898$$
上面的这个方程表明: le rayonnment est une notion autant thermodynamique qu'electromagn\'etique. Les seules m\'ethodes pratiques pour mesurer des t\'emp\'eratures \'elev\'ee sont radiatives.

物体(en l\'equilibre de temp\'erature T)的辐射波长范围(在这个范围内包括了$98\%$的能量):
$$[\frac{\lambda _m }{2},8 \lambda _m]$$

\noindent
\`A l'\'equilibre thermique,le flux radiatif surfacique $\varphi^R$ est nul.\\
\`A l'\'equilibre thermique,$\varphi^i=\varphi^p,\varphi^R=\varphi^p-\varphi^i=0$. \\
Le flux\\
($\theta_1,\theta_2$ les angles $(n_1,u_1),(n_2,u_2)$ et $u_1$ le vecteur unitaire de $O_1$ vers $O_2$)
\begin{equation}
\begin{aligned}
d \Phi_{\lambda} 
& = L^{'}_{\lambda}(O_1,u_1) \frac{d S_1 \cos \theta_1  d S_2 \cos \theta_2}{O_1 O_2^2} d \lambda \\
& = L^{'}_{\lambda}(O_1,u_1) d S_1 \cos \theta_1 d\Omega_1 d \lambda \eqnote{从$O_1$看$dS_2$}\\
& = L^{'}_{\lambda}(O_1,u_1) d S_2 \cos \theta_2 d\Omega_2 d \lambda \eqnote{从$O_2$看$dS_1$}
\end{aligned}
\end{equation}
方程中$L^{'}$ 中一撇表示这个$L$是与direction 有关的;上面的关系可以表示incident, absorb\'e, r\'efl\'echi, \'emis et partant

\subsection{H\'emisph\'erique}
Cas g\'en\'erale(\textbf{H\'emisph\'erique})

$\theta$为半顶角($\theta \in [0,\pi/2]$)的c\^one

从\href{http://i.imgbox.com/90aLTQdS.png}{立体角的计算}中可以得出立体角的公式为$d\Omega = \sin \theta d\theta d\varphi $\\
其中$\theta$ 为经线圈上的夹角, $\varphi$ 为纬线圈上的夹角.

从\href{http://i.imgbox.com/1adl5otW.png}{$\theta$的积分域从$0$到$\pi/2$的图析}中可以看到, 当$\theta$取一个值时, 实际上得到的是一个c\^one, 因为$\varphi$ 会旋转$2\pi$个角度, 同时, 由于只考虑半球面,所以$\theta$ 只用取$0$到$\pi/2$.

角度为$\theta'$的c\^one的立体角为:
$$\Omega = \int d\Omega= \int_0^{2\pi} \int_0^{\theta'} \sin \theta d\theta d\varphi = 2\pi(1-\cos \theta')$$

\begin{equation}
	\begin{split}
d\varphi_{\lambda }^s  & = \frac{d \Phi_{\lambda }^s}{dS_1} \\
 & =  \int_{0}^{\pi/2}L_{\lambda }^s(O_1,\theta)\cos \theta d\Omega d \lambda \\
 & = d \lambda \int_{\Omega = 2\pi(1-\cos \theta)} L_{\lambda }^s(O_1,\theta_1)\cos \theta_1 2\pi \sin\theta_1 d\theta_1
	\end{split}
\end{equation}
\lasteq 中是除以$dS_1$, 所以是通过$dS_1$的flux, 也就是说以$dS_1$为研究对象

Cas d'un rayonnement isotrope(luminance ind\'epandante de la direction)
\begin{equation}
	\begin{split}
d\varphi_{\lambda }^s &= \frac{d \Phi_{\lambda }^s}{dS_1}\\
&=d \lambda \int_{0}^{\pi/2}L_{\lambda }^s(O_1)\cos \theta_1 2\pi \sin\theta_1 d\theta_1 \\
&= \pi L_{\lambda }^s(O_1)d\lambda
	\end{split}
\end{equation}

Flux radiatif:
$$d\varphi_{\lambda }^R =d\varphi_{\lambda }^e - d\varphi_{\lambda }^a = d\varphi_{\lambda }^p - d\varphi_{\lambda }^i$$

使用$\varphi^e=\varepsilon \sigma T^4$的前提是要求le corps est isotrope(c'est \`a dire ind\'ependante de direction)\\
如果要用这个公式乘上面积(用一个点的值代替这个面积中所有点的值)还需额外要求le corps est homo\`gene,也就是说各个点的情况是一样的

\begin{description}
\item [\'Emissivit\'e monochromatique directionnelle $\varepsilon_{\lambda}$:] $L_{\lambda }^e(O_1,\theta_1,\varphi_1) = \varepsilon_{\lambda }(O_1,\theta_1,\varphi_1,T_1)L_{\lambda }^0(T_1)$
\item [absorb\'e]  $L_{\lambda}^a =\alpha_{\lambda }(O_1,\theta_1,\varphi_1,T_1)L_{\lambda}^i $
\item [r\'efl\'echi]  $L_{\lambda}^r = L_{\lambda}^i - L_{\lambda}^a$
\item [Corps opaque,pas de transimission] $\alpha_{\lambda }(O_1,\theta_1,\varphi_1,T_1)=\varepsilon_{\lambda }(O_1,\theta_1,\varphi_1,T_1)$
\end{description}

\bigskip
\`A l'\'echelle \'el\'ementaire, les ph\'enom\`enes de transfert par rayonnement et conduction pr\'esentent une grande analogie.
\begin{itemize}
\item Le flux surfacique radiatif provient des contributions \'energ\'etiques des photons traversant dans toutes les directions de l'espace l'unit\'e de surface par unit\'e de temps
\item Le flux surfacique conductif provient, dans le cas d'un gaz monoatomique pour simplifier, des contributions des \'energ\'es cin\'etiques des atomes traversant dans toutes les directions de l'espace l'unit\'e de surface par unit\'e de temps
\end{itemize}

\subsection{Corps particuliers usuels}
\subsubsection{Corps gris}
Les propri\'et\'es radiatives($\alpha,\varepsilon$) sont \textbf{ind\'ependantes de la longeeur d'onde} $\lambda $ \textbf{mais d\'ependante de la direction}.\\
Sur l'hypoth\`ese que les N surfaces $S_j$ sont grises:
$$\forall j: \quad \varepsilon_j =\alpha_j = 1- \rho_j$$
\begin{eqnarray}
\varphi_j^R=\varphi_j^p - \varphi_j^i \\
\varphi_j^p = \varepsilon_j \sigma_j T_j^4 +(1-\varepsilon_j)\varphi_j^i\\
S_j \varphi_j^i = \sum_{k=1}^N S_k f_{kj}\varphi_k^p \Rightarrow \varphi_j^i = \sum_{k=1}^N f_{jk}\varphi_k^p\\
S_k f_{kj}=S_j f_{jk}
\end{eqnarray}

\subsubsection{Corps \`a propri\'et\'es radiatives isotropes}
ind\'ependante de la direction, mais d\'epandante de la longueur d'onde.
$$\varepsilon_{\lambda}(O_1, T_1) = \alpha_{\lambda}(O_1, T_1) = 1 - \rho_{\lambda}(O_1, T_1)$$

\subsubsection{Corps noir}
$$\varepsilon=\alpha=1,\varphi^r=0$$
\todo{$partant - incident = 0?$}
\textbf{Absorber tout rayonnement},donc ne r\'efl\'echit aucun rayonnement.

La luminance monochromatique du rayonnement \'emis:
$$L_{\lambda}^{e^{C.N}}= L_{\lambda}^0(T) \quad \forall(\theta_1,\varphi_1)$$

\subsection{Rayonnement d'\'equilibre}
La luminance du rayonnement d'\'equilibre et celle du rayonnement \'emis par un corps noir est:
$$L^0(T)=\int_{0}^{\infty}L_{\lambda}^0(T)d \lambda =\frac{\sigma}{\pi}T^4$$
$\sigma$ est dite \textit{constante de Stefan}:
$$\sigma=\frac{ 2\pi^5 k_B^4}{15c^2 h^3}=5.670\times 10^{-8} Wm^{-2}K^{-4}$$

Luminance du rayonnement d\'equilibre pour les diff\'erentes t\'emp\'eratures, \href{http://www.afhalifax.ca/magazine/wp-content/sciences/Planck/Le\%20corps\%20noir\_files/rayonnement\_corps\_noir.png}{rayonnement\_corps\_noir}\\
从图上可以看到, 对于不同的温度, 曲线是不会相交的

Le flux surfacique total de rayonnement isotrope incident sur un \'el\'ement de surface ou partant de cet \'el\'ement, \`a l'\'equilibre \`a la temp\'erature $T$:
$$\varphi^i=\varphi^p=\int_{0}^{\infty}\pi L_{\lambda}^0(T)d \lambda =\sigma T^4$$
由于这个积分是从$0$到$\infty$, 实际中通常考虑的是一段波长, 可以采用下面的公式, 并借助查表来计算:
$$
z[\frac{ \lambda _1}{\lambda _m(T)}, \frac{ \lambda _2}{\lambda _m(T)}]
= \frac{ \int_{\lambda _1}^{\lambda _2} L_{\lambda}^0(T)d \lambda }{\int_{0}^{\infty}L_{\lambda}^0(T)d \lambda }
= \frac{ \int_{\lambda _1}^{\lambda _2}\pi L_{\lambda}^0(T)d \lambda }{\sigma T^4}
$$

\section{Transferts convectifs}
La distribution de la température de surface de la \href{http://i.imgbox.com/o4djIdyE.png}{fenêtre}

在$dt$ 时间内穿过$dS$ 面的masse, $dm = \rho \vec{v} dt \cdot \vec{n} dS$
Le d\'ebit de masse \`a traverse $dS$:
$$d\dot{m} = \dfrac{dm}{dt} = \rho \vec{v} \cdot \vec{n} dS$$
\`ou, $\rho$ la masse volumique et $\vec{v}$ la vitesse local du fluide par rapport \`a la surface $dS$, $\vec{n}$ la direction normale de $dS$.\\
后面出现的: $\dot{m} = \rho \vec{v} \cdot \vec{n}S$

Transfer d'enthalpie en M par rapport \`a $Oyz$, caract\'eris\'e par un flux convectif $d\Phi^{cv}$:
$$ d \Phi^{cv} = \rho \vec{v} \cdot \vec{n} h dS = d\dot{m} h $$
\`ou $h$ est l'enthalpie massique local du fluide en M.

Le vecteur flux surfacique convectif est d\'efini par:
$$ \vec{q}^{cv} = \rho \vec{v} h \eqnote{tel que:} \varphi^{cv} = \rho \vec{v} \cdot \vec{n} h$$

\subsection{Th\'eor\`eme de transport} 
\subsubsection{Th\'eor\`eme de R\'enaulde}
Grandeur int\'egrale
$$ \Phi(t)=\int_{V(t)} \varphi(\vec{x},t)dv $$
$\varphi(\vec{x},t)$ grandeur locale volumique

$$
\frac{ d\Phi}{dt}=
\underbrace{\int_{V(t)}\frac{\partial \varphi}{\partial t}dv}_{\text{Instationnarit\'e de } \varphi(\vec{x},t)}
+
\underbrace{\int_{S(t)} \varphi \vec{W}.\vec{n}dS}_{\text{Mouvement de }S(t)}
$$
$\vec{n}$: vecteur normal unitaire\\
$\vec{W}$:vitesse locale en un point de $S(t)$(n'est définie que sur S)\\
\begin{itemize}
\item Domaine mat\'eriel :$\vec{W}=\vec{U}$
\item []由于是syst\`eme mat\'eriel, $dS$ 表面质点的速度必定与syst\`eme 本身的速度一样
\item Domaine fixe : $\vec{W}=0$
\end{itemize}

\begin{example}
\text{\textbf{Conversation de la masse} avec } $\Phi = M,\varphi=\rho,\vec{ W}=\vec{ U}\text{ domaine mat\'eriel}$
\begin{equation}
	\begin{split}
\frac{ dM}{dt} & =\int_D \frac{\partial \rho}{\partial t}dv+ \int_S \rho \vec{ U}.\vec{ n}dS \\
& = \int_D (\frac{\partial \rho }{\partial t} + div(\rho \vec{ U}))dv \\
& =0	
	\end{split}
\end{equation}

En prenant \emph{la limite} d'une particule, on obtient la forme locale
$$
\frac{\partial \rho }{\partial t} + div(\rho \vec{ U})=0
$$
这也被叫做\'equation de continuit\'e.
\end{example}

\begin{example}
$A = \rho a$
$$
\begin{aligned}
\dfrac{\mathcal{A}}{dt} 
& = \dfrac{d}{dt} \int_{V(t)} A dV \\
& = \dfrac{d}{dt} \int_{V(t)} \rho a dV \\
& = \int_{V(t)} \frac{\partial (\rho a)}{\partial t}dV + \int_{S(t)} (\rho a) \vec{v} . \vec{n}dS \\
& = \int_{V(t)} (\rho \frac{\partial a}{\partial t} + a \frac{\partial \rho}{\partial t})dV + \int_{V(t)} \divergence{\rho a \vec{v}}dV \\
& = \int_{V(t)} (\rho \frac{\partial a}{\partial t} + a \frac{\partial \rho}{\partial t} + a \divergence{\rho \vec{v}} + \rho \vec{v}\cdot \gradien{a})dV \\
& = \int_{V(t)} (\rho \frac{\partial a}{\partial t} + a (\frac{\partial \rho}{\partial t} + \divergence{\rho \vec{v}}) + \rho \vec{v}\cdot \gradien{a})dV \\
& = \int_{V(t)} \rho (\frac{\partial a}{\partial t} + \vec{v} \cdot \gradien{a})dV \\
& = \int_{V(t)} \rho \frac{da}{dt}dV
\end{aligned}
$$
\end{example}

\subsection{Bilan d'\'energie}
\subsubsection{Point de vue d'un syst\`eme mat\'eriel}
$$\mathcal{H}_m = mh=\rho V h=\int \rho h dv$$

\begin{equation}
\dfrac{d \mathcal{H}_m}{dt} = \int_{S_m(t)} -(\vec{q}^{\ cd} + \vec{q}^{\ R})\cdot \vec{n}dS + \int_{V_m(t)} P dV
\end{equation}
其中$P$是puissance(en $W$)
Pas de term de convection du point de vue d'un syst\`eme mat\'eriel, 
因为syst\`eme mat\'eriel 没有与外界的物质交换, 所以没有convection(convection本身是通过物质交换来完成的)

\begin{equation}
 	\begin{split}
	\frac{ d\mathcal{H}_m}{dt}
	= & \frac{d}{dt} \int_V \rho h dV\\
	= & \int_V \rho \frac{dh}{dt}dv \ \ ,(dh = c_pdT)\\
	 = & \int_V \rho c_p\frac{dT}{dt}dv
	\end{split}
\end{equation}

\begin{equation}
\dfrac{d \mathcal{H_m}}{dt}
= \int_{V_m(t)} \rho c_p\frac{dT}{dt}dv
= \int_{S_m(t)} -\vec{q}^{\ cd} \cdot \vec{n}dS + \int_{V_m(t)} (P^{R} + P)dV
\end{equation}

\subsubsection{Point de vue d'un syst\`eme ouvert}
Surface: $S_o(t)$, volume: $V_o(t)$

flux convectif d'enthalpie $\vec{q}^{\ cv} = \rho(\vec{v} - \vec{w})h$\\
\`ou $v, w$ sont les vitesses respectives du fluide et de d\'eplacement de la surface par rapport au r\'ef\'erentiel.

$$
\dfrac{d\mathcal{H_o}}{dt}
= \dfrac{d}{dt} \int_{V_o(t)} Hdv
= \int_{V_o(t)} \frac{\partial \rho h}{\partial t}dV + \int_{S_o(t)} \rho h \vec{w} . \vec{n}dS
$$

L\'equation de bilan d'enthalpie pour ce syst\`eme ouvert:
$$
\dfrac{d\mathcal{H_o}}{dt}
= \int_{V_o(t)} \frac{\partial \rho h}{\partial t}dV + \int_{S_o(t)} \rho h \vec{w} . \vec{n}dS
= - \int_{S_o(t)} (\vec{q}^{\ cv} + \vec{q}^{\ cd}). \vec{n}dS + \int_{V_0(t)}(P^R + P )dV
$$

\subsection{Application simple: transferts dans une conduite}
\'ecoulement stationnaire, d\'ebit massique $\dot{m}$ constant dans une conduite cylindrique(de section constante $\Sigma$)

\textbf{Hypoth\`eses simplificatrices}\\
\begin{enumerate}
\item Les propri\'et\'es thermophysiques du fluide ($\lambda, c_p, \ldots$) sont suppos\'es unifromes, ind\'ependantes de la t\'emp\'erature
\item []La masse volumique $\rho$ d\'epends de la t\'emp\'erature
\item La vitesse du fluide $\vec{v}$(valeur alg\'ebrique $u$) est, en tout point, parall\`ele aux g\'en\'eratrices de la conduite $Ox$
\item []en r\'egime stationnaire($\dfrac{\partial \rho}{\partial t} = 0$) , la conversation de la masse($\dfrac{\partial \rho}{\partial t} + \divergence{\rho \vec{v}}=0$), on peut d\'eduire que $\dfrac{\partial (\rho u)}{\partial x} = 0$
\item Le fluide est un gaz parfait ou un liquide
\end{enumerate}

\subsubsection{Bilan d'\'energie en r\'egime stationnaire}
根据syst\`eme mat\'eriel 中的关系:
\begin{equation}
\dfrac{d \mathcal{H}}{dt}
= \int_V \rho c_p\frac{dT}{dt}dv
= \int_V \rho c_p\frac{dT}{dt}dxdydz
\end{equation}

其中(考虑到$v$ 沿着 $Ox$方向, 且 stationnaire)
$$
\dfrac{dT}{dt}
= \frac{\partial T}{\partial t} + \vec{v} . \gradien{T}
= \frac{\partial T}{\partial t} + u_x \frac{\partial T}{\partial x} + u_y \frac{\partial T}{\partial y} + u_z \frac{\partial T}{\partial z}
= u_x \frac{\partial T}{\partial x}
= u \frac{\partial T}{\partial x}
$$
代入\lasteq 得到
\begin{equation}
\dfrac{d \mathcal{H}}{dt}
= \int_V \rho c_p\frac{dT}{dt}dxdydz
= \int_V \rho c_p u \frac{\partial T}{\partial x} dxdydz
\end{equation}
考虑到我们是取的conduite 的$dx$ 一小段为研究对象, 所以我们可以认为在$dx$这一小段, 各个物理参数保持不变,即 $\rho, \dfrac{\partial T}{\partial x}$在$dx$上为定值; 同时由于$c_p$ 为constante.\\
所以上述方程\lasteq 可以化为
\begin{equation}
\dfrac{d \mathcal{H}}{dt}
= \int_V \rho c_p u \frac{\partial T}{\partial x} dxdydz
= dx \int_\Sigma \rho c_p u \frac{\partial T}{\partial x} dydz
= c_p dx \int_\Sigma \rho u \frac{\partial T}{\partial x} dydz
\end{equation}

由于
$$
\frac{\partial}{\partial x}(\rho u T)
= \rho u \frac{\partial T}{\partial x} + T \frac{\partial \rho u}{\partial x}
= \rho u \dfrac{\partial T}{\partial x}
$$
\lasteq 可以继续化简为:
\begin{equation}
\dfrac{d \mathcal{H}}{dt}
= c_p dx \int_\Sigma \rho u \frac{\partial T}{\partial x} dydz
= c_p dx \int_\Sigma \frac{\partial}{\partial x}(\rho u T) dydz
= c_p dx \dfrac{d}{dx}\int_\Sigma \rho u T dydz
\end{equation}

\textbf{t\'emp\'erature de m\'elange} $T_m(x)$:
$$
T_m(x) 
= \dfrac{\int_\Sigma \rho u T dydz}{\int_\Sigma \rho u dydz}
= \dfrac{\int_\Sigma \rho u T dydz}{\dot{m}}
$$
这个温度相当于界面$\Sigma$的平均温度.

得到
$$\int_\Sigma \rho u T dydz = T_m(x) \dot{m} $$
使用这个平均温度, 我么可以化简\lasteq

\begin{equation}
\dfrac{d \mathcal{H}}{dt}
= c_p dx \dfrac{d}{dx}\int_\Sigma \rho u T dydz
= \dot{m} c_p dT_m(x)
\end{equation}

\textbf{bilan d'\'energie pour une tranche \'el\'ementaire}\\
Pour simplifier, on suppose que:
\begin{enumerate}
\item les \'echanges radiatifs sont lin\'erisables et inclus dans $h$ ou n\'egligeables
\item la conduction axiale est n\'egligeable
\end{enumerate}

\begin{equation}
\dfrac{d \mathcal{H}}{dt}
= \dot{m} c_p dT_m(x)
= P dx h [T_p(x) - T_m(x)]
\end{equation}
$T_p(x)$ la t\'emp\'erature de la paroi

Dans la cas \`ou la $T_p$ est constante en $x$
$$ \dfrac{T_m - T_p}{T_0 - T_p} = \exp(\dfrac{-Phx}{\dot{m} c_p}) $$

\section{Math}
\subsection{erf(x)}
In mathematics, the error function (also called the Gauss error function) is a special function (non-elementary) of sigmoid shape which occurs in probability, statistics and partial differential equations. It is defined as:
\begin{equation}
 \mbox{erf}(x)=\frac{2}{\sqrt{\pi}}\int_{0}^{x}e^{-t^2}dt
\end{equation}
The complementary error function, denoted erfc, is defined as
\begin{equation}
 \mbox{erfc}(x)=1-\mbox{erf}(x)=\frac{2}{\sqrt{\pi}}\int_{x}^{\infty }e^{-t^2}dt
\end{equation}
$$\mbox{ierfc}(x)=\int_x^{\infty}\mbox{erfc}(u)du=\frac{ 1}{\sqrt{\pi}}e^{-x^2} -  x \cdot \mbox{erfc}(x)$$
常用数值:
$$\mbox{erf}(\infty)=1$$
$$\mbox{erf}(0.5) \simeq 0.5$$
$$\mbox{ierfc}(0)=\frac{ 1}{\sqrt{\pi}}$$

偶函数\\
For any complex number z:
\begin{equation}
 erf(\overline{z})=\overline{erf(z)}
\end{equation}

\textbf{材料的导热性}
多孔性 凡是良好的保温材料,在结构上大多都是多孔物质,有时人为制成泡沫状,纤维状或层状结构 \\
例如,石头的导热系数约为$1Wm^{-1}K^{-1}$, 但把石头碎成砂子,其导热系数则为$0.3Wm^{-1}K^{-1}$.再把石头制成纤维状(像石棉一样的东西),则导热系数约为$0.05Wm^{-1}K^{-1}$ \\
如上处理后,导热系数降低的理由是因为颗粒间或纤维间的接触面积变小了,期间产生了相当大的接触热阻,再加上间隙中含有大量空气,
而空气的导热系数才$0.023Wm^{-1}K^{-1}$,比固体小多了,所以阻挡了热流的移动.因此无论怎样选择良好的保温材料,几乎都不如空气的导入系数小.
一般情况下,每单位体积的质量越小的物质,导热系数越小. 但是,档函有空间的间隙过大时,由于间隙中的空气产生对流,反而对传热不利.当间隙达到$1cm$时,就会发生这种现象.

\textbf{肋片}
在各种截面形状的最佳肋片中,最佳肋片是倒抛物线形截面肋片,这种肋片所需材料及质量可以比三角形直肋片节约百分之几,
但从加工难易程度及耐久性方面,实际工程都认为三角形直肋片为最佳肋片.

\section{TTM2}
\subsection{Des nombres importants}
Biot:$$B_i = \frac{ hD}{\lambda }=\frac{\tau^{cd}}{\tau^{cc}}$$
$$B_i = \frac{ \text{coefficient de transfert } \times \text{dimension caract\'eristique} }{\text{conductivit\'e thermique}}$$
Comme $$Re=\frac{U\times l}{\nu}$$
$U$: la vitess de fluide(transfert)\\
$\nu$: coefficient de viscosit\'e dynamique\\
Il qui caract\'erise le r\'egime d'\'ecoulement

Diffusivit\'e thermique $a$ en $m^2/s$:$$a=\frac{\lambda }{\rho c_p};\frac{\partial T}{\partial t}=a\triangle T + \frac{P}{\rho c_p}$$

Nombre de Fourier:$$F_0=\frac{ at}{x^2}$$

Le temps caract\'eristique de conduction thermique:$$\tau^{cd}=\frac{x^2}{a}$$
Le temps caract\'eristique de convection thermique:$$\tau^{cc}=\frac{\tau^{cd}}{B_i}=\frac{e^2/a }{he/\lambda }=\frac{\rho c_p e}{h}$$

La longueur caract\'eristique de conduction thermique \`a l'instant $t$:$$l^{cd}=\sqrt{at}$$

Effusivit\'e thermique du mat\'eriau:$$b=\sqrt{\lambda \rho c}$$
il caract\'erise sa capacit\'e \`a \'echanger de l'\'energie thermique avec son environnement\\

La profondeur de p\'en'etration $e_p$:$$e_p=\sqrt{\frac{2a}{\omega}}$$
透射率(当透射率小于某一个值时,可以认为ne passe pas):$$\tau = exp(-\frac{x}{e_p})$$


流体的温度-$x$变化曲线的横截距,式中$\lambda$为fluide的参数:$$\eta=\frac{\lambda }{h}$$

\subsection{Ecoulement}
\subsection{Stationnaire Uniform \'Etabli}
Stationnaire
$$ \frac{\partial  }{\partial t}=0 $$
Uniforme
$$ \frac{\partial  }{\partial x}=0 $$
\'Etablli
Quand T suffisament grand
$$ \frac{\partial  }{\partial x}=0 $$

\textbf{Laminaire}
无论在流体内部还是在流体与表面的界面处,热只通过分子传导进行传递.即,在流体粒子内作为内能所储备的热,按照粒子准微观的分子运动,横跨流线传递给相邻流线上的粒子.
\textbf{Turbulent}
在湍流时,存在横跨流线运输流体块的涡流,通过这个涡流的宏观运动,被运送到其他流体层的流体粒子跟那里的粒子混合,在流体层间传递能量.这种机理被称为涡传导.\\
湍流时,由于分子传导又附加了涡流传导,所以只与分子传导的导热情况相比,其传热效果显著增强了.因此涡流混合的程度越大,传热的效率也越大.

\subsection{Bilan}
\textbf{Hypoth\`eses simplificatrices}
\begin{itemize}
\item \'ecoulement stationnaire
\item propri\'et\'es thermophysiques de fluide $\rho, c_p, \mu,\lambda $ uniform(variation de $T$ et $P$ faible)
\item force volumique $\vec{f}$ ignor\'e
\item puissance thermique $P_{th}=0$
\item fluide soit opaque($\vec{\varphi}^R = 0$ soit transparant($\mbox{div} \vec{\varphi}^R=0)$)
\item \'ecoulement subsonique $\Rightarrow \sum \tau_{ij}\frac{\partial  v_i}{\partial x_j},\chi\frac{ DP_{th}}{Dt}$ n\'egligeable
\end{itemize}
\textbf{Bilan de masse}
$$\frac{\partial  \rho}{\partial t} + \mbox{div} (\rho \vec{v})=0 $$
化简为
$$\mbox{div} \vec{v}=0$$
\textbf{Bilan de quantit\'e de mouvement}
$$
\rho(\frac{\partial  \vec{v}}{\partial t}+ \vec{v}\cdot \vec{\mbox{grad}}\vec{v}) =  \vec{f} - \vec{\mbox{grad}} p + \mu \triangle\vec{v} + \frac{\mu}{3}\vec{\mbox{grad}} \mbox{div} \vec{v}
$$
化简为
$$
\rho(  \vec{v}\cdot \vec{\mbox{grad}}\vec{v}) =   - \vec{\mbox{grad}} p + \mu \triangle\vec{v}
$$
\textbf{Bilan de l'enthalpie}
$$
\rho c_p (\frac{\partial  T}{\partial t} + \vec{v}\cdot \vec{\mbox{grad}} T)=P_{th} - \mbox{div} (\vec{\varphi}^{CD}+\vec{\varphi}^R) +  \sum_{ij} \tau_{ij}\frac{\partial  v_i}{\partial x_j} + \chi\frac{ DP_{th}}{Dt}
$$
化简为
$$
\rho c_p ( \vec{v}\cdot \vec{\mbox{grad}} T)=- \mbox{div} (\vec{\varphi}^{CD}) =\lambda \triangle T
$$

\subsection{Convection forc\'ee}
le nombre de Nusselt:$$N_u = \frac{hD}{\lambda }=\frac{D/\lambda }{1/h}=\frac{R^{cd}}{R^{cc}}=A\cdot (R_e)^{\alpha}\cdot (P_r)^{\beta}$$
C'est donc le rapport de la r\'esistance thermique de conduction par la r\'esistance thermique de convection. Il est d'autant plus \'elev\'e que la convection est pr\'edominante sur la conduction. Il caract\'erise le type de transfert
de chaleur.\\
\textbf{Propri\'et\'es thermophysiques du fluide $\rho, \mu, \nu, C_p, \lambda, a$ sont \'evalu\'es \`a la temp\'erature de film $T_f = (T_0 + T_p)/2$}

Le nombre de Prandtl:
$$
P_r = \frac{ \mu C_p}{\lambda }=\frac{ \nu}{a}
$$
$$
\frac{\text{Diffusivit\'e de quantit\'e de mouvement $\nu$ (ou viscosit\'e cin\'ematique)}}{\text{Diffusivit\'e thermique}}
$$
对于常见的流体:
\begin{description}
\item [Fluides ususels:] $P_r$ de l'ordre $1$
\item [M\'etaux liquides:] $P_r$ de l'ordre $10^{-2}$ 说明液态金属的传热性能非常好
\item [Huiles:] $P_r$ de l'ordre $10^3$
\end{description}

Le nombre de P\'eclet:$$P_e = R_e \cdot P_r$$

\subsubsection{Convection forc\'ee externe laminaire}
\begin{eqnarray}
 0.6<P_r<10 & \frac{ \delta _{th}}{\delta _m}=pr^{-1/3} \\
 & h(x)=\frac{ \lambda }{\frac{ 2}{3}\delta _{th}(x)}
\end{eqnarray}

\subsection{Convection naturelle}
Le nombre de Grashof(comme nombre de Reynolds):
$$G_{r_L} = \frac{ \rho^2 g \beta (T_p - T_0)L^3}{\mu ^2}$$
$$N_{u_L} = C G_{r_L}^{\alpha}P_r^{\beta}$$
$$G_{r_x} = \frac{ \rho^2 g \beta (T_p - T_0)x^3}{\mu ^2}$$
$$N_{u_x} = C^{\prime} G_{r_x}^{\alpha}P_r^{\beta}$$

\begin{itemize}
\item $G_{r_x}\leq 10^9$ laminaire
\item $G_{r_x}\geq 10^9$ et $G_{r_x} \leq 10^{10}$ transitoire
\item $G_{r_x}\geq 10^{10}$ turbulent
\end{itemize}

Le nombre de Rayleigh:\\
Nombre de Rayleigh fait intervenir le terme moteur(Archim\`ede) et les deux ph\'enom\`enes de diffusion:le frottement visqueux $\nu$ mais aussi la conduction thermique $a$, ce qui est plus physique.
$$R_{a_x}(x)=G_{r_x}P_r=\frac{ g\beta(T_p - T_0)x^3}{\nu a}$$
$$N_{u_L} = C R_{a_L}^{\alpha'}P_r^{\beta'}$$
$$N_{u_x} = C^{\prime} R_{a_x}^{\alpha'}P_r^{\beta'}$$
Ces r\'esultats sont faiblement d\'ependants du nombre de Prandtl.

\subsection{Massique}
\begin{tabular}{|c|c|}
\hline
transfert de chaleur & transfert d'esp\`ece \\
\hline
$\varphi^{cc}= - \lambda \vec{\mbox{grad}}T$ & $q^{cc}= - D_{s/m} \vec{\mbox{grad}}c $\\
\hline
$\varphi^{cc}=h^{cc}(T_p - T_f)$ & $q^{cc}=h^{cc}(c_p - c)$\\
\hline
$h^{cc}$ coefficient de transfert conducto-convectif ($Wm^{-2}K^{-1}$) & $h^{cc} $($m/s$) \\
\hline
$R_{e_x}=\frac{ux}{\nu}$ et $R_{e_L}=\frac{ uL}{\nu}$ & $R_{e_x}=\frac{ux}{\nu}$ et $R_{e_L}=\frac{ uL}{\nu}$ \\
\hline
$P_r=\frac{ \nu}{a} $& $S_c=\frac{\nu}{D_{s/m}} $\\
\hline
$N_u$ & $S_h $\\
\hline
$\frac{ dH}{dt}=\varphi S$ & $\frac{dM}{dt}=qS$\\
\hline
\end{tabular}
\end{document}

