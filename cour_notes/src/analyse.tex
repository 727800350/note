\chapter{Analyse}

\textbf{COERVIT\'E}
Une fonction f d\'efinie sur un espace norm\'e X \`a valeurs dans $ \bar{\R}:=\R\cup\{-\infty,+\infty\}$ est dite coercive sur une partie non born\'ee $ P$ de $X$ si
$$ \lim_{\mytop{\| x\| \to +\infty}{x\in P}}f(x)=+\infty $$

ou de mani\`ere plus pr\'ecise
$$
\forall\,\nu\in\R,\quad \exists\,\rho\geqslant0:\quad (x\in X ~\mbox{et}~ \|x\|\geqslant\rho) \quad\Longrightarrow\quad f(x)\geqslant\nu.
$$

Il revient au m\^eme de dire que les intersections avec P des ensembles de sous-niveau de la fonction sont born\'ees :
$$
\forall\,\nu\in\R,\qquad\{x\in P: f(x)\leqslant\nu\}~\mbox{est born\'e.}
$$
Si l'on ne sp\'ecifie pas la partie P, il est sous-entendu que P=X.

\underline{Cas d'une forme bilin\'eaire} \newline
Plus sp\'ecifiquement, une forme bilin\'eaire $a:X\times X\to\R$ est dite coercive si elle v\'erifie :
$$
\exists\,\alpha>0,\quad\forall\,x\in X:\qquad a(x,x) \geqslant \alpha\|x\|^2.
$$
Certains auteurs pr\'ef\`erent utiliser l'appellation X-elliptique pour cette derni\`ere d\'efinition. Celle-ci intervient entre autres dans le th\'eor\`eme de Lax-Milgram et la th\'eorie des op\'erateurs elliptiques, accessoirement dans la m\'ethode des \'el\'ements finis.

\underline{Lien entre les d\'efinitions}\newline
Dans le cas où a est une forme bilin\'eaire, en posant $f(u)=a(u,u)$ on a \'equivalence entre la coercivit\'e de a et celle de $f$. En effet, $\scriptstyle\lim_{\| x\|\to\infty}f(x)=+\infty $implique qu'il existe $R>0 $tel que $\scriptstyle\|x\|\geqslant R\Rightarrow f(x)\geqslant 1$. Ainsi:
$$
\left(\frac{R}{\|u\|}\right)^2a(u,u)=a\left(\frac{R}{\|u\|}u,\frac{R}{\|u\|}u\right)=f\left(\frac{R}{\|u\|}u\right)\geqslant 1
$$
et
$$
a(u,u)\geqslant\left(\frac{\|u\|}{R}\right)^2.
$$
维基百科上没有直接给出coercivit\'e的定义, 但是按照维基百科的解释, 我觉得coercivit\'e的定义是
$$
 \alpha=\left(\frac{1}{R}\right)^2
$$

\bigskip
\textbf{Application contractante}\newline
Une application contractante, ou contraction, est une application k-lipschitzienne avec $0 \geq k \le 1$
\bigskip

\textbf{Th\'eor\`eme du point fixe pour une application contractante}\newline
Soient $ E$  un espace m\'etrique complet (non vide) et $ f$  une application $ k$ -contractante de $ E$  dans $ E$ . Il existe un point fixe unique $x^*$ de f (c'est-\`a-dire un $x^*$ dans $ E$  tel que $ f(x^* ) = x^*)$ . De plus, toute suite d'\'el\'ements de E v\'erifiant la r\'ecurrence
$$x_{n+1}=f(x_n)$$
v\'erifie la majoration
$$d(x_n,x^*) \le \frac {k^n}{1-k} d(x_0,x_1)$$
donc converge vers $x^*$
\bigskip

\textbf{In\'egalit\'e de Poincar\'e}
un r\'esultat de la th\'eorie des espaces de Sobolev\\
Cette in\'egalit\'e permet de borner une fonction \`a partir d'une estimation sur ses d\'eriv\'ees et de la g\'eom\'etrie de son domaine de d\'efinition\\
L'in\'egalit\'e de Poincar\'e classique
Soit $p$, tel que $ 1 \leq p < \infinity$  et $\Omega$ un ouvert de largeur finie (born\'e dans une direction). Alors il existe une constante  $  C$  , d\'ependant uniquement de $\Omega$ et $p$, telle que, pour toute fonction $u$   de l'espace de Sobolev $W_0^{1,p(\Omega)}$
$$
\| u \|_{L^{p} (\Omega)} \leq C \| \nabla u \|_{L^{p} (\Omega)}
$$

\section{算子(laplace, gradien etc.)}
\textbf{拉普拉斯算子: 梯度的散度}
$$\laplace f = \mbox{div} (\vec{\mbox{grad}}f)$$

Cart\'esien: $$\laplace f = \sum_{i=1}^n \frac{\partial^2 f }{\partial x_i^2} $$

函数的拉普拉斯算子也是该函数的黑塞矩阵的迹 $\laplace f = tr(H(f))$

极坐标下的拉普拉斯算子表示法
$$\laplace f = \frac{ 1}{\rho}\frac{\partial  }{\partial \rho}(\rho \frac{\partial  f}{\partial \rho}) + \frac{ 1}{\rho^2}\frac{\partial ^2 f}{\partial\theta^2} + \frac{\partial^2 f}{\partial z^2}$$

\bigskip
\textbf{Equation de poisson}
$$\laplace \varphi = f$$
si $f=0$, 那么泊松方程就会变成一个齐次方程, 称为拉普拉斯方程

\bigskip
\textbf{高斯}

高斯公式用散度表示为:
$$
\iiint_{\Omega}\mathrm{div}\mathbf{A}dv=
\int\!\!\!\!\int_{\Sigma}\!\!\!\!\!\!\!\!\!\!\!\!\!\!\;\;\;\bigcirc\,\,A_{n}dS
=
\int\!\!\!\!\int_{\Sigma}\!\!\!\!\!\!\!\!\!\!\!\!\!\!\;\;\;\bigcirc\,\,\mathbf{A}\cdot d\mathbf{S}
$$
其中Σ是空间闭区域Ω的边界曲面,而
$$
A_n=\mathbf{A}\cdot\mathbf{n}=P\cos\alpha+Q\cos\beta+R\cos\gamma
$$
$$
d\mathbf{S}=\mathbf{n}\cdot S
$$
$n$是向量$A$在曲面$\Sigma$的外侧法向量上的投影。

\bigskip
%\vskip 0.5cm
\textbf{斯托克斯公式}

$\mathbf{R}^3$上的斯托克斯公式
设$S$是分片光滑的有向曲面,$S$的边界为有向闭曲线$Γ$,即$\Gamma=\partial S$,且$Γ$的正向与$S$的侧符合右手规则: 函数$P(x,y,z),Q(x,y,z),R(x,y,z)$都是定义在"曲面$S$连同其边界$Γ$"上且都具有一阶连续偏导数的函数,则有
$$\iint\limits_{S}(\frac{\partial R}{\partial y}-\frac{\partial Q}{\partial z})dydz+(\frac{\partial P}{\partial z}-\frac{\partial R}{\partial x})dzdx+(\frac{\partial Q}{\partial x}-\frac{\partial P}{\partial y})dxdy
=\oint\limits_{\Gamma}Pdx+Qdy+Rdz$$

这个公式叫做$\mathbf{R}^3$上的斯托克斯公式或开尔文-斯托克斯定理、旋度定理。这和函数的旋度有关,用梯度算符可写成
$$
 \int_{S} \nabla \times \mathbf{F} \cdot d\mathbf{S} = \oint_{\partial S} \mathbf{F} \cdot d \mathbf{r}
$$

通过以下公式可以在"对坐标的"曲线积分""和对面积的"面积积分""之间相互转换:
$$
\iint\limits_{\Sigma}\begin{vmatrix} \cos \alpha & \cos \beta & \cos \gamma \\ \frac{\partial}{\partial x} & \frac{\partial}{\partial y} & \frac{\partial}{\partial z} \\ P & Q & R \end{vmatrix}dS=\oint\limits_{\Gamma}Pdx+Qdy+Rdz
$$

\bigskip
\textbf{格林公式}
设闭区域$D$由分段光滑的曲线$\partial D$($\partial D$是$D$取正向的边界曲线)围成,函数$P(x,y)$及$Q(x,y)$在$D$上具有一阶连续偏导数,则有
$$\oint_{\partial D} (Pdx+Qdy) = \iint_D (\frac{\partial Q}{\partial x} - \frac{\partial  P}{\partial y}dxdy)$$

\bigskip
\textbf{格林第一公式}
设函数$u(x,y,z)$和$v(x,y,z)$在闭区域$\Omega$上具有一阶及二阶连续偏导数,则有
$$\iiint_{\Omega} u\laplace v dxdydz
=
\oint\!\oint_{\Sigma} u \frac{\partial v}{\partial n}dS -
\iiint_{\Omega}(
\frac{\partial  u}{\partial x}\frac{\partial v}{\partial x}+
\frac{\partial  u}{\partial y}\frac{\partial v}{\partial y}+
\frac{\partial  u}{\partial z}\frac{\partial v}{\partial z}+
)dxdydz$$
其中$\Sigma$是闭区域$\Omega$的整个边界曲面,$\frac{\partial v}{\partial n}$为函数$v(x,y,z)$沿$\Sigma$的外法线方向的方向导数

更加简洁的写法:
$$
\int_{\Omega}u \laplace v d\Omega = \int_{\Sigma}u \frac{\partial v}{\partial n}d\Sigma - \int_{\Omega} \vec{\mbox{grad}}u \cdot \vec{\mbox{grad}} v d\Omega
$$
也可以用int\'egration par partie 来理解:
div是二次微分, grad是一次微分, div可以看成是在grad的基础上再来一次微分, 而
$ \frac{\partial  v}{\partial n} = <\vec{\mbox{grad}} v,\vec{n}>$, 所以
$\frac{\partial  v}{\partial n}$ 可以理解成一次微分

\bigskip
\textbf{格林第二公式}
设$u(x,y,z)$、$v(x,y,z)$是两个定义在闭区域$\Omega$上的具有二阶连续偏导数的函数,$\frac{\partial u}{\partial n},\frac{\partial v}{\partial n}$依次表示$u(x,y,z)$、$v(x,y,z)$沿$\Sigma$的外法线方向的方向导数,则有
$$\iiint\limits_{\Omega}(u\Delta v - v\Delta u)dxdydz=\oint\!\oint_{\Sigma}(u \frac{\partial v}{\partial n}-v\frac{\partial u}{\partial n})dS$$

\section{Autres}
Une matrice
$$
P=
\left(
             \begin{array}{ccc}
               x_1 & y_1 & 1 \\
               x_2 & y_2 & 1 \\
               x_3 & y_3 & 1 \\
             \end{array}
          \right)
$$
由 $(x_1,y_1)$, $(x_2,y_2)$, $(x_3,y_3)$ 三个点组成的三角形的面积:$surf = \frac{1}{2}\det{P}$
