\chapter{线性规划的对偶问题}
\href{http://course.cug.edu.cn/cugFirst/operational\_research/main/charpter2/p1.htm}{对偶}

\section{对偶的引出}
假设某厂计划生产甲,乙两种产品,其主要原材料有钢材3600kg,铜材2000kg及专用设备能力3000台时,已原材料和设备的单间消耗定额以及单位产品所获利润如下表所示\href{http://course.cug.edu.cn/cugFirst/operational\_research/main/charpter1/p1.files/image001.gif}{表1-1}
问如何安排生产方使该厂所获利润最大?

为了求解这一问题,设甲,乙两种产品的计划产量分别为$x_1, x_2$件.

这个问题的数学形式表达为
\begin{equation}
max\ z = 70x_1 + 120x_2
\mbox{ subject to }
\left\{
  \begin{array}{ll}
		 9x_1 + 4 x_2 \leq 3600; \\
		 4x_1 + 5 x_2 \leq 2000; \\
		 3x_1 + 54 x_2 \leq 3000; \\
		 x_1 \geq 0, x_2 \geq 0,
  \end{array}
\right.
\label{dual.original}
\end{equation}

现在,从另一个角度来考虑该问题,假设这家企业想将自己的原材料生产产品然后卖成品\textbf{改为直接卖原材料},
此时,工厂决策者必须考虑怎样为这三种资源定价的问题.
设$y_1, y_2, y_3$分别代表转让两种资源和出租设备的价格和租金.

定价的原则是:\textbf{直接卖原材料的获利不能低于卖成品}.\\
生产一个单位的甲产品需消耗9个单位的钢材,4个单位的铜材,3个单位的设备台时,获利70个单位,
那么,将这些资源全部转让时所获得的利润应不少于70个单位,即$9y_1 + 4y_2 + 3y_3 \geq 70$\\
同样的分析,有$4y_1 + 5y_2 + 10y_3 \geq 120$\\
此时,企业的总获利(即对方的总付出)为$W = 3600y_1 + 2000y_2 + 300y_3$

\textbf{为使对方容易接受},该厂只能在上面的两个约束条件下求$W$的最小值,即
\begin{equation}
min\ W = 3600y_1 + 2000y_2 + 300y_3
\mbox{ subject to }
\left\{
  \begin{array}{ll}
		9y_1 + 4y_2 + 3y_3 \geq 70 \\
		4y_1 + 5y_2 + 10y_3 \geq 120 \\
		y_1 \geq 0, y_2 \geq 0, y_3 \geq 0,
  \end{array}
\right.
\label{dual.dual}
\end{equation}

上述两个模型\eqref{dual.original}和\eqref{dual.dual}是对同一问题的两种不同考虑的数学描述,其间有着一定的内在联系,
我们对此进行比较分析,并从中找出规律,两个模型的对应关系有:
\begin{enumerate}
\item 两个问题的系数矩阵互为转置,
\item 一个问题的变量个数等于另一个问题的约束条件个数,
\item 一个问题的右端系数是另一个问题的目标函数的系数,
\item 一个问题的目标函数为极大化,约束条件为"$\leq$"类型,另一个问题的目标函数为极小化,约束条件为"$\geq$"
\end{enumerate}
我们把这种对应关系称为对偶关系,如果把式\eqref{dual.original}称为原问题,则式\eqref{dual.dual}称为对偶问题.

\section{Theory}
原问题与对偶问题在某种意义上来说,实质上是一样的,因为第二个问题仅仅在第一个问题的另一种表达而已

对偶问题的对偶是原问题

\subsection{对称}
当原问题对偶问题只含有不等式约束时,称为对称形式的对偶, 上面的形式即为对称的

原问题
\begin{equation}
max\ z = \ps{C}{X}
\mbox{ subject to }
\left\{
  \begin{array}{ll}
	  A \cdot X \leq b \\
		X \geq 0
  \end{array}
\right.
\label{dual.model.ori}
\end{equation}

对偶问题
\begin{equation}
min\ w = \ps{b}{Y}
\mbox{ subject to }
\left\{
  \begin{array}{ll}
		A^t \cdot Y \geq C \\
		Y \geq 0
  \end{array}
\right.
\label{dual.model.dual}
\end{equation}

当原问题为3个约束, 两个变量时:
$$X=(x_1, x_2)^t$$
$$A = A_{3 \times 2},\ A^t = A^t_{2 \times 3}$$
$$b = (b_1, b_2, b_3)^t$$
$$Y = (b_1, b_2, b_3)^t$$
$$C=(c_1, c_2)^t$$

\subsection{非对称}
若原问题的约束条件是等式, 称为非对称形式的对偶

原问题
\begin{equation}
max\ z = \ps{C}{X}
\mbox{ subject to }
\left\{
  \begin{array}{ll}
	  A \cdot X = b \\
		X \geq 0
  \end{array}
\right.
\end{equation}

对偶问题
\begin{equation}
min\ w = \ps{b}{Y}
\mbox{ subject to }
\left\{
  \begin{array}{ll}
		A^t \cdot Y \geq C \\
		Y \mbox{无约束}
  \end{array}
\right.
\end{equation}

\begin{proof}
$$
\mbox{原问题}
\Leftrightarrow
\left\{
  \begin{array}{ll}
	max\ z = \ps{C}{X} \\
	  A \cdot X \geq b \\
	  A \cdot X \leq b \\
		X \geq 0
  \end{array}
\right.
\Rightarrow
\left\{
  \begin{array}{ll}
	max\ z = \ps{C}{X} \\

    \begin{bmatrix}
        A \\
        -A \\
    \end{bmatrix}

    X \leq

    \begin{bmatrix}
        b \\
        -b \\
    \end{bmatrix} \\

		X \geq 0
  \end{array}
\right.
$$

根据对称形式的对偶模型,可直接写出上述问题的对偶问题
$$
\Rightarrow
\left\{
  \begin{array}{ll}
	min\ w = \ps{
    \begin{bmatrix}
        b \\
        -b \\
    \end{bmatrix}
	}{Y} \\

    \begin{bmatrix}
        A \\
        -A \\
    \end{bmatrix}^t

    Y \geq C \\

		Y \geq 0
  \end{array}
\right.
$$

$$
\mbox{设 }
Y =
    \begin{bmatrix}
        Y_1 \\
        Y_2 \\
    \end{bmatrix}
$$

$$
\Rightarrow
\left\{
  \begin{array}{ll}
	min\ w = \ps{b}{Y_1 - Y_2} \\
	A(Y_1 - Y_2) \geq C \\
	Y_1, Y_2 \geq 0
  \end{array}
\right.
$$
再令$Y = Y_1 - Y_2$(这个新的$Y$和之前的$Y$不同), 得到对偶问题为:
$$
\left\{
  \begin{array}{ll}
	min\ w = \ps{b}{Y} \\
	A(Y) \geq C \\
	Y \mbox{无约束}
  \end{array}
\right.
$$
\end{proof}

\subsection{Theorem}
\begin{theorem}
\textbf{弱对偶性定理}
若$\overline{X}$ 和 $\overline{Y}$ 分别是原问题\eqref{dual.model.ori}及对偶问题\eqref{dual.model.dual}的可行解,则有
$$ \ps{C}{\overline{X}} \leq \ps{b}{\overline{Y}} $$
\end{theorem}
\begin{proof}
$$
\left\{
  \begin{array}{l}
	  A \cdot \overline{X} \leq b \Rightarrow \overline{Y}^t \cdot A \cdot \overline{X} \leq \overline{Y}^t \cdot b \Rightarrow \ps{A\overline{X}}{\overline{Y}} \leq \ps{b}{\overline{Y}} \\
	  A^t \cdot \overline{Y} \geq C \Rightarrow \overline{X}^t \cdot A^t \cdot \overline{Y} \geq \overline{X}^t \cdot C \Rightarrow \ps{A\overline{X}}{\overline{Y}} \geq \ps{C}{\overline{X}}
  \end{array}
\right.
\Rightarrow
\ps{C}{\overline{X}} \leq \ps{b}{\overline{Y}}
$$
\end{proof}

从弱对偶性可得到以下重要结论:
\begin{enumerate}
\item 极大化问题(原问题)的任一可行解所对应的目标函数值是对偶问题最优目标函数值的下界.
\item 极小化问题(对偶问题)的任一可行解所对应的目标函数值是原问题最优目标函数值的上界.
\item 若原问题可行,但其目标函数值无界,则对偶问题无可行解.
\item 若对偶问题可行,但其目标函数值无界,则原问题无可行解.
\item 若原问题有可行解而其对偶问题无可行解,则原问题目标函数值无界.
\item 对偶问题有可行解而其原问题无可行解,则对偶问题的目标函数值无界.
\end{enumerate}

\begin{theorem}
\textbf{最优性定理}:
若$X^*$ 和 $Y^*$ 分别是原问题\eqref{dual.model.ori}及对偶问题\eqref{dual.model.dual}的可行解,且有$\ps{C}{X^*} = \ps{b}{Y^*}$,
则$X^*,Y^*$分别是\eqref{dual.model.ori}和\eqref{dual.model.dual}的最优解.
\end{theorem}

\begin{theorem}
\textbf{强对偶定理}:
若原问题及其对偶问题均具有可行解,则两者均具有最优解,且它们最优解的目标函数值相等.
\end{theorem}
\begin{proof}
原问题与对偶问题的解一般有三种情况:
\begin{enumerate}
\item 一个有有限最优解 $\Rightarrow$ 另一个有有限最优解.
\item 一个有无界解 $\Rightarrow$ 另一个无可行解.
\item 两个均无可行解.
\end{enumerate}
\end{proof}

\begin{theorem}
\textbf{(对偶定理}:
若线性规划问题\eqref{dual.model.ori}和\eqref{dual.model.dual}之一有最优解,则另一问题也有最优解,并且两者的目标函数值是相等的
\end{theorem}

\section{对偶问题的经济解释}
\href{http://course.cug.edu.cn/cugFirst/operational\_research/main/charpter2/p4.htm}{demo}

对于线性规划问题\eqref{dual.model.ori},用它来处理资源分配问题时,其决策变量代表的是产品的产量.它的对偶问题\eqref{dual.model.dual},其对偶变量也有明显的经济意义.
事实上,若$X^* = (x^*_1, x^*_2, \ldots, x^*_n)^t$
为原问题\eqref{dual.model.ori}的最优解,最优目标函数值为$z^*$.根据对偶定理,对偶问题也有最优解$Y^* = (y^*_1, y^*_2, \ldots, y^*_n)^t$.
且两者的最优目标函数值相等,即:
$$z^* = \ps{C}{X^*} = \ps{b}{Y^*} = w^*$$,
 
这也就是说,原问题的目标函数值等于$\sum_i b_i y^*_i$.由于这个和为各个$b_i y^*_i$相加而成,故可以将每个$b_i y^*_i$ 看成是第i种资源对目标函数值所做的贡献.
又由于 $b_i$ 是第i种资源的拥有量,因此, $\dfrac{b_i y^*_i}{b_i} = y^*_i$ 便可以理解为是每个单位第i种资源对目标函数值的贡献.
即:增加或减少单位第i种资源所引起总收益(目标函数)的改变量.

我们称 $y^*_i$ 为第i种资源的\textbf{影子价格}.显然这种价格不同于第i种资源的市场价格,它是由企业内部的条件决定的.\\
影子价格是一种边际价格, $\dfrac{\partial z^*}{b_i} = y^*_i$

同一种资源在不同的企业影子价格一般可以不同.一种资源的影子价格越大,则增加或减少一个单位这种资源,对总收益的影响越大,
如果一种资源的影子价格为零,则在一定范围内增加或减少一个单位这种资源对总收益没有影响.

