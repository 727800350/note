
\chapter{Logique et ensembles}
La relation $1+1 \neq 2$ est fausse,la double n\'egation est la relation $1+1=2$. Elle est vraie.
\newline
disjonction 或,析取 \newline
conjonction 和,合取 \newline
axiome 公理 \newline
conjecture 猜想\newline
le quantificateur existentiel $\exists$ \newline
le quantificateur universel $\forall$ \newline
\\
Un th\'eor\`eme est simplement une phrase vraie. \newline

%当且仅当
P si et seulement si Q: \newline
P si Q, c'est si Q, alors P.\quad $Q \Rightarrow P$ \newline
P seulement si Q: c'est si P, alors Q. \quad $P \Rightarrow Q$ \\
$P \Rightarrow (Q \Rightarrow R)\Leftrightarrow (P \Rightarrow Q)\rightarrow (P \Rightarrow R)$ \newline
La diff\'erence sym\'etrique de deux ensembles $E,F$.
C'est l'ensemble des \'el\'ements qui appartient \^a exactement un des ensembles $E,F$. On le note $E \Delta F$. \\
$
E \Delta F=\{x|x\in E ~\mathrm{et} ~x \not \in F\}\cup\{x|x\in F et x \not \in E\}=(E-F)\cup(F-E)
$

Une application est la m\^eme chose qu'une famille \`a valeurs dans $F$ index\'ee par $E$.
Si on note ${\cal F}(E,F)$ l'ensemble de toutes les applications de $E$ vers $F$, on a bien s\^ur
$$
{\cal F}(E,F)=F^E
$$

Si $E$ , $F$ sont finis, alors ${\cal F}(E,F)$ et $F^E$ sont finis, et
$$
|{\cal F}(E,F)|=|F^E|=|F|^|E|
$$

