% !Mode:: "TeX:UTF-8"
%\documentclass{article}
%    \usepackage[french]{babel}
    \usepackage{ctex}
    \usepackage{amsmath, amsthm, amssymb}
    \usepackage{graphicx}
   \usepackage[top=2.54cm, bottom=2.54cm, left=3.18cm, right=3.18cm]{geometry} % ms word
   %\usepackage[top=0.2cm, bottom=0.2cm, left=0cm, right=0cm,paperwidth=9cm,paperheight=11.7cm]{geometry} % kindle

%ֱ��ʹ��\lasteq ������һ������
\newcommand\lasteq{(\theequation)}    %导入需要用到的package
%\begin{document}
%% !Mode:: "TeX:UTF-8"
\documentclass{article}
% !Mode:: "TeX:UTF-8"
\usepackage[english]{babel}
\usepackage[UTF8]{ctex}
\usepackage{amsmath, amsthm, amssymb}

% Figure
\usepackage{graphicx}
\usepackage{float} %% H can fix the location
\usepackage{caption}
\usepackage[format=hang,singlelinecheck=0,font={sf,small},labelfont=bf]{subfig}
\usepackage[noabbrev]{cleveref}
\captionsetup[subfigure]{subrefformat=simple,labelformat=simple,listofformat=subsimple}
\renewcommand\thesubfigure{(\alph{subfigure})}

\usepackage{epstopdf} %% convert eps to pdf
\DeclareGraphicsExtensions{.eps,.mps,.pdf,.jpg,.png} %% bmp, gif not supported
\DeclareGraphicsRule{*}{eps}{*}{}
\graphicspath{{figure/}{../figure/}} %% fig directorys

%% \usepackage{pstricks} %% a set of macros that allow the inclusion of PostScript drawings directly inside TeX or LaTeX code
%% \usepackage{wrapfig} %% Wrapping text around figures

% Table
\usepackage{booktabs} %% allow the use of \toprule, \midrule, and \bottomrule
\usepackage{tabularx}
\usepackage{multirow}
\usepackage{colortbl}
\usepackage{longtable}
\usepackage{supertabular}

\usepackage[colorinlistoftodos]{todonotes}

% Geometry
\usepackage[paper=a4paper, top=1.5cm, bottom=1.5cm, left=1cm, right=1cm]{geometry}
%% \usepackage[paper=a4paper, top=2.54cm, bottom=2.54cm, left=3.18cm, right=3.18cm]{geometry} %% ms word
%% \usepackage[top=0.1cm, bottom=0.1cm, left=0.1cm, right=0.1cm, paperwidth=9cm, paperheight=11.7cm]{geometry} %% kindle

% Code
%% \usepackage{alltt} %% \textbf can be used in alltt, but not in verbatim

\usepackage{listings}
\lstset{
    backgroundcolor=\color{white},
    columns=flexible,
    breakatwhitespace=false,
    breaklines=true,
    captionpos=tt,
    frame=single, %% Frame: show a box around, possible values are: none|leftline|topline|bottomline|lines|single|shadowbox
    numbers=left, %% possible values are: left, right, none
    numbersep=5pt,
    showspaces=false,
    showstringspaces=false,
    showtabs=false,
    stepnumber=1, %% interval of lines to display the line number
    rulecolor=\color{black},
    tabsize=2,
    texcl=true,
    title=\lstname,
    escapeinside={\%*}{*)},
    extendedchars=false,
    mathescape=true,
    xleftmargin=3em,
    xrightmargin=3em,
    numberstyle=\color{gray},
    keywordstyle=\color{blue},
    commentstyle=\color{dkgreen},
    stringstyle=\color{mauve},
}

% Reference
%% \bibliographystyle{plain} % reference style

% Color
\usepackage[colorlinks, linkcolor=blue, anchorcolor=red, citecolor=green, CJKbookmarks=true]{hyperref}
\usepackage{color}
\def\red#1{\textcolor[rgb]{1.00,0.00,0.00}{#1}}
\newcommand\warning[1]{\red{#1}}

% Other
%% \usepackage{fixltx2e} %% for use of \textsubscript
%% \usepackage{dirtree}  %% directory structure, like the result of command tree in bash shell


% !Mode:: "TeX:UTF-8"
% Equation Number
\makeatletter\@addtoreset{equation}{subsection}\makeatother %% reset the equation number in subsection
\renewcommand\theequation{\thepart\arabic{section}-\thepart\arabic{subsection}-\thepart\arabic{equation}} %% section-subsection-equation style

% Theorem
\newtheorem{definition}{D\'efintion} %% document global number
\newtheorem*{thmwn}{Thm} %% without numbers
\newtheorem{theorem}{Th\'eor\`eme}[section] %% section-theorem style
\newtheorem{corollary}{Corollary}[theorem] %% theorem-corollary style
\newtheorem{lemma}{Lemma}
\newtheorem{proposition}{Proposition}[section]
\newtheorem*{attention}{Attention}
\newtheorem*{note}{Note}
\newtheorem*{remark}{Remark}
\newtheorem{example}{Example}
\newtheorem{question}{Question}[section]
\newtheorem{problem}{Problem}
\newtheorem*{answer}{Answer}
\newtheorem{fact}{Fact}

% Header and Footer
\newif\ifheader
\headerfalse
\ifheader
	\setlength\textheight{100.0pt}
	\setlength\textwidth{430.0pt}
	\usepackage{fancyhdr}
	\usepackage{lastpage} %% \pageref{LastPage}: get the last page
	\usepackage{pdfpages}
	\usepackage{layout}
	\footskip = 20pt
	\pagestyle{fancy}
	\fancyhead{} %% clear all fields
%% 	\newsavebox{\headpic}
%% 	\sbox{\headpic}{\includegraphics[height=1cm]{logo}} %% set header logo
%% 	\lhead{\usebox{\headpic}}
	\rhead{\small\leftmark}
	\fancyfoot{} %% clear all fields
	\cfoot{\thepage}
	\renewcommand{\headrulewidth}{1pt}  %% header line width, can be set 0 to get rid of it
	\setlength{\skip\footins}{0.5cm}    %% distance between of footnote and body text
	\renewcommand{\footnotesize}{}      %% footnote size
\fi


\begin{document}
\title{Math\'ematique}
\author{Common}
\maketitle
\tableofcontents
\newpage
\section{Error measure}
MAPE(Mean absolute percentage error)
$$ MAPE = \dfrac{1}{n} \sum_{i=1}^n \dfrac{\norm{\estimate{y}_i - y_i}}{y_i} \cdot 100\%$$

MSE(Mean square error)
$$MSE = \dfrac{1}{n} \sum_{i=1}^n (\estimate{y}_i - y_i)^2$$

RMSE(Root mean square error)
$$RMSE  = \sqrt{MSE}$$

CE(Cummulative Error)
$$CE = \sum_{i=1}^n (\estimate{y}_i - y_i)$$

ACE(Absolute Cummulative Error)
$$ACE = \sum_{i=1}^n \norm{\estimate{y}_i - y_i}$$

VAF(Variance Accounting For)
$$VAF = (1 - \dfrac{\var(y - \estimate{y})}{\var(y)})$$
\end{document}
  %导入数学方面的定义
\section{力学中的变分原理及其应用}
$$
\text{力学原理}
\left\{
	\begin{array}{ll}
	 \text{非变分的}
	 	\left\{
			\begin{array}{ll}
                    \text{微分的(如达朗贝尔原理)}\\
                    \text{积分的(如能量守恒原理)}
			\end{array}
			\right.  \\
	 \text{变分的}
	 	\left\{
			\begin{array}{ll}
                    \text{微分的(如虚位移原理)}\\
                    \text{积分的(如哈密顿原理)}
			\end{array}
			\right.
	\end{array}
	\right.
$$

\begin{equation}
\text{应变与位移的关系}
\left\{
	\begin{array}{ll}
	\varepsilon_x = \frac{\partial u}{\partial x} \\
	\varepsilon_y = \frac{\partial v}{\partial y} \\
	\varepsilon_z = \frac{\partial w}{\partial z} \\
    \gamma_{xy} = \gamma_{yx} = \frac{\partial u}{\partial y}+ \frac{\partial v}{\partial x}\\
    \gamma_{yz} = \gamma_{zy} = \frac{\partial v}{\partial z}+ \frac{\partial w}{\partial y}\\
    \gamma_{zx} = \gamma_{xz} = \frac{\partial w}{\partial x}+ \frac{\partial u}{\partial z}\\
	\end{array}
	\right.
\label{equation.cauchy}
\end{equation}

\begin{equation}
\text{应变张量}
[\varepsilon_{ij}]=
\left[
  \begin{array}{ccc}
  \varepsilon_{xx} & \varepsilon_{xy} & \varepsilon_{xz} \\
  \varepsilon_{yx} & \varepsilon_{yy} & \varepsilon_{yz} \\
  \varepsilon_{zz} & \varepsilon_{zz} & \varepsilon_{zz}
  \end{array}
\right]
=
\left[
  \begin{array}{ccc}
  \varepsilon_{x} & \frac{1}{2}\gamma_{xy} & \frac{1}{2}\gamma_{xz} \\
  \frac{1}{2}\gamma_{yx} & \varepsilon_{y} & \frac{1}{2}\gamma_{yz} \\
  \frac{1}{2}\gamma_{zz} & \frac{1}{2}\gamma_{zz} & \varepsilon_{z}
  \end{array}
\right]
\end{equation}

\begin{equation}
 \varepsilon_{ij}=\frac{1}{2}(u_{i,j}+u_{j,i})
\end{equation}

由柯西方程(\ref{equation.cauchy})可得
\begin{equation}
  \left\{
	\begin{array}{ll}
	\frac{\partial^2 \varepsilon_x }{\partial y^2} + \frac{\partial^2 \varepsilon_y }{\partial x^2} = \frac{\partial^2 \gamma_{xy}}{\partial x\partial y} , \frac{\partial}{\partial x}(- \frac{\partial \gamma_{yz}}{\partial x} +\frac{\partial \gamma_{zx}}{\partial y} +\frac{\partial \gamma_{xy}}{\partial z} ) = 2 \frac{\partial^2 \varepsilon_x}{\partial y\partial z}
	\\\\
	\frac{\partial^2 \varepsilon_y }{\partial z^2} + \frac{\partial^2 \varepsilon_z }{\partial y^2} = \frac{\partial^2 \gamma_{yz}}{\partial y\partial z} , \frac{\partial}{\partial y}(- \frac{\partial \gamma_{zx}}{\partial y} +\frac{\partial \gamma_{xy}}{\partial z} +\frac{\partial \gamma_{yz}}{\partial x} ) = 2 \frac{\partial^2 \varepsilon_y}{\partial z\partial x}
\\\\
	\frac{\partial^2 \varepsilon_z }{\partial x^2} + \frac{\partial^2 \varepsilon_x }{\partial z^2} = \frac{\partial^2 \gamma_{zx}}{\partial z\partial x} , \frac{\partial}{\partial z}(- \frac{\partial \gamma_{xy}}{\partial z} +\frac{\partial \gamma_{yz}}{\partial x} +\frac{\partial \gamma_{zx}}{\partial y} ) = 2 \frac{\partial^2 \varepsilon_z}{\partial x\partial y}
	\end{array}
	\right.
\label{equation.compatibility}
\end{equation}
(\ref{equation.compatibility})称为应变协调方程或相容性方程\\

\textbf{平衡方程—六个应力分量的三个平衡方程}
\textbf{几何方程—6个应变分量与3个位移分量}

由六个应变分量求解三个位移分量,其方程个数多于未知数个数,方程组要么矛盾,要么相关。由于变形连续,弹性体任意一点的变形必须受到其相邻单元体变形的约束。
——应变协调方程—反映应变分量之间的关系
六个应变分量必须满足一定的条件\\

从几何方程中消去位移分量,第一式和第二式分别对y和 x求二阶偏导数,然后相加,可得:
$$
\frac{\partial^2 \varepsilon_x }{\partial y^2} + \frac{\partial^2 \varepsilon_y }{\partial x^2} =
\frac{\partial^2}{\partial x \partial y}(\frac{\partial v}{\partial x}+\frac{\partial u}{\partial y})=
\frac{\partial^2 \gamma_{xy}}{\partial x\partial y}
$$

将几何方程的四,五,六式分别对z,x,y,求一阶偏导数,前后两式相加并减去中间一式,则
$$
- \frac{\partial \gamma_{yz}}{\partial x} +\frac{\partial \gamma_{zx}}{\partial y} +\frac{\partial \gamma_{xy}}{\partial z}
=2 \frac{\partial ^2 u}{\partial y \partial z}
$$
对x求一阶偏导数,则
$$
\frac{\partial}{\partial x}(- \frac{\partial \gamma_{yz}}{\partial x} +\frac{\partial \gamma_{zx}}{\partial y} +\frac{\partial \gamma_{xy}}{\partial z} ) = 2 \frac{\partial^2 \varepsilon_x}{\partial y\partial z}
$$
分别轮换x,y,z,则可得如下六个关系式

\textbf{变形协调方程的数学意义}
使3个位移为未知函数的六个几何方程不相矛盾

\textbf{变形协调方程的物理意义}
物体变形后每一单元体都发生形状改变,如变形不满足一定的关系,变形后的单元体将不能重新组合成连续体,其间将产生缝隙或嵌入现象。
为使变形后的物体保持连续体,应变分量必须满足一定的关系。

\begin{equation}
\text{应力张量}
\sigma=
[\sigma_{ij}]=
\left[
  \begin{array}{ccc}
  \sigma_{11} & \sigma_{12} & \sigma_{13} \\
  \sigma_{21} & \sigma_{22} & \sigma_{23} \\
  \sigma_{31} & \sigma_{32} & \sigma_{33}
  \end{array}
\right]
=
\left[
  \begin{array}{ccc}
  \sigma_{xx} & \sigma_{xy} & \sigma_{xz} \\
  \sigma_{yx} & \sigma_{yy} & \sigma_{yz} \\
  \sigma_{zz} & \sigma_{zz} & \sigma_{zz}
  \end{array}
\right]
=
\left[
  \begin{array}{ccc}
  \sigma_{x} & \tau_{xy} & \tau_{xz} \\
  \tau_{yx} & \sigma_{y} & \tau_{yz} \\
  \tau_{zz} & \tau_{zz} & \sigma_{z}
  \end{array}
\right]
\end{equation}
应力张量是对称张量

\subsection{常用的应变能公式}
设L为直杆长度,A为直杆面积,E为直杆材料弹性模量,在直杆为线弹性的情况下,根据材料力学的理论,下列应变能公式成立\\
\textbf{拉伸(伸缩)应变能}
\begin{equation}
 U=\frac{P\Delta L}{2}=\frac{\sigma \varepsilon}{2}AL=\frac{EAL\varepsilon^2}{2}=\frac{P^2L}{2EA}
\end{equation}
\begin{equation}
 U=\frac{1}{2}\int_{0}^{L}\frac{P^2}{EA}dx=\frac{1}{2}\int_{0}^{L}EA(\frac{du}{dx})^2dx
\end{equation}
式中,P为轴向力,$\Delta L$为直杆变形量,u为相应x长度轴向位移量,$\sigma,\varepsilon$分别为直杆的应力和应变

\textbf{弯曲应变能}
\begin{equation}
 U=\frac{1}{2}\int_{0}^{L}\frac{M^2}{EI}dx=\frac{1}{2}\int_{0}^{L}EIk^2dx
\end{equation}
式中,M为弯矩,k是曲率,EI为横截面抗弯刚度,I为截面惯性矩\\
根据高等数学,曲率的表达式为
\begin{equation}
 k=|\frac{y''}{(1+y')^{\frac{3}{2}}}|
\end{equation}
在小变形的情况下,$y''=0$,于是弯曲应变能可改写成
\begin{equation}
 U=\frac{1}{2}\int_{0}^{L}\frac{M^2}{EI}dx=\frac{1}{2}\int_{0}^{L}EIy''^2dx
\end{equation}

\textbf{剪切应变能}
\begin{equation}
 U=\frac{1}{2}\int_{0}^{L}\frac{kQ^2}{GA}dx=\frac{1}{2}\int_{0}^{L}\frac{GA\gamma^2}{k}dx
\end{equation}
式中,Q为剪力,$\gamma$为剪应变,k为截面剪应力分布有关的系数,$GA/k$为截面抗剪刚度

\textbf{园轴扭转应变能}
\begin{equation}
 U=\frac{M_t^2 L}{2GJ_p}=\frac{GJ_p\phi^2}{2L}
\end{equation}
\begin{equation}
 U=\frac{1}{2}\int_{0}^{L}\frac{M_t^2}{GJ_p}dx=\frac{1}{2}\int_{0}^{L}GJ_p(\frac{d\phi^2}{dx})^2dx
\end{equation}
式中$M_t$为扭矩,$\phi$为扭转角,$GJ_p$为圆轴截面抗扭刚度,$J_p$为截面积惯性矩

\subsection{虚位移原理}
\subsubsection{弹性体的广义虚位移原理}
设弹性体在直角坐标系中的体积为V,表面积为S,取一微元体dV,所受单位体积的体积力为 $\overline{F_x} \overline{F_y}\mbox{和} \overline{F_z} $ ,记作$\overline{F_i}(i=1,2,3)$,\emph{字母上加一横线表示这个量已给定},单位面积的表面力为X,Y,Z 记作$X_i$。以
$ \sigma_x, \sigma_y, \sigma_z, \tau_{yz},\tau_{zx},\tau_{xy} $
表示一点处的应力分量,记作$\sigma_{ij}$,当弹性体在各力作用下处于平 衡(或运动)状态时,根据弹性力学理论,有下式成立
\begin{equation}
  \left\{
	\begin{array}{ll}
	 \frac{\partial \sigma_x}{\partial x}+ \frac{\partial \tau_{xy}}{\partial y}+ \frac{\partial \tau_{zx}}{\partial z}+ \overline{F_x}=0 & (\mbox{或}\rho\frac{\partial ^2u}{\partial t^2}) \\\\
	 \frac{\partial \sigma_y}{\partial y}+ \frac{\partial \tau_{yz}}{\partial z}+ \frac{\partial \tau_{xy}}{\partial x}+ \overline{F_y}=0 & (\mbox{或}\rho\frac{\partial ^2v}{\partial t^2}) \\\\
	 \frac{\partial \sigma_z}{\partial z}+ \frac{\partial \tau_{zx}}{\partial x}+ \frac{\partial \tau_{yz}}{\partial y}+ \overline{F_z}=0 & (\mbox{或}\rho\frac{\partial ^2w}{\partial t^2})
	\end{array}
	\right.
\label{equation.move.equilibre}
\end{equation}
式中,$\rho$表來单位体积的质量即密度,$\frac{\partial ^2u}{\partial t^2},\frac{\partial ^2v}{\partial t^2},\frac{\partial ^2w}{\partial t^2}$表示微元体积在坐标$x,y,z$方向的加速度分量,它们与密度乘积表示在三个坐标方向的负的单位体积惯性力,其中$u,v,w$为三个坐标方向的位移分量,
记作$u_i$。一般情况下,这些变量都是坐标和时间的函数。
当加速度分量等于零时,式(\ref{equation.move.equilibre})称为弹性体的\textbf{平衡微分方程},简称平衡方程;当加速度分
量不等于零时,式(\ref{equation.move.equilibre})称为弹性体的\textbf{运动微分方程},简称运动方程。
\\
若引用爱因斯坦求和约定,则式(\ref{equation.move.equilibre})可合并成一个方程
\begin{equation}
 \frac{\partial \sigma_{ij}}{\partial x_j} + \overline{F_i}=0 ~(\mbox{或}\rho u_{itt})
\end{equation}
式中,$u_{itt}=\frac{\partial ^2 u_i}{\partial t^2}$
利用逗号约定,上式可以更简单的写成如下形式
\begin{equation}
 \sigma_{ij,j} + \overline{F_i}=0 (\mbox{或}\rho u_{itt})
\end{equation}
设惯性体的边界为$S=S_u+S_{\sigma}$,其中$S_u$称为\textbf{位移边界},在该边界上给定位移
\begin{equation}
u_i=\overline{u}_i
\label{equation.limit.displacement}
\end{equation}
因$S_u$上位移已给定,故在$S_u$上位移的变分为零
$S_\sigma$称为\textbf{应力边界},在该边界上给定表面力,即它满足力学边界条件
\begin{equation}
X=\overline{X},
Y=\overline{Y},
Z=\overline{Z}
\label{equation.limit.contraint}
\end{equation}
其中$X_i=n_j \sigma_{ij}$
如果弹性体的位移满足物体内部的连续性条件即几何方程(\ref{equation.cauchy})和$S_u$上的位移边界条件 (\ref{equation.limit.displacement}) ,则称为弹性体的可能位移或容许位移,记作$u_i^p$与可能位移相对应的应变称为可能应变 或容许应变,记作$\varepsilon_{ij}^p$

如果弹性体的应力满足平衡微分方程(\ref{equation.move.equilibre})和$S_{\sigma}$上的应力边界条件(\ref{equation.limit.contraint}),则称为弹
性体的可能应力或容许应力,记作 因为可能应力状态是与给定体积力和表面力(或惯性力)相
平衡的应力状态,所以满足平衡(运动)微分方程和应力边界条件,故有
\begin{equation}
 \sigma_{ij,j}^p + \overline{F_i}=0 ~ (ou ~\rho u_{itt},dans V)
 \label{equation.equilibre.differentiel}
\end{equation}
\begin{equation}
 X_i=\overline{X}_i=n_j \sigma_{ij}^p  ~ (sur ~S_{\sigma})
\end{equation}
将平衡方程(\ref{equation.equilibre.differentiel})乘以可能位移,并在整个体积V上积分,可得
\begin{equation}
 \iiint_V(\sigma_{ij,j}^p + \overline{F_i})u_i^p dV=0
 \label{equation.integration}
\end{equation}
将上式中的第一项先进行分部积分,在应用高斯公式,然后利用式,可得
\begin{eqnarray}
    \iiint_V\sigma_{ij,j}^p u_i^p dV & = & \iiint_V(\sigma_{ij}^p u_{i}^p)_{,j}dV - \iiint_V \sigma_{ij}^p u_{i,j}^p dV \\
                                    &  =  & \iint_S n_j(\sigma_{ij}^p u_{i}^p)dS - \iiint_V \sigma_{ij}^p u_{i,j}^p dV \\
                                    &  =  &\iint_S (\overline{X}_i u_i^pdS) - \iiint_V \sigma_{ij}^p u_{i,j}^p dV
\label{equation.process}
\end{eqnarray}
注意到$\sigma_{ij}^p=\sigma_{ji}^p$,经过验证可知
\begin{equation}
\sigma_{ij}^p u_{i,j}^p = \sigma_{ij}^p \varepsilon_{ij}^p
\end{equation}
将上式代入(\ref{equation.process}),再将(\ref{equation.process})代入(\ref{equation.integration})可得
\begin{equation}
 \iiint_V \overline{F}_i u_i^p dV + \iint_S \overline{X}_i u_i^p dS = \iiint_V \sigma_{ij}^p \varepsilon_{ij}^p dV
 \label{equation.theory.virtual.displacement}
\end{equation}
(\ref{equation.theory.virtual.displacement})式称为弹性体的\textbf{广义虚位移原理},广义虚功原理或可能功原理
因广义虚位移原理的推导过程并未涉及本构关系,故式(\ref{equation.theory.virtual.displacement})中的可能应力$\sigma_{ij}^p$
和可能应变$\varepsilon_{ij}^p$ 可以互不相干,它们分别在各自容许的条件下独立变化。

广义虚功原理可叙述为:\textbf{外力(体积力和表面力)在可能位移上做的功等
于静力可能应力在与可能位移相应的可能应变上做的功。}广义虚功原理是能量守恒原理在弹性力学中的一个具体表现形式。

\subsection{弹性体的虚位移原理}
\begin{equation}
 \iiint_V \overline{F}_i u_i dV + \iint_S \overline{X}_i u_i dS = \iiint_V \sigma_{ij} \varepsilon_{ij} dV
\end{equation}
弹性体的虚位移原理可表述为:
\textbf{弹性体处于平衡状态的充要条件是:对于任意微小的虚位移,作用在弹性体上的外力(体力和面力)在任意虚位移过程中所做的虚功等于弹
性体的虚应变能。}
\\
值得指出的是,虽然上面从物体弹性平衡的角度导出了虚位移原理的表达式,但是一般说来,虚位移原理具有普遍意义。它可以适用于一切结构,不论材料是线性还是非线性、弹性还是塑性、静载还是动载等均能适用

\subsection{最小势能原理}
弹性体在平衡位置时,其总势能有极值。研究结果表明,在稳定的平衡位置弹 性体的势能具有极小值。于是,对于位移和变形都很小的弹性体,最小势能原理可以叙述为:弹性体 在给定的外力作用下,在满足变形相容条件和位移边界条件的所有可能位移中,真实位移使弹性体 总势能取得极小值。根据最小势能原理,可以把求位移微分方程的边值问题转化为求总势能泛函的变分问题。求出了弹性体的位移,就可以求得应力,以分析弹性体的强度。

\subsection{哈密顿原理及其应用}
\subsubsection{质点系的哈密顿原理}
设有n个质点组成的非自由质点系,其中第i个质点$P_i$所受的合力可分为两类:主动力$F_i$和约束反力$R_i$. 若约束反力在系统的任意虚位移上所作元功之和恒等于零,即
\begin{equation}
\sum_{i=1}^n R_i \cdot \delta r_i =0
\end{equation}
则这种约束称为理想约束. 式中$R_i$表示作用于系统中任意一质点$i$上的约束反力,$\delta r_i$表示该质点的任意虚位移.
\\
设$n$个质点组成的系统受到理约束并处于运动状态,其中第$i$个质点$P_i$, 所受的主动力的合 力为$F_i$,约束反力的合力为$R_i$,质量为$m_i$,具有加速度$a_i$,且在初始时刻$t_0$和纟f 了时刻$t_l$,系统的 正路和旁路在同一位置上。根据牛顿第二定律,在任一瞬时,有
\begin{equation}
 m_i a_i = F_i + R_i  ~(i=1,2,\ldots,n)
\end{equation}
上式可改写成
\begin{equation}
   F_i- m_i a_i + R_i =0 ~(i=1,2,\ldots,n)
\end{equation}
上式表明,在任一瞬时,作用于质点系内每个质点的主动力$F_i$,约束反力$R_i$和惯性力$-m_ia_i$构成平衡力系,这称为质点系的\textbf{达朗贝尔原理}。
在此瞬时,给系统以任意虚位移$\delta r_i ~ (i=1,2,\ldots,n)$并求和,因系统受理想约束,故有
\begin{equation}
 \sum_{i=1}^n(F_i - m_i a_i)\cdot \delta r_i =\sum_{i=1}^nF_i \cdot \delta r_i +\sum_{i=1}^n(- m_i a_i)\cdot \delta r_i=0
 \label{equation.dynamic.general}
\end{equation}

式 (\ref{equation.dynamic.general})
称为\textbf{动力学普遍方程},或称为\textbf{达朗贝尔一拉格朗日方程},有时也称为达朗贝尔原 理的拉格朗日形式。该方程可表述为:\textbf{具有理想约束的质点系运动时,在任意时刻,主动力和惯性力 在任意虚位移上所做的元功之和为零。}全部动力学的定理和方程都可由动力学普遍方程推导出来。显然,在动力学普遍方程中,理想约束的约束反力没有出现。

式 (\ref{equation.dynamic.general})
第一项可写成
\begin{equation}
 \delta 'W=\sum_{i=1}^n F_i \cdot \delta r_i
 \label{eq.travail.virtual}
\end{equation}
式中,$\delta'W$为给定力系在虚位移上所作的元功即虚功。需要指出是,第$i$个质点受到的主动力的合力$F_i$可能依赖于$r_i,\dot{r}_i,t$,
但虚功表达式中并不包含带有$\dot{r}_i$的项。因此虚功$\delta 'W$ 一般情况下并不表示总功$W$的变分,也就是说,在得到力的功的表达式以后,一般不能由该表达式求变分而得到该力的虚功。只是数量积$\sum_{i=1}^n F_i \cdot \delta r_i$的简写,它并不一定是$W$的变分。
\\
惯性力的虚功
$
-m_ia_i\cdot \delta r_i
$
可改写成
\begin{equation}
-m_ia_i\cdot \delta r_i=-m_i \frac{dv_i}{dt} \cdot \delta r_i = - \frac{d}{dt}(m_i v_i \cdot \delta r_i) + m_i v_i \cdot \frac{d}{dt}\delta r_i
\label{equation.travail.virtual}
\end{equation}
由于微分和变分可交换次序,故有
\begin{equation}
 \frac{d}{dt}\delta r_i =\delta \frac{dr_i}{dt} =\delta v_i
 \label{equation.change}
\end{equation}
将(\ref{equation.change}) 代入\eqref{equation.travail.virtual}并求和,得
\begin{equation}
 \begin{split}
  \sum_{i=1}^n (-m_ia_i \delta r_i)& = -\frac{d}{dt}(\sum_{i=1}^n m_i v_i \cdot \delta r_i) + \sum_{i=1}^n m_i v_i\cdot \delta v_i\\
  & = -\frac{d}{dt}(\sum_{i=1}^n m_i v_i \cdot \delta r_i) + \sum_{i=1}^n \delta (\frac{m_i v_i \cdot v_i}{2}) \\
  & = -\frac{d}{dt}(\sum_{i=1}^n m_i v_i \cdot \delta r_i) + \delta T
 \end{split}
\end{equation}
式中,$T=\sum_{i=1}^n \frac{m_i v_i \cdot v_i}{2}$成为系统的总动能
\\
把式\eqref{eq.travail.virtual}和\lasteq 代入\eqref{equation.dynamic.general},得
\begin{equation}
 \delta T + \delta 'W =\frac{d}{dt}\sum_{i=1}^n (m_i v_i \cdot \delta r_i)
\end{equation}
对式 \lasteq 从$t_0$至$t_1$时刻积分,并注意到当
$t=t_0$ 和 $t=t_1$
时正路和旁路占有相同的位置 $M_0$ 和 $M_1$ ,即
$ \delta r_i|_{t=t_0}= \delta r_i|_{t=t_1}= 0 $,则
\begin{equation}
 \int_{t_0}^{t_1}(\delta T + \delta ' W)dt=\int_{t_0}^{t_1}\frac{d}{dt}(\sum_{i=1}^n m_i v_i \cdot \delta r_i)dt=\sum_{i=1}^n m_i v_i \cdot \delta r_i|_{t=t_0}^{t=t_1}=0
\end{equation}
或
\begin{equation}
 \int_{t_0}^{t_1}(\delta T + \delta ' W)dt=0
\end{equation}
式 \lasteq 称为\textbf{哈密顿原理的广义形式}

当$\delta 'W$恰为某个函数的变分$\delta W$, \lasteq 可改写成
\begin{equation}
 \delta \int_{t_0}^{t_1}(T+W)dt=0
\end{equation}

当主动力为有势力时,有$\delta W=-\delta V$,这里$V$是系统的势能,一般情况下,他只是系统位置坐标的单值连续函数,也称为\textbf{势能函数或势函数}.故
\begin{equation}
 \delta T + \delta W =\delta T -\delta V =\delta (T-V)=\delta L
\end{equation}
式中,\textbf{$L=T-V$称为拉格朗日函数}.于是得到下面的哈密顿原理
\textbf{哈密顿原理}:对于任何有势力作用下的完整系统的质点系,在给定始点$t_0$和终点$t_1$的状态后,其真实运动与任何容许运动的区别是真是运动使泛函
\begin{equation}
 J=\int_{t_0}^{t_1}(T-V)dt=\int_{t_0}^{t_1}Ldt
\end{equation}
达到极值,即
\begin{equation}
 \delta J=\delta \int_{t_0}^{t_1}Ldt=0
\end{equation}

哈密顿原理虽未指明真实路径使泛函取极大值还是极小值,但一般情况下,哈密顿原理所涉及的泛函在真实路径上都是取极小值。
哈密顿原理又称为稳定作用量原理或最小作(量)用原理。\emph{该原理是力学中的基本原理,与动 力学普遍方程等价,它把力学原理化为更一般的形式,并且与坐标系的选择无关,在理论上具有意 义的普遍性,在应用上具有广泛的适应性.}哈密顿原理只涉及两个动力学函数,即系统的动能和势能。 若把T、V 和L 分别看作质点系在时刻$t$的动能密度(即单位体积的动能)、势能密度和拉格朗日密度函数,则哈密顿原理可写成如下形式
\begin{equation}
 \delta J=\delta \int_{t_0}^{t_1} \iiint_V (T-V)dVdt=\delta \int_{t_0}^{t_1} \iiint_V LdVdt =0
\end{equation}
式中,微分号下的V是质点系所占据的空间域。

%\end{document}
