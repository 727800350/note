\chapter{Espace vectoriel}
\section{基坐标与坐标变换}
\begin{question}
设$V$为域$F$上的$n$维线性空间, 给定$V$的两个基$\alpha, \beta$.
$\alpha = (\alpha_1, \alpha_2, \ldots, \alpha_n)$,
$\beta = (\beta_1, \beta_2, \ldots, \beta_n)$.
设$V$中向量$v$在这两个基底的坐标分别为$X = (x_1, x_2, \ldots, x_n)^t, Y = (y_1, y_2, \ldots, y_n)^t$. \\
则$X$与$Y$之间有什么样的关系?
\end{question}
\begin{answer}
$\alpha$ 为$V$的一个基底, 因此有
$$
\begin{cases}
\begin{aligned}
\beta_1 & = a_{11} \alpha_1 + a_{21} \alpha_2 + \cdots + a_{n1} \alpha_n \\
\beta_2 & = a_{12} \alpha_1 + a_{22} \alpha_2 + \cdots + a_{n2} \alpha_n \\
		\vdots \\
\beta_n & = a_{1n} \alpha_1 + a_{2n} \alpha_2 + \cdots + a_{nn} \alpha_n \\
\end{aligned}
\end{cases}
$$
$$
\Rightarrow
(\beta_1, \beta_2, \ldots, \beta_n) = (\alpha_1, \alpha_2, \ldots, \alpha_n)
\begin{bmatrix}
a_{11} & a_{12}  & \cdots & a_{1n} \\
a_{21} & a_{22}  & \cdots & a_{2n} \\
	\vdots \\
a_{n1} & a_{n2}  & \cdots & a_{nn} \\
\end{bmatrix}
$$
最后一个大矩阵记为$P_{\alpha \rightarrow \beta}$, 意思是从$\alpha$基底变换到$\beta$基底.
$$\Rightarrow (\beta_1, \beta_2, \ldots, \beta_n) = (\alpha_1, \alpha_2, \ldots, \alpha_n) \cdot P_{\alpha \rightarrow \beta} $$

$$v = (\alpha_1, \alpha_2, \ldots, \alpha_n) \cdot X$$
$$v = (\beta_1, \beta_2, \ldots, \beta_n) \cdot Y = (\alpha_1, \alpha_2, \ldots, \alpha_n) \cdot P_{\alpha \rightarrow \beta} \cdot Y $$
$$\Rightarrow X = P_{\alpha \rightarrow \beta} Y$$
\end{answer}

\begin{question}
设$\alpha = (\alpha_1, \alpha_2, \ldots, \alpha_n)$为$V$的一个基底, 且另一向量组$\beta = (\beta_1, \beta_2, \ldots, \beta_n)$满足
$(\beta_1, \beta_2, \ldots, \beta_n) = (\alpha_1, \alpha_2, \ldots, \alpha_n) \cdot P$.\\
证明: $\beta = (\beta_1, \beta_2, \ldots, \beta_n)$为$V$的一个基底的充要条件是$P$ inversible.
\end{question}
\begin{proof}
$\beta = (\beta_1, \beta_2, \ldots, \beta_n)$ 为$V$的一个base \\
$\Leftrightarrow$ 从 $k_1 \beta_1 + k_2 \beta_2 + \cdots + k_n \beta_n = 0$ 可以推出 $k_1 = k_2 = \cdots = k_n = 0$ \\
$\Leftrightarrow$ 从 $(\beta_1, \beta_2, \ldots, \beta_n) \begin{bmatrix}k_1 \\ k_2 \\ \vdots \\ k_n\end{bmatrix} = 0$
	可以推出$\begin{bmatrix}k_1 \\ k_2 \\ \vdots \\ k_n\end{bmatrix} = 0$ \\
$\Leftrightarrow$ 从 $(\alpha_1, \alpha_2, \ldots, \alpha_n) \cdot P \cdot (k_1, k_2, \cdots, k_n)^t = 0$ 可以推出$(k_1, k_2, \cdots, k_n)^t = 0$ \\
$\Leftrightarrow$ 从 $P \cdot (k_1, k_2, \cdots, k_n)^t = 0$ 可以推出$(k_1, k_2, \cdots, k_n)^t = 0$  \eqnote{因为$\alpha$为一个base}\\
$\Leftrightarrow$ $Px = 0$只有零解 \\
$\Leftrightarrow$ $detP = 0$ \\
$\Leftrightarrow$ $P$ inversible
\end{proof}

\section{子空间的直和}
\begin{theorem}
  On a une d\'ecomposition en somme directe de deux sous-espaces $ V = W_1 \oplus W_2$  si et seulment si
  $$
    \begin{cases}
    V=W_1+W_2 \\
    W_1\cap W_2=\{0\}
    \end{cases}
  $$
\end{theorem}
\begin{attention}
  Si le nombre de sous-espaces est $n\geqslant 3$, on peut plus dire que
  $$
    \begin{cases}
    V=W_1+\dots+W_n \\
    \forall i \neq j,W_i\cap W_j=\{0\}
    \end{cases}
  $$
  implique $ V = W_1 \oplus \dots \oplus W_n$
\end{attention}

