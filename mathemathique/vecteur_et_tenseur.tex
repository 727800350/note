% !Mode:: "TeX:UTF-8"
\chapter{矢量和张量}
\section{置换符号$\varepsilon$}
行列式
$$
det(a_{ij})=\varepsilon_{rst}a_{r1}a_{s2}a_{t3}
$$
矢量差乘
\begin{equation}
\begin{split}
u \times v &= \varepsilon_{rst}u_s v_t e_r \\
			&= (u_2 v_3 - u_3 v_2)e_1 + (u_3 v_1 - u_1 v_3)e_2 + (u_1 v_2 - u_2 v_1)e_3
\end{split}
\end{equation}

\subsection{$\varepsilon-\delta$ 恒等式}
$$
\varepsilon_{ijk}\varepsilon_{ist}=\delta_{js}\varepsilon_{kt}-\delta_{ks}\varepsilon_{jt}
$$
\section{坐标变换}
两个具有同一坐标原点O的右手笛卡尔直角坐标系
$x_1,x_2,x_3$ and $x'_1,x'_2,x'_3$
用x表示P的点的矢径,其分量分别为
$x_1,x_2,x_3$ and $x'_1,x'_2,x'_3$.
设
$e_1,e_2,e_3$ and $e'_1,e'_2,e'_3$
分别表示他们的基向量
$$
x= x_1\cdot e_1+ x_2\cdot e_2+ x_3\cdot e_3=x_j\cdot e_j
$$
$$
x=x_j \cdot e_j=x'_j \cdot e'_j
$$
用$e_i$同时点乘方程两边得到:
$$
x_j(e_j\cdot e_i)=x'_j(e'_j\cdot e_i)
$$
but we have
$$
x_j(e_j\cdot e_i)=x_j \delta_{ji} =x_i
$$
so $$x_i=(e'_j\cdot e_i)x'_j$$
now we define
$$(e'_j * e_i)\equiv \beta_{ji}$$
so we have:$$x_i=\beta_{ji}x'_j ~(j=1,2,3)$$
de la m\^eme facon, on a $$x'_i=\beta_{ij}x_j ~(i=1,2,3)$$

\section{张量运算}
$$\overrightarrow{ \overrightarrow{ a}}+\overrightarrow{ \overrightarrow{ b}}\Leftrightarrow a_{ij}+b_{ij}$$

Le produit de tenseurs est un tenseur. La convention d'Einstein s'applique en notation indicielle.
En notation vectorielle, \textbf{chaque indice muet est repr\'esent\'e par le symbole} $\bullet$.
$$\lambda \overrightarrow{ \overrightarrow{ a}} \Leftrightarrow \lambda a_ij$$

$$
\overrightarrow{ c}=\overrightarrow{ \overrightarrow{ a}}\bullet \overrightarrow{ b} \Leftrightarrow a_{ij}b_j
$$
一个点,所以只有一个indice muet
$$
\overrightarrow{ \overrightarrow{ c}}=\overrightarrow{ a}\overrightarrow{ b}
\Leftrightarrow c_{ij}=a_ib_j
$$

$$
\overrightarrow{ \overrightarrow{ a}} \bullet\bullet \overrightarrow{ \overrightarrow{ b}}=a_{ij}b_{ij}~~
\text{两个bullet是上下叠放的}
$$
两个点,所以只有两个indice muet

\textbf{Contraction d'un tenseur} \newline
Etant donn\'e un tenseur d'ordre 2 ou plus, on peut choisir deux de ses indices libres et appliquer
une contraction. Les lettres associ\'ees aux deux indices sont chang\'ees pour les rendre identiques.
Exemples:
$$
a_{ij} \rightarrow a_{ii}, a_{ijk}\rightarrow a_{ijj} ,a_ib_j\rightarrow a_ib_i
$$
cette op\'eration rend les indices concern\'es muets et r\'eduit l'ordre du tenseur de n \`a n - 2.
dans le cas d'un tenseur d'ordre 2, il n'y a qu'une seule contraction possible, mais si l'ordre est sup\'erieur \`a 2, plusieurs choix des indices de contraction sont possibles.

\section{张量变换}
\begin{equation}
\left\{
		\begin{array}{ll}
		\overline{t}_{ij}(\overline{x}_1,\overline{x}_2,\overline{x}_3)=t_{mn}(x_1,x_2,x_3)\beta_{im}\beta{jn}\\
        t_{ij}(x_1,x_2,x_3)=\overline{t}_{mn}(\overline{x}_1,\overline{x}_2,\overline{x}_3)\beta_{mi}\beta{nj}
		\end{array}
		\right.
\end{equation}

\section{grandien, divergence etc}
$\lambda$ est un scalaire:
$$\overrightarrow{\text{grad}}\lambda=\frac{\partial \lambda }{\partial x_i}$$

$\overrightarrow{a}$ est un vecteur:
$$\overrightarrow{a}.\overrightarrow{\text{grad}}=a_i \cdot \frac{ \partial}{\partial x_i}$$

$$
(\overrightarrow{ a}\cdot \overrightarrow{ \text{grad}})\overrightarrow{ b}=a_j \cdot \frac{\partial b_i}{\partial x_j}\cdot e_i
$$

$$\text{div}\overrightarrow{a}=\frac{\partial a_i}{\partial x_i}$$
$$\overrightarrow{\text{rot}}\overrightarrow{a}=\varepsilon_{ijk} \cdot \frac{\partial a_k}{\partial x_j} \cdot \overrightarrow{e}_i$$
$$
\bigtriangleup \lambda =\text{div}(\overrightarrow{\text{grad}}\lambda)=\frac{\partial ^2 \lambda }{\partial x_i\partial x_i}
$$

$$
\bigtriangleup \overrightarrow{a} =\text{div}(\overrightarrow{\text{grad}}\lambda)=\frac{\partial ^2 a_i }{\partial x_j\partial x_j}
$$

$$
\overrightarrow{b}=\text{div}\overrightarrow{ \overrightarrow{a}}\Leftrightarrow b_i=\frac{\partial a_{ij}}{\partial x_j}
$$

