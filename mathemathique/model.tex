% !Mode:: "TeX:UTF-8"
\documentclass{article}
% !Mode:: "TeX:UTF-8"
\usepackage[english]{babel}
\usepackage[UTF8]{ctex}
\usepackage{amsmath, amsthm, amssymb}

% Figure
\usepackage{graphicx}
\usepackage{float} %% H can fix the location
\usepackage{caption}
\usepackage[format=hang,singlelinecheck=0,font={sf,small},labelfont=bf]{subfig}
\usepackage[noabbrev]{cleveref}
\captionsetup[subfigure]{subrefformat=simple,labelformat=simple,listofformat=subsimple}
\renewcommand\thesubfigure{(\alph{subfigure})}

\usepackage{epstopdf} %% convert eps to pdf
\DeclareGraphicsExtensions{.eps,.mps,.pdf,.jpg,.png} %% bmp, gif not supported
\DeclareGraphicsRule{*}{eps}{*}{}
\graphicspath{{figure/}{../figure/}} %% fig directorys

%% \usepackage{pstricks} %% a set of macros that allow the inclusion of PostScript drawings directly inside TeX or LaTeX code
%% \usepackage{wrapfig} %% Wrapping text around figures

% Table
\usepackage{booktabs} %% allow the use of \toprule, \midrule, and \bottomrule
\usepackage{tabularx}
\usepackage{multirow}
\usepackage{colortbl}
\usepackage{longtable}
\usepackage{supertabular}

\usepackage[colorinlistoftodos]{todonotes}

% Geometry
\usepackage[paper=a4paper, top=1.5cm, bottom=1.5cm, left=1cm, right=1cm]{geometry}
%% \usepackage[paper=a4paper, top=2.54cm, bottom=2.54cm, left=3.18cm, right=3.18cm]{geometry} %% ms word
%% \usepackage[top=0.1cm, bottom=0.1cm, left=0.1cm, right=0.1cm, paperwidth=9cm, paperheight=11.7cm]{geometry} %% kindle

% Code
%% \usepackage{alltt} %% \textbf can be used in alltt, but not in verbatim

\usepackage{listings}
\lstset{
    backgroundcolor=\color{white},
    columns=flexible,
    breakatwhitespace=false,
    breaklines=true,
    captionpos=tt,
    frame=single, %% Frame: show a box around, possible values are: none|leftline|topline|bottomline|lines|single|shadowbox
    numbers=left, %% possible values are: left, right, none
    numbersep=5pt,
    showspaces=false,
    showstringspaces=false,
    showtabs=false,
    stepnumber=1, %% interval of lines to display the line number
    rulecolor=\color{black},
    tabsize=2,
    texcl=true,
    title=\lstname,
    escapeinside={\%*}{*)},
    extendedchars=false,
    mathescape=true,
    xleftmargin=3em,
    xrightmargin=3em,
    numberstyle=\color{gray},
    keywordstyle=\color{blue},
    commentstyle=\color{dkgreen},
    stringstyle=\color{mauve},
}

% Reference
%% \bibliographystyle{plain} % reference style

% Color
\usepackage[colorlinks, linkcolor=blue, anchorcolor=red, citecolor=green, CJKbookmarks=true]{hyperref}
\usepackage{color}
\def\red#1{\textcolor[rgb]{1.00,0.00,0.00}{#1}}
\newcommand\warning[1]{\red{#1}}

% Other
%% \usepackage{fixltx2e} %% for use of \textsubscript
%% \usepackage{dirtree}  %% directory structure, like the result of command tree in bash shell

   %导入需要用到的package
% !Mode:: "TeX:UTF-8"
% Equation Number
\makeatletter\@addtoreset{equation}{subsection}\makeatother %% reset the equation number in subsection
\renewcommand\theequation{\thepart\arabic{section}-\thepart\arabic{subsection}-\thepart\arabic{equation}} %% section-subsection-equation style

% Theorem
\newtheorem{definition}{D\'efintion} %% document global number
\newtheorem*{thmwn}{Thm} %% without numbers
\newtheorem{theorem}{Th\'eor\`eme}[section] %% section-theorem style
\newtheorem{corollary}{Corollary}[theorem] %% theorem-corollary style
\newtheorem{lemma}{Lemma}
\newtheorem{proposition}{Proposition}[section]
\newtheorem*{attention}{Attention}
\newtheorem*{note}{Note}
\newtheorem*{remark}{Remark}
\newtheorem{example}{Example}
\newtheorem{question}{Question}[section]
\newtheorem{problem}{Problem}
\newtheorem*{answer}{Answer}
\newtheorem{fact}{Fact}

% Header and Footer
\newif\ifheader
\headerfalse
\ifheader
	\setlength\textheight{100.0pt}
	\setlength\textwidth{430.0pt}
	\usepackage{fancyhdr}
	\usepackage{lastpage} %% \pageref{LastPage}: get the last page
	\usepackage{pdfpages}
	\usepackage{layout}
	\footskip = 20pt
	\pagestyle{fancy}
	\fancyhead{} %% clear all fields
%% 	\newsavebox{\headpic}
%% 	\sbox{\headpic}{\includegraphics[height=1cm]{logo}} %% set header logo
%% 	\lhead{\usebox{\headpic}}
	\rhead{\small\leftmark}
	\fancyfoot{} %% clear all fields
	\cfoot{\thepage}
	\renewcommand{\headrulewidth}{1pt}  %% header line width, can be set 0 to get rid of it
	\setlength{\skip\footins}{0.5cm}    %% distance between of footnote and body text
	\renewcommand{\footnotesize}{}      %% footnote size
\fi

   %导入需要用到的package

\begin{document}
\title{Models}
\author{Commun}
\maketitle
\tableofcontents
\newpage

\section{排队模型}
M/M/1排队模型(M/M/1 model)是一种单一服务器(single-server)的(排队模型),可用作模拟不少系统的运作.
依据开恩特罗符号必须有下列的条件:
\begin{itemize}
\item 到达时间服从泊松分布(Poisson distribution)
\item 服务时间是指数分布(exponentially distributed)
\item 只有一部服务器(server)
\item 队列长度无限制
\item 可加入队列的人数为无限
\end{itemize}
\href{http://upload.wikimedia.org/wikipedia/commons/thumb/6/65/Mm1\_queue.svg/292px-Mm1\_queue.svg.png}{Mm1}

这种模型是一种出生-死亡过程,此随机过程中的每一个状态代表模型中人数的数目.因为模型的队列长度无限且参与人数亦无限,故此状态数目亦为无限.
例如状态0表示模型闲置,状态1表示模型有一人在接受服务,状态2表示模型有二人(一人正接受服务,一人在等候),
如此类推. 此模型中,出生率(即加入队列的速率)$\lambda$ 在各状态中均相同,死亡率(即完成服务离开队列的速率)$\mu$亦在各状态中相同(除了状态0,因其不可能有人离开队列).

故此,在任何状态下,只有两种事情可能发生:
\begin{itemize}
\item 有人加入队列.如果模型在状态k,它会以速率$\lambda$ 进入状态$k + 1$
\item 有人离开队列.如果模型在状态k(k不等于0),它会以速率$\mu$进入状态$k ? 1$
\end{itemize}

\href{http://upload.wikimedia.org/wikipedia/commons/thumb/e/e6/MM1\_queue\_state\_space.svg/605px-MM1\_queue\_state\_space.svg.png}{State space transmission}

由此可见,模型的隐定条件为$\lambda < \mu$.如果死亡率小于出生率,则队列中的平均人数为无限大,故此这种系统没有平衡点.

此模型中有几项数值常被测量,例如:
\begin{itemize}
\item 一人在系统中的平均逗留时间
\item 一人在接受服务前的平均等候时间
\item 整个系统中的平均人数
\item 等候队列的平均人数
\item 一单位时间内系统完成服务人数,即服务速度
\end{itemize}

稳定状态下的公式

定义 $\scriptstyle \rho \,=\,{\tfrac  {\lambda }{\mu }}$.
则模型在状态i的机率为
$$ {\mbox{Prob}}(q=i)=\pi _{i}=(1-\rho )\rho ^{i}.\, $$
由此,可给出各测量数值的公式:
整个系统的平均人数N:$\overline N={\frac  {\rho }{1-\rho }}$,
且其变异(variance)为$\sigma _{N}^{2}={\frac  {\rho }{(1-\rho )^{2}}}$.
一单位时间内系统完成服务的人数:$\overline N_{S}=\lambda \overline x=\rho$
在队列中等候服务的人数:$\overline N_{Q}={\frac  {\rho ^{2}}{1-\rho }}$
一人在系统中的平均逗留(等候+接受服务)时间:$T={\frac  {1}{\mu -\lambda }}$.
一人的平均等候时间:$W={\frac  {\overline N_{Q}}{\lambda }}=T-\overline x=T-{\frac  {1}{\mu }}={\frac  {\rho }{\mu -\lambda }}$

\begin{example}
可用M/M/1模型的例子众多,例如只有一位员工的邮局,只有一队列.客人进来,排队,接受服务,离开.
如果客人进来的数目符合泊松过程,且服务时间是指数分布,则可用M/M/1模拟,并算出平均队列长度,不同等候时间的机率等.

M/M/1可一般化成为M/M/n模型,使可用时接受服务的人数为大于一.
历史上,M/M/n模型首先被用来模拟电话系统,因为荷兰工程师Erlang发现客人打电话的速率符合泊松过程,且通话时间是指数分布,
所以占用通讯线路的数目和等待接线的人数符合M/M/n模型.
\end{example}
\end{document}
