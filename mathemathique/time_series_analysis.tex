% !Mode:: "TeX:UTF-8"
\documentclass{article}
% !Mode:: "TeX:UTF-8"
\usepackage[english]{babel}
\usepackage[UTF8]{ctex}
\usepackage{amsmath, amsthm, amssymb}

% Figure
\usepackage{graphicx}
\usepackage{float} %% H can fix the location
\usepackage{caption}
\usepackage[format=hang,singlelinecheck=0,font={sf,small},labelfont=bf]{subfig}
\usepackage[noabbrev]{cleveref}
\captionsetup[subfigure]{subrefformat=simple,labelformat=simple,listofformat=subsimple}
\renewcommand\thesubfigure{(\alph{subfigure})}

\usepackage{epstopdf} %% convert eps to pdf
\DeclareGraphicsExtensions{.eps,.mps,.pdf,.jpg,.png} %% bmp, gif not supported
\DeclareGraphicsRule{*}{eps}{*}{}
\graphicspath{{img/}{figure/}{../figure/}} %% fig directorys

%% \usepackage{pstricks} %% a set of macros that allow the inclusion of PostScript drawings directly inside TeX or LaTeX code
%% \usepackage{wrapfig} %% Wrapping text around figures

% Table
\usepackage{booktabs} %% allow the use of \toprule, \midrule, and \bottomrule
\usepackage{tabularx}
\usepackage{multirow}
\usepackage{colortbl}
\usepackage{longtable}
\usepackage{supertabular}

\usepackage[colorinlistoftodos]{todonotes}

% Geometry
\usepackage[paper=a4paper, top=1.5cm, bottom=1.5cm, left=1cm, right=1cm]{geometry}
%% \usepackage[paper=a4paper, top=2.54cm, bottom=2.54cm, left=3.18cm, right=3.18cm]{geometry} %% ms word
%% \usepackage[top=0.1cm, bottom=0.1cm, left=0.1cm, right=0.1cm, paperwidth=9cm, paperheight=11.7cm]{geometry} %% kindle

% Code
%% \usepackage{alltt} %% \textbf can be used in alltt, but not in verbatim

\usepackage{listings}
\lstset{
    backgroundcolor=\color{white},
    columns=flexible,
    breakatwhitespace=false,
    breaklines=true,
    captionpos=tt,
    frame=single, %% Frame: show a box around, possible values are: none|leftline|topline|bottomline|lines|single|shadowbox
    numbers=left, %% possible values are: left, right, none
    numbersep=5pt,
    showspaces=false,
    showstringspaces=false,
    showtabs=false,
    stepnumber=1, %% interval of lines to display the line number
    rulecolor=\color{black},
    tabsize=2,
    texcl=true,
    title=\lstname,
    escapeinside={\%*}{*)},
    extendedchars=false,
    mathescape=true,
    xleftmargin=3em,
    xrightmargin=3em,
    numberstyle=\color{gray},
    keywordstyle=\color{blue},
    commentstyle=\color{green},
    stringstyle=\color{red},
}

% Reference
%% \bibliographystyle{plain} % reference style

% Color
\usepackage[colorlinks, linkcolor=blue, anchorcolor=red, citecolor=green, CJKbookmarks=true]{hyperref}
\usepackage{color}
\def\red#1{\textcolor[rgb]{1.00,0.00,0.00}{#1}}
\newcommand\warning[1]{\red{#1}}

% Other
%% \usepackage{fixltx2e} %% for use of \textsubscript
%% \usepackage{dirtree}  %% directory structure, like the result of command tree in bash shell


% !Mode:: "TeX:UTF-8"
%+++++++++++++++++++++++++++++++++++article+++++++++++++++++++++++++++++++++
%customize the numbering of equation, to make it like section-subsection-equation style, for example,1-2-3
\makeatletter\@addtoreset{equation}{subsection}\makeatother
\renewcommand\theequation{%
\thepart\arabic{section}%
-\thepart\arabic{subsection}%
-\thepart\arabic{equation}%
}
%theorem
\newtheorem{definition}{D\'efintion} %% 整篇文章的全局编号
\newtheorem*{thmwn}{Thm} %% without numbers
\newtheorem{theorem}{Th\'eor\`eme}[section] %% 从属于section编号
\newtheorem{corollary}{Corollary}[theorem] %% 从属于theorem编号
\newtheorem{lemma}{Lemma}
\newtheorem{proposition}{Proposition}[section]
\newtheorem{example}{Example}
\newtheorem*{attention}{Attention}
\newtheorem*{note}{Note}
\newtheorem*{remark}{Remark}
\newtheorem{question}{Question}[section]
\newtheorem{problem}{Problem}
\newtheorem{fact}{Fact}


\begin{document}
\title{Time Series Analysis: Forecasting and Control}
\author{}
\maketitle
\tableofcontents
\newpage

Time Series Analysis: Forecasting and Control
作者: George E. P. Box / Gwilym M. Jenkins / Gregory C. Reinsel
\href{http://book.douban.com/subject/4159347/}{豆瓣}

\section{Introduction}
\subsection{Four important practical problems}
\begin{enumerate}
\item{Forecasting Time series}\\
Forcasts are usually needed by a period known as the \textbf{lead time}(A lead time is the latency (delay)
between the initiation and execution of a process).

$\hat{z}_t(l)$ forecast function at origin $t$.\\
$\hat{z}_t(l)$ is the forecast of $z_{t+l}$

\textbf{Assumption}: The time series $z_t$ follows a stochastic model of a known form.

\item{Estimation of Transfer Functions}\\
eg: the dynamic relationship between between two time series $Y_t$ and $X)t$ can be determined, past values of \textbf{both series} may be used in
forecasting $Y_t$

\item {Analysis of Effects of Unusual Intervention Events to a System}\\
intervention events: indicator variable taking only 1 and 0 to indicate(quantitatively) the presence or absence of the event.

\item {Discrete Control Systems}\\
When we can measure fluctuations in an input variable that can be observed but not changed, it may be possible to make appropriate compensatory changes
in some other control variable. This is referred to as \textbf{feedforward control}.

Alternatively, we may be able to use the deviation from target or "error signal" of the output characteristic itself to calculate appropriate
compensatory changes in the control variable. This is called \textbf{feedback control}.
\end{enumerate}

\subsection{Stochastic Mathematical Models}
\begin{description}
\item [backward shift operator $B$] $Bz_t = z_{t-1}$, hence $B^m z_t = z_{t-m}$
\item [forward shift operator $F$] $Fz_t = z_{t+1}$, $F = B^{-1}$
\item [backward difference operator $\nabla$] $\nabla z_t = z_t - z_{t-1} = (1-B)z_t$
\end{description}

\textbf{Linear filter model}\\
$$z_t = \mu + a_t + \psi_1 a_{t-1} + \psi_2 a_{t-2} + \cdots = \mu + \psi(B) a_t$$
$\mu$: level of the process, and
$$\psi(B) = 1 + \psi_1 B + \psi_2 B^2 + \cdots$$

\section{Simple Exponential Smoothing(SES)}
Widely used class of procedures for smoothing discrete time series in order to forecast the immediate future.\\
Simple exponential smoothing model is only good for non-seasonal patterns with approximately zero-trend and for short-term forecasting

Let an observed time series be $\vector{y}$.the simple exponential smoothing equation takes the form of:
$$\estimate{y}_{i+1} = \alpha y_i + (1 - \alpha) \estimate{y_i} $$
where $y_i$ is the actual, known series value for time period $i$, $\estimate{y_i}$ is the forecast value of the variable $Y$ for time period $i$.
$\alpha$ is the smoothing constant.

Rewriting the model to see one of the neat things about the SES model
$$\estimate{y}_{i+1} - \estimate{y_i} = \alpha (y_i - \estimate{y_i})$$
可以看到: change in forecasting value is proportionate to the forecast error.\\
So exponential smoothing forecast is the old forecast plus an adjustment for the error that occurred in the last forecast
$$
\begin{aligned}
\estimate{y}_{i+1}
& = \alpha y_i + (1 - \alpha) [\alpha y_{i-1} + (1-\alpha)\estimate{y}_{i-1}] = \alpha y_i + \alpha (1 - \alpha) y_{i-1} + (1-\alpha)^2 \estimate{y}_{i-1}\\
& = \alpha y_i + \alpha (1 - \alpha) y_{i-1} + \alpha (1-\alpha)^2 y_{i-2} + (1-\alpha)^3 \estimate{y}_{i-2} \\
& = \alpha y_i + \alpha (1 - \alpha) y_{i-1} + \alpha (1-\alpha)^2 y_{i-2} + \alpha (1-\alpha)^3 y_{i-3} + (1-\alpha)^4 \estimate{y}_{i-3} \\
& \ldots \\
& = \alpha y_i + \alpha (1 - \alpha) y_{i-1} + \alpha (1-\alpha)^2 y_{i-2} + \cdots + \alpha (1-\alpha)^{i-2} y_2 + \alpha (1-\alpha)^{i-1} y_1 \\
& = \alpha \sum_{k=0}^{i-1} (1-\alpha)^k y_{i-k}
\end{aligned}
$$
Therefore, $\estimate{y}_{i+1}$ is the weighted moving average of all past observations. \\
The weights are $\alpha, \alpha(1-\alpha), \alpha(1-\alpha)^2, \ldots $.
These \textbf{weights decline toward zero in an exponential fashion}.
\end{document}
